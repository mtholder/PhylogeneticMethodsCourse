\documentclass[landscape]{foils} 
\input{../common-preamble-start}
\input{../preamble.tex}
\usepackage{url}
\usepackage{hyperref}
\hypersetup{backref,  linkcolor=blue, citecolor=red, colorlinks=true, hyperindex=true}

\usepackage{pdfpages}
\usepackage{bm}
\usepackage{ifsym}
\newcommand{\hilite}[1]{{\color{red} \bf #1}}
\newcommand{\disruption}{\theta}
\begin{document}
\pagecolor{white}
\unitlength=1mm
\begin{center}
{\Large Some of these slides have been borrowed from Dr.\ Paul Lewis, Dr.\ Joe Felsenstein. Thanks!}
\vskip 15mm
\large Paul has many great tools for teaching phylogenetics at his web site: \\
\url{http://hydrodictyon.eeb.uconn.edu/people/plewis}
\end{center}




\myNewSlide
\includepdf[pages={5-22}]{../nonfreeimages/pol/pol-ancstates.pdf}

\myNewSlide
\section*{Estimation of where changes take place}
\large
\begin{compactenum}
	\item What is the probability that a change $0\rightarrow1$ took place on a particular branch in the tree?
	\item What is the probability that the character changed state exactly $x$ times on the tree ?  
\end{compactenum}

Should we answer these questions with a joint or marginal ancestral character state reconstruction?

\myNewSlide
\Huge
{\center{No}}


\myNewSlide
\large
We should figure out the probability of explaining the character state distribution with exactly the set of changes we are interested in.

Marginalize over all states that are not crucial to the question

\myNewSlide
\begin{compactenum}
	\item What is the probability that a change $0\rightarrow1$ took place on a particular branch in the tree? -- fix the ancestral node to 0 and the descendant to 1, calculate the likelihood and divide that by the total likelihood.
	\item {\color{grey}What is the probability that the character changed state exactly $x$ times on the tree ?  }
\end{compactenum}

\myNewSlide
\begin{compactenum}
	\item {\color{grey}What is the probability that a change $0\rightarrow1$ took place on a particular branch in the tree?}
	\item What is the probability that the character changed state exactly $x$ times on the tree ?  -- Sum the likelihoods over all reconstructions that have exactly $x$ changes.  This is not easy to do!  Recall that $\Pr(0\rightarrow 1|\nu,\theta)$ from our models accounts for multiple hits.  A solution is to sample histories according to their posterior probabilities and summarize these samples.
\end{compactenum}


\myNewSlide
\includepdf[pages={5-16}]{../nonfreeimages/pol/pol-mapping.pdf}

\myNewSlide
\section*{Sojourn times are exponentially distributed}
\large
If $\lambda$ is the total rate of leaving state $0$, then you can simulate a soujourn 
by:
\begin{compactitem}
	\item drawing a time according to the cumulative distribution function of the exponential distribution ($1 - e^{-\lambda t}$).
	\item pick the ``destination'' state by normalizing the rates away from state 0 such that the rates sum to one.
\end{compactitem}

This is picking $j$ and $t$ by:
\[\Pr(0\rightarrow j, t) = \Pr(\mbox{mut. at }t)\Pr(j|\mbox{mut.})\]
But we want quantities like:
\[\Pr(0\rightarrow j, t|0\rightarrow 2, \nu_1)\]

\myNewSlide
\section*{Conditioning on the data}
We simulate according to $\Pr(0\rightarrow j, t|0\rightarrow 2, \nu_1)$ by rejection sampling:
\begin{compactenum}
	\item Simulate a series of draws of the form: $\Pr(0\rightarrow j, t)$ until the sum of $t$ exceeds $\nu_1$,
	\item If we ended up in the correct state, then accept this series of sojourns as the mapping.
	\item If not, go back to step 1.
\end{compactenum}



\myNewSlide
\includepdf[pages={20-22}]{../nonfreeimages/pol/pol-mapping.pdf}

\myNewSlide
\section*{``Stochastic character mapping'' or ``Mutational Mapping''}
\begin{compactitem}
	\item allows you to sample from the posterior distribution of histories for characters
	\item conditional on an assumed model
	\item conditional on the observed data
	\item implemented in Simmap (v1 by J. Bollback; v2 by J. Bollback and J. Yu) and Mesquite (module by P. Midford)
\end{compactitem}

\myNewSlide
\section*{``Stochastic character mapping'' }
You can calculate summary statistics on the set of simulated histories to answer almost any type of question that depends on knowing ancestral character states or the number of changes:
\begin{compactitem}
	\item What proportion of the evolutionary history was `spent' in state 0?
	\item What is the posterior probability that there were between 2 and 4 changes of state?
	\item Does the state in one character appear to affect the probability of change at another? -- Map both characters and do a contingency analysis.
\end{compactitem}
	
See \citet{MininS2008MB,MininS2008PRSoc,ObrienMS2009} for rejection free approaches to generate mappings.

\myNewSlide

Often we are more interested in the ``forces'' driving evolutionary patterns than the outcome.

Perhaps we should focus on fitting a model of character evolution rather than the ancestral character state estimates.

\myNewSlide
Rather than this:
\begin{compactenum}
	\item estimate a tree,
	\item estimate character states,
	\item figure out what model of character evolution best fits those ancestral character state reconstructions.
\end{compactenum}

We could do this:
\begin{compactenum}
	\item estimate a tree,
	\item figure out what model of character evolution best fits tip data
\end{compactenum}

Or even this:
\begin{compactenum}
	\item Jointly estimate character evolution in trees in MCMC (marginalizing over trees when answering questions about character evolution).
\end{compactenum}



\myNewSlide
\includepdf[pages={35}]{../nonfreeimages/joe/joe-compmethods.pdf}

\myNewSlide
\includepdf[pages={2-3}]{../nonfreeimages/pol/pol-mapping.pdf}

\myNewSlide
\section*{Model-based approach to detecting correlated evolution}
View the problem as model-selection or use LRT if you prefer the hypothesis testing framework.

Consider two binary characters. Under the simplest model, there is only one rate:
\begin{table}[htdp]
\begin{center}
\begin{tabular}{lr|c|c|c|c|}
& & 0 & 1 & 2 & 3 \\
& & (0,0) & (0,1) & (1,0) & (1,1) \\
\hline
0 & (0,0) & - & $q$ & $q$ & 0 \\
1 & (0,1) & $q$ & -  & 0 & $q$ \\
2 & (1,0) & $q$ & 0 & - & $q$  \\
3 & (1,1) & 0 & $q$ &  $q$ & - \\
\end{tabular}
\end{center}
\label{default}
\end{table}%

Why are there 0's?
\myNewSlide
Under a model assuming character independence:
\begin{table}[htdp]
\begin{center}
\begin{tabular}{lr|c|c|c|c|}
& & 0 & 1 & 2 & 3 \\
& & (0,0) & (0,1) & (1,0) & (1,1) \\
\hline
0 & (0,0) & - & $f_{01}$ & $s_{01}$ & 0 \\
1 & (0,1) & $q_{\ast0}$ & -  & 0 & $q_{1\ast}$ \\
2 & (1,0) & $q_{0\ast}$ & 0 & - & $q_{\ast1}$  \\
3 & (1,1) & 0 & $q_{0\ast}$ &  $q_{\ast0}$ & - \\
\end{tabular}
\end{center}
\label{default}
\end{table}%
Under a the most general model:
\begin{table}[htdp]
\begin{center}
\begin{tabular}{lr|c|c|c|c|}
& & 0 & 1 & 2 & 3 \\
& & (0,0) & (0,1) & (1,0) & (1,1) \\
\hline
0 & (0,0) & - & $q_{01}$ & $q_{02}$ & 0 \\
1 & (0,1) & $q_{10}$ & -  & 0 & $q_{13}$ \\
2 & (1,0) & $q_{20}$ & 0 & - & $q_{23}$  \\
3 & (1,1) & 0 & $q_{31}$ &  $q_{32}$ & - \\
\end{tabular}
\end{center}
\label{default}
\end{table}%

\myNewSlide
\begin{figure}
\begin{center}
\label{altTrees}
\begin{picture}(0,0)(50,0)
	\thicklines
	\put(-20,-30){\circle{30}} 
	\put(150,-30){\color{red}\circle{30}} 
	\put(-35,-180){\line(0,1){30}} 
	\put(-35,-180){\line(1,0){30}} 
	\put(-5,-150){\line(0,-1){30}} 
	\put(-5,-150){\line(-1,0){30}} 
	\put(10,-28){\vector(1,0){100}} 
	\put(110,-35){\vector(-1,0){100}} 

	\put(-25,-33){\large 0,0} 
	\put(144,-33){\large 0,{\color{red}1}} 
	\put(-28,-168){\large 1,0} 
	\put(142,-168){\large 1,{\color{red}1}} 

	\put(10,-158){\vector(1,0){100}} 
	\put(110,-165){\vector(-1,0){100}} 
	\put(135,-180){\color{red}\line(0,1){30}} 
	\put(135,-180){\color{red}\line(1,0){30}} 
	\put(165,-150){\color{red}\line(0,-1){30}} 
	\put(165,-150){\color{red}\line(-1,0){30}} 
	\put(-25,-140){\vector(0,1){90}} 
	\put(-15,-50){\vector(0,-1){90}} 
	\put(144,-140){\vector(0,1){90}} 
	\put(154,-50){\vector(0,-1){90}} 

	\put(60,-20){${s}_{0{\color{red}1}}$} 
	\put(60,-42){${s}_{{\color{red}1}0}$} 
	\put(60,-150){${s}_{0{\color{red}1}}$} 
	\put(60,-172){${s}_{{\color{red}1}0}$} 

	\put(52,-21){\Large$\circ$} 
	\put(52,-43){\Large$\circ$} 
	\put(54,-150){\line(0,1){5}} 
	\put(54,-150){\line(1,0){5}} 
	\put(59,-145){\line(0,-1){5}} 
	\put(59,-145){\line(-1,0){5}} 
	\put(54,-173){\line(0,1){5}} 
	\put(54,-173){\line(1,0){5}} 
	\put(59,-168){\line(0,-1){5}} 
	\put(59,-168){\line(-1,0){5}} 

	\put(-40,-90){${f}_{{\color{black}1}0}$} 
	\put(-10,-90){${f}_{0{\color{black}1}}$} 
	\put(125,-90){${\color{red}{f}_{10}}$} 
	\put(160,-90){${\color{red}{f}_{01}}$} 
\end{picture}
\end{center}
\vskip 4.1cm
\end{figure}

\myNewSlide
We can calculate maximum likelihood scores under any of these models and do model selection.

Or we could put priors on the parameter values and using Bayesian model selection ({\em e.g.} reversible
jump methods in BayesTraits).
\myNewSlide
\bibliography{phylo}

\end{document}