\documentclass[landscape]{foils} 
\input{../common-preamble-start}
\input{../preamble.tex}
\usepackage{url}
\usepackage{hyperref}
\hypersetup{backref,  linkcolor=blue, citecolor=red, colorlinks=true, hyperindex=true}

\usepackage{pdfpages}
\usepackage{bm}
\usepackage{ifsym}

\newcommand{\disruption}{\theta}
\begin{document}
\pagecolor{white}
\unitlength=1mm
\begin{center}
{\Large Some of these slides have been borrowed from Dr.\ Paul Lewis, Dr.\ Joe Felsenstein. Thanks!}
\vskip 15mm
\large Paul has many great tools for teaching phylogenetics at his web site: \\
\url{http://hydrodictyon.eeb.uconn.edu/people/plewis}
\end{center}

\myNewSlide
\includepdf[pages={2-3}]{../nonfreeimages/pol/polratehet.pdf} 

\myNewSlide
\section*{Question: Why is rate heterogeneity ubiquituous?}
Answer: Differences in mutational rates and (mainly) {\bf selective constraint}
\begin{compactitem}
	\item Many sites are under purifying (stabilizing) selection:
	\begin{compactitem}
		\item Any mutation results in a different amino acid, AND
		\item A amino acid replacement at the site results in dramatically worse functioning of the protein.
		\item These sites will show {\em low} rates of evolution on a tree.
	\end{compactitem}
	\item Other sites are less constrained.
	\begin{compactitem}
		\item A mutation results in the same amino acid, OR
		\item Many amino acids will work equally well at that position in the protein.
		\item These sites will show {\em high} rates of evolution on a tree.
	\end{compactitem}
\end{compactitem}

\myNewSlide
\section*{Rate heterogeneity in protein-coding genes: terms}
\large
\begin{compactitem}
	\item {\bf Synonymous} mutations result in the same amino acid.
	\item {\bf Non-synonymous} mutations result in the different amino acid.
	\item {\bf Conservative} changes are non-synonymous changes that result in a chemically similar amino acid.
	\item {\bf Neutral} mutations result in a new genotype that has the same fitness as the genotypes currently fixed in the population.
\end{compactitem}

\myNewSlide
\section*{Rate heterogeneity in protein-coding genes: generalities}
\large
\begin{compactitem}
	\item Synonymous changes are often neutral (or close to neutral),
	\item Third base positions and untranslated regions (introns and other non-coding regions) tend to have high rates because changes to these sites lead to synonymous changes.
	\item Transitions tend to lead to more synonymous or conservative changes.
	\item Amino acid residues that are embedded, involved in salt bonding, or part of the active site tend to be more constrained.
	\item Loops of amino acid residues on the outside of proteins often tolerate a wide range of substitutions (or even indels).
\end{compactitem}
\myNewSlide
\normalsize
\begin{table}[htdp]
\begin{center}
\begin{tabular}{|c|cc|cc|cc|cc|}
\hline
 & \multicolumn{8}{c|}{2nd Base}\\
 & U && C && A && G& \\
\hline
 \multirow{4}{*}{U} & UUU & \multirow{2}{*}{\bf F} & UCU & \multirow{4}{*}{\bf S} & UAU & \multirow{2}{*}{\bf Y} & UGU & \multirow{2}{*}{\bf C}\\
	&  UUC & & UCC & & UAC & & UGC & \\
	&  UUA & \multirow{2}{*}{\bf L} & UCA & & UAA & \multirow{2}{*}{\bf *} & UGA & {\bf *} \\
	&  UUG & & UCG &  & UAG & & UGG & {\bf W} \\
 \hline 
 \multirow{4}{*}{C} & CUU & \multirow{4}{*}{\bf L} & CCU & \multirow{4}{*}{\bf P} & CAU & \multirow{2}{*}{\bf H} & CGU & \multirow{4}{*}{\bf R}\\
	 & CUC & & CCC & & CAC & & CGC & \\
	 & CUA & & CCA & & CAA & \multirow{2}{*}{\bf Q} & CGA &  \\
	 & CUG & & CCG &  & CAG & & CGG &  \\
 \hline 
 \multirow{4}{*}{A} & AUU & \multirow{3}{*}{\bf I} & ACU & \multirow{4}{*}{\bf T} & AAU & \multirow{2}{*}{\bf N} & AGU & \multirow{2}{*}{\bf S}\\
 & AUC & & ACC & & AAC & & AGC & \\
 & AUA&  & ACA & & AAA & \multirow{2}{*}{\bf K} & AGA & \multirow{2}{*}{\bf R} \\
 & AUG & {\bf M}& ACG &  & AAG & & AGG & \\
 \hline 
 \multirow{4}{*}{G} & GUU & \multirow{2}{*}{\bf V} & GCU & \multirow{4}{*}{\bf A} & GAU & \multirow{2}{*}{\bf D} & GGU & \multirow{4}{*}{\bf G}\\
 & GUC & & GCC & & GAC & & GGC & \\
 & GUA & \multirow{2}{*}{\bf L} & GCA & & GAA & \multirow{2}{*}{\bf E} & GGA & \\
 & GUG & & GCG &  & GAG & & GGG & \\
 \hline 
\end{tabular}
\end{center}
\label{default}
\end{table}%


\myNewSlide
\includepdf[pages={6-8}]{../nonfreeimages/pol/polratehet.pdf} 

\myNewSlide
\section*{Transition probabilities under the JC69 model}
\Large
with no rate heterogeneity:
\begin{eqnarray*} 
	\Pr(i\rightarrow i | \nu) & = & \frac{1}{4}+\frac{3}{4}\;e^{\frac{-4\nu}{3}} \\
	\Pr(i\rightarrow j |\nu) & = & \frac{1}{4}-\frac{1}{4}\;e^{\frac{-4\nu}{3}} \\
\end{eqnarray*}

\myNewSlide
\section*{Transition probabilities under the JC69 model}
\Large
First base positions under a {\em site-specific rates} model:
\begin{eqnarray*} 
	\Pr(i\rightarrow i | \nu) & = & \frac{1}{4}+\frac{3}{4}\;e^{\frac{-4 r_1\nu}{3}} \\
	\Pr(i\rightarrow j |\nu) & = & \frac{1}{4}-\frac{1}{4}\;e^{\frac{-4r_1\nu}{3}} \\
\end{eqnarray*}

\myNewSlide
\includepdf[pages={11-12}]{../nonfreeimages/pol/polratehet.pdf} 


\myNewSlide
\Large
\begin{compactitem}
	\item {\bf Constant site} -- a site in which all of the taxa display the same character state.
	\item {\bf Invariable site} -- a site in which only one character state is allowed. A site that cannot change state.
\end{compactitem}
All invariable sites are constant, but not all constant sites have to be invariable.


\myNewSlide
	\begin{eqnarray*} 
	\Pr(i\rightarrow i | \mbox{ invariable}) & = & \frac{1}{4}+\frac{3}{4}\;e^{\frac{-4 0\nu}{3}} \\
	 & = & \frac{1}{4}+\frac{3}{4}\;e^0 \\
	 & = & 1 \\
	\Pr(i\rightarrow j | \mbox{ invariable}) & = & \frac{1}{4}-\frac{1}{4}\;e^{\frac{-4 0\nu}{3}} \\
	& = & 0
\end{eqnarray*}

\myNewSlide
\section*{A site's likelihood under the JC+ I model}
\normalsize
$x_i$ is the data pattern for site $i$. General form:
	\begin{eqnarray*} 
		\Pr(x_i|\mbox{JC+I}) & = & p_{\mbox{inv}}\Pr(x_i|\mbox{ inv}) + (1-p_{\mbox{inv}})\Pr\left(x_i|\mbox{JC}, \frac{\bm \nu}{1-p_{\mbox{inv}}}\right)
	 \end{eqnarray*} 	
If $x_i$ is a  variable site:
	\begin{eqnarray*} 
		\Pr(x_i|\mbox{JC+I}) & = & (1-p_{\mbox{inv}})\Pr\left(x_i|\mbox{JC}, \frac{\bm \nu}{1-p_{\mbox{inv}}}\right)
	 \end{eqnarray*} 	
If $x_i$ is a constant site:
	\begin{eqnarray*} 
		\Pr(x_i|\mbox{JC+I}) & = & p_{\mbox{inv}}\Pr(x_i|\mbox{ inv}) + (1-p_{\mbox{inv}})\Pr\left(x_i|\mbox{JC}, \frac{\bm \nu}{1-p_{\mbox{inv}}}\right)
	 \end{eqnarray*} 	

\myNewSlide
\Large
Why $\frac{\bm \nu}{1-p_{\mbox{inv}}}$ ?

\large
We want the mean rate of change to be 1.0 over all sites (so we can interpret the branch lengths in terms of the expected \# of changes per site).

If $r$ is the rate of change for the variable sites then:
\begin{eqnarray*} 
	1 & = & 0 p_{\mbox{inv}} + r \left(1-p_{\mbox{inv}}\right)\\
	 & = & r \left(1-p_{\mbox{inv}}\right) \\
	 r & = & \frac{1}{1-p_{\mbox{inv}}}
\end{eqnarray*} 	

\myNewSlide
\section*{Variable (but unknown) rates}
\begin{compactitem}
	\item We expect more ``shades of grey'' rather than the on-or-off view of the pInvar model.
	\item {\em a priori} we do not know which sites are fast and which are slow
	\item We may be able to characterize the {\em distribution} of rates across sites -- high variance or low variance.
\end{compactitem}

\myNewSlide
\includepdf[pages={20}]{../nonfreeimages/pol/polratehet.pdf} 

\myNewSlide
\section*{Gamma distribution}
\begin{eqnarray*} 
	f(r) & = & \frac{r^{\alpha-1}\beta^{\alpha}e^{-\beta r}}{\Gamma(\alpha)} \\
	\mbox{mean} & = & \alpha/\beta \\
\mbox{mean (in phylogenetics)}	& = & 1 \\
	\mbox{(in phylogenetics) }\beta & = & \alpha \\
	\mbox{variance} & = & \alpha/{\beta^2} \\
\mbox{variance (in phylogenetics)}	& = & 1/{\alpha} \\
\end{eqnarray*} 


\myNewSlide
\section*{Using Gamma-distributed rates across sites}
\begin{compactitem}
	\item We usually use a discretized version of the gamma with 4-8 categories (the computation time increases linearly with the number of categories).
	\[\Pr(x_i|JC+G) = \sum_j^{\mbox{ncat}} \Pr(x_i|JC, r_j{\bm \nu})\Pr(r_j)\]
	where:
		\[\sum_j^{\mbox{ncat}} r_j\Pr(r_j) = 1\]

\myNewSlide
\section*{Discrete gamma (continued)}
We ``break up'' the continuous gamma into intervals each of which has an equal probability, and use the mean rate within each interval as the representative rate for that rate category:
\[\Pr(r_j) = \frac{1}{\mbox{ncat}}\]
So:
	\[\Pr(x_i|JC+G) = \frac{1}{\mbox{ncat}} \sum_j^{\mbox{ncat}} \Pr(x_i|JC, r_j{\bm \nu})\]

\end{compactitem}

\myNewSlide
\includepdf[pages={21}]{../nonfreeimages/pol/polratehet.pdf} 

\myNewSlide
\includepdf[pages={23-24}]{../nonfreeimages/pol/polratehet.pdf} 

\myNewSlide
\includepdf[pages={2-4}]{../nonfreeimages/pol/pollrt.pdf} 

\myNewSlide
\includepdf[pages={6-9}]{../nonfreeimages/pol/pollrt.pdf} 

\myNewSlide
\includepdf[pages={11}]{../nonfreeimages/pol/pollrt.pdf} 


\myNewSlide


\bibliography{phylo}

\end{document}     

\myNewSlide
\section*{How can we calculate the likelihood score}
\Large 
Under the JC (or K2P) model:
\begin{eqnarray*}
	\ln L & =  &  12 \ln\pi_A + 7 \ln\pi_C + 7 \ln\pi_G + 6 \ln\pi_T \\
		 & =  &  12 \ln0.25 + 7 \ln0.25 + 7 \ln0.25 + 6 \ln0.25 \\
		 & = & -44.361
\end{eqnarray*}

\myNewSlide
\section*{How can we calculate the likelihood score}
\Large 
Under the F81 (or HKY or GTR) model:
\begin{eqnarray*}
	\ln L & =  &  12 \ln\pi_A + 7 \ln\pi_C + 7 \ln\pi_G + 6 \ln\pi_T \\
\end{eqnarray*}
But what are the values for the parameters: $\pi_A, \pi_C, \pi_G, \pi_T$ ?

In many cases we refer to these parameters as ``nuisance parameters.''
They must be specified in order to calculate the likelihood, but we are not interested in them by themselves.


\myNewSlide
\section*{ML parameter estimates}
\large
We can find the maximum likelihood estimates of the parameters to give us the ML score: the maximum likelihood obtainable under this model:
\begin{eqnarray*}
	\ln L & =  &  12 \ln\pi_A + 7 \ln\pi_C + 7 \ln\pi_G + 6 \ln\pi_T \\
	 & =  &  12 \ln\widehat{\pi_A} + 7 \ln\widehat{\pi_C} + 7 \ln\widehat{\pi_G} + 6 \ln\widehat{\pi_T} \\
	 & =  &  12 \ln0.375 + 7 \ln0.21875 + 7 \ln0.21875+ 6 \ln0.1875 \\
	 & =  & -43.091 
\end{eqnarray*}
But how did I get the numbers to fill for the parameters?
How do we know that $\widehat{\pi_A} = 0.375$ and $\widehat{\pi_C} = 0.21875$

\myNewSlide
\section*{ML parameter estimates}
We might guess that:
\begin{eqnarray*}
	\widehat{\pi_A} & = &  \frac{12}{32} = 0.375\\
	\widehat{\pi_C} & = &  \frac{7}{32} = 0.21875\\
	\widehat{\pi_G} & = &  \frac{7}{32} = 0.21875\\
	\widehat{\pi_T} & = &  \frac{6}{32} = 0.1875
\end{eqnarray*}
but how do we prove it?

\myNewSlide
\section*{ML parameter estimates}
For simple problems we solve for the point in parameter space for which derivatives with respect to all parameters are 0 (we also have to consider boundary points).

We would have to do constrained optimization because \[\pi_A + \pi_C + \pi_G + \pi_T = 1\]
and that is a pain.

\myNewSlide
\section*{ML parameter estimates}
\normalsize
We can reparameterize:
\begin{eqnarray*}
	r & = & {\pi_A+\pi_G} \\
   a  & = & \frac{\pi_A}{\pi_A+\pi_G} \\
   c & = & \frac{\pi_C}{\pi_C+\pi_T}
\end{eqnarray*}
and always recover the original parameters:
\begin{eqnarray*}
	\pi_A & = & ra \\
	\pi_G & = & r(1-a) \\
	\pi_C & = & (1-r)c \\
	\pi_T & = & (1-r)(1-c) \\
\end{eqnarray*}

\myNewSlide
\section*{ML parameter estimates}
\begin{eqnarray*}
	\ln L & = & 12 \ln\pi_A + 7 \ln\pi_C + 7 \ln\pi_G + 6 \ln\pi_T \\
	& = & 12 \ln\left[ra\right] + 7 \ln\left[(1-r)c\right] + 7 \ln\left[r(1-a)\right] + 6 \ln\left[(1-r)(1-c)\right] 
\end{eqnarray*}
Recall that:
\begin{eqnarray*}
	\frac{\partial \ln f(x)}{\partial x} =\frac{\frac{\partial f(x)}{\partial x}}{f(x)}
\end{eqnarray*}


\myNewSlide
\begin{eqnarray*}
	\ln L & = & 12 \ln\left[ra\right] + 7 \ln\left[(1-r)c\right] + 7 \ln\left[r(1-a)\right] + 6 \ln\left[(1-r)(1-c)\right]  \\
	\frac{\partial \ln L}{\partial a} & = &\frac{12 r}{ra} + \frac{7(-r)}{r(1-a)} \\
	& = & \frac{12}{a} -\frac{7}{(1-a)}\\
	 0 & = & \frac{12}{\hat{a}} -\frac{7}{(1-\hat{a})}  \\
	 \hat{a} & = & \frac{12}{19}
\end{eqnarray*}

\myNewSlide
\begin{eqnarray*}
	\ln L & = & 12 \ln\left[ra\right] + 7 \ln\left[(1-r)c\right] + 7 \ln\left[r(1-a)\right] + 6 \ln\left[(1-r)(1-c)\right]  \\
	\frac{\partial \ln L}{\partial c} & = &\frac{7 (1-r)}{(1-r)c} + \frac{6-(1-r)}{(1-r)(1-a)} \\
	& = & \frac{7}{c} -\frac{6}{(1-c)}\\
	 0 & = & \frac{7}{\hat{c}} -\frac{6}{(1-\hat{c})}  \\
	 \hat{c} & = & \frac{7}{13}
\end{eqnarray*}

\myNewSlide
\begin{eqnarray*}
	\ln L & = & 12 \ln\left[ra\right] + 7 \ln\left[(1-r)c\right] + 7 \ln\left[r(1-a)\right] + 6 \ln\left[(1-r)(1-c)\right]  \\
	\frac{\partial \ln L}{\partial r} & = &\frac{12 a}{ra} + \frac{7(-c)}{(1-r)c} + \frac{7(1-a)}{r(1-a)} + \frac{6(-(1-c))}{(1-r)(1-c)} \\
	& = & \frac{12}{r} -\frac{7}{(1-r)} + \frac{7}{r} - \frac{6}{1-r}\\
	& = & \frac{19}{r} -\frac{13}{(1-r)}\\
	 0 & = & \frac{19}{\hat{r}} -\frac{13}{(1-\hat{r})}  \\
	 \hat{r} & = & \frac{19}{32}
\end{eqnarray*}


\myNewSlide
ML inference displays ``scale invariance'' so we can just transform the ML estimates into our original parameters:
\begin{eqnarray*}
	\widehat{\pi_A} = \hat{r}\hat{a}  & = & \left(\frac{19}{32}\right)\left(\frac{12}{19}\right) = \frac{12}{32} \\
	\widehat{\pi_G}  = \hat{r}(1-\hat{a})& = & \left(\frac{19}{32}\right)\left(\frac{7}{19}\right) = \frac{7}{32} \\
	\widehat{\pi_C}  =  (1-\hat{r})\hat{c} & = & \left(\frac{13}{32}\right)\left(\frac{6}{13}\right) = \frac{7}{32} \\
	\widehat{\pi_T}  =  (1-\hat{r})(1-\hat{c})& = & \left(\frac{13}{32}\right)\left(\frac{6}{13}\right) = \frac{6}{32} \\
\end{eqnarray*}

\myNewSlide
\section*{Likelihoods on the simplest possible tree}
\begin{center}
{\Huge
\tt GA$\rightarrow$GG}
\end{center}
\large
\begin{eqnarray*}
	L & = & L_1 L_2 \\
	 & = & \Pr(G)\Pr(G\rightarrow G) \Pr(A)\Pr(A\rightarrow G)\\
	 & = & \Pr(G)\Pr(G\rightarrow G|\nu) \Pr(A)\Pr(A\rightarrow G|\nu) \\
	 & = & \left(\frac{1}{4}\right)\left(\frac{1}{4} + \frac{3}{4}\;e^{\frac{-4\nu}{3}}\right)\left(\frac{1}{4}\right)\left(\frac{1}{4}-\frac{1}{4}\;e^{\frac{-4\nu}{3}}\right)
\end{eqnarray*}

\myNewSlide
\normalsize
\begin{eqnarray*}
	d & = & \frac{1}{4}-\frac{1}{4}\;e^{\frac{-4\nu}{3}} \\
	\left(\frac{1}{4} + \frac{3}{4}\;e^{\frac{-4\nu}{3}}\right) & = & 1 -3d \\
	L & = & \left(\frac{1}{4}\right)\left(\frac{1}{4} + \frac{3}{4}\;e^{\frac{-4\nu}{3}}\right)\left(\frac{1}{4}\right)\left(\frac{1}{4}-\frac{1}{4}\;e^{\frac{-4\nu}{3}}\right) \\
	& = & \frac{(1-3d)d}{16} \\
	\frac{\partial \ln L}{\partial d} & = &\frac{1-6d}{16}\\
	0 & = &\frac{1-6\hat{d}}{16} \\
	\hat{d}& = &\frac{1}{6}\\
	\hat{\nu } & = & 0.82396 \\
	L & = & 0.005208
\end{eqnarray*}

\myNewSlide
\large
You may recall that the JC distance correction from lecture 8 looked like this:
\[\nu = \frac{-3}{4}\;\ln\left(1-\frac{4p}{3}\right) \]
If you put in $p= 0.5$, because half the sites differ in our example then you the same branch length:
\[\nu  = 0.82396 \]
Our JC distance correction formula is actually an ML estimator of the branch length between a pair of taxa.

\myNewSlide
The first 30 nucleotides of the $\psi\eta$-globin gene
{\tt
\begin{table}[htdp]
\begin{center}
\begin{tabular}{lc}
gorilla & G{\color{red}A}A{\color{red}G}TCCTTGAGAAATAAACTGCACACTGG \\
orangutan & G{\color{red}G}A{\color{red}C}TCCTTGAGAAATAAACTGCACACTGG\\
\end{tabular}
\end{center}
\end{table}%
}
\vskip -2cm
\[ L = \left[\left(\frac{1}{4}\right)\left(\frac{1}{4} + \frac{3}{4}\;e^{\frac{-4\nu}{3}}\right)\right]^28 \left[\left(\frac{1}{4}\right)\left(\frac{1}{4}-\frac{1}{4}\;e^{\frac{-4\nu}{3}}\right)\right]^2\]
\begin{picture}(0,0)(0,40)
	\put(0,0){\makebox(0,0)[l]{\includegraphics[scale=0.5]{../nonfreeimages/pol/GorOrangProfile.pdf}}}
	\put(70,-20){$\hat{\nu}=0.06982$}
	\put(70,-30){$\ln L= -51.13396$}
\end{picture}

\myNewSlide
\includepdf[pages={9-12}]{../nonfreeimages/pol/pol.pdf} 

\myNewSlide
\begin{picture}(0,0)(-30,200)
	\put(0,0){\makebox(0,0)[l]{\includegraphics{../images/pruning_tree.pdf}}}
	\put(0,170){\begin{tabular}{|c|c|}
\hline
Taxon & Character \\
\hline
1 & A \\
2 & C \\
3 & C \\
4 & C \\
5 & G \\
\hline
\end{tabular}
}
\end{picture}

\myNewSlide
\[L = \sum_x \sum_y \sum_z \sum_w\Pr(x,y,z,w,A,C,C,C,G|{\bm\nu}) \]
\begin{picture}(0,0)(-30,200)
	\put(0,100){\makebox(0,0)[l]{\includegraphics{../images/pruning_treeC.pdf}}}
\end{picture}

\myNewSlide
\normalsize
\begin{eqnarray*}
L & = & \sum_x \sum_y \sum_z \sum_w\Pr(x)\Pr(y|x,\nu_6)\Pr(A|y,\nu_1)\Pr(C|y,\nu_2) \cdots \\
	&& \Pr(z|x,\nu_8)\Pr(C|z,\nu_3)\Pr(w|z,\nu_7) \Pr(C|w,\nu_4)\Pr(G|w,\nu_5) 
\end{eqnarray*}
\begin{picture}(0,0)(-30,200)
	\put(0,100){\makebox(0,0)[l]{\includegraphics{../images/pruning_treeC.pdf}}}
\end{picture}

\myNewSlide
\normalsize
\begin{eqnarray*}
L & = & \sum_x \sum_y \sum_z \Pr(x)\Pr(y|x,\nu_6)\Pr(A|y,\nu_1)\Pr(C|y,\nu_2) \cdots \\
	&& \Pr(z|x,\nu_8)\Pr(C|z,\nu_3){\color{red}\Big(\sum_w\Pr(w|z,\nu_7) \Pr(C|w,\nu_4)\Pr(G|w,\nu_5) \Big)}
\end{eqnarray*}
\begin{picture}(0,0)(-30,200)
	\put(0,100){\makebox(0,0)[l]{\includegraphics{../images/pruning_treeC.pdf}}}
\end{picture}

\myNewSlide
\normalsize
\begin{eqnarray*}
L & = & \sum_x \sum_y  \Pr(x)\Pr(y|x,\nu_6)\Pr(A|y,\nu_1)\Pr(C|y,\nu_2) \cdots \\
	&& {\color{cyan}\Big(\sum_z\Pr(z|x,\nu_8)\Pr(C|z,\nu_3)}{\color{red}\Big(\sum_w\Pr(w|z,\nu_7) \Pr(C|w,\nu_4)\Pr(G|w,\nu_5) \Big)}{\color{cyan}\Big)}
\end{eqnarray*}
\begin{picture}(0,0)(-30,200)
	\put(0,100){\makebox(0,0)[l]{\includegraphics{../images/pruning_treeC.pdf}}}
\end{picture}

\myNewSlide
\normalsize
\begin{eqnarray*}
L & = & \sum_x   \Pr(x){\color{darkgreen}\Big(\sum_y\Pr(y|x,\nu_6)\Pr(A|y,\nu_1)\Pr(C|y,\nu_2)\Big)} \cdots \\
	&& {\color{cyan}\Big(\sum_z\Pr(z|x,\nu_8)\Pr(C|z,\nu_3)}{\color{red}\Big(\sum_w\Pr(w|z,\nu_7) \Pr(C|w,\nu_4)\Pr(G|w,\nu_5) \Big)}{\color{cyan}\Big)}
\end{eqnarray*}
\begin{picture}(0,0)(-30,200)
	\put(0,100){\makebox(0,0)[l]{\includegraphics{../images/pruning_treeC.pdf}}}
\end{picture}

\myNewSlide
\includepdf[pages={13-14}]{../nonfreeimages/pol/pol.pdf} 

