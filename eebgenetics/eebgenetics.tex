\documentclass[landscape]{foils} 
\input{../common-preamble-start}
\input{../preamble.tex}
\usepackage{url}
\usepackage{hyperref}
\hypersetup{backref,  linkcolor=blue, citecolor=red, colorlinks=true, hyperindex=true}

\usepackage{pdfpages}
\usepackage{bm}
\usepackage{ifsym}
\newcommand{\hilite}[1]{{\color{red} \bf #1}}
\newcommand{\disruption}{\theta}
\begin{document}
\pagecolor{white}
\unitlength=1mm
\begin{center}
{\Large many slides have been borrowed from Joe Felsenstein -- Thanks, Joe!}
\end{center}


\myNewSlide
\includepdf[pages={14}]{../nonfreeimages/joe/joe-coal.pdf}
\includepdf[pages={17}]{../nonfreeimages/joe/joe-coal.pdf}
\includepdf[pages={24}]{../nonfreeimages/joe/joe-coal.pdf}
\includepdf[pages={23}]{../nonfreeimages/joe/joe-coal.pdf}
\includepdf[pages={25}]{../nonfreeimages/joe/joe-coal.pdf}
\includepdf[pages={30}]{../nonfreeimages/joe/joe-coal.pdf}
\includepdf[pages={31}]{../nonfreeimages/joe/joe-coal.pdf}
\includepdf[pages={36}]{../nonfreeimages/joe/joe-coal.pdf}

\myNewSlide
\section*{Inconsistency of estimation from concatenated gene sequences}
\citet{DegnanR2006} show that the most likely topology for a gene tree
is not necessarily the tree that agrees with the 
phylogenetic tree.

For some phylogenetic shapes (e.g. imbalanced trees
with short internal nodes) there exists (at least) one other tree shape
that has a higher probability of agreeing with a gene tree.

Argues for explicitly considering the coalescent process in phylogenetic inference.


\myNewSlide
\section*{Likelihood-based inference using the coalescent}
$S$ is set of sequences.\par
$G$ is a genealogy.\par
$\mu$ is the mutation rate.\par
$N_e$ is the effective population size.
$$ \mbox{Likelihood}(\mu,N_e) = \Pr(S|\mu,N_e) $$
$$ \Pr(S|\mu,N_e) = \int \Pr(S|G',\mu)\Pr(G'|N_e)dG'$$
It is important to integrate over all $G'$, because we can't estimate $G$ with much precision.

\includepdf[pages={45}]{../nonfreeimages/joe/joe-coal.pdf}


\myNewSlide
\section*{Recombination means that the genealogies for linked sites are neither identical nor independent}
\begin{picture}(0,0)(0,0)
	\put(0,-180){\makebox(0,0)[l]{\includegraphics[scale=1.5]{../images/KuhnerYF2000.pdf}}}
	\put(-20,-120){{\tiny Figure from Kuhner, Yamato, Felsentein 2000}}
\end{picture}

\myNewSlide
Accounting for recombination becomes important for estimating any parameter from coalescent trees,
whenever your sequences so long enough that $$\Pr(\mbox{recombination before coalescence})$$ is not tiny.

\begin{itemize}
	\item $R_{min}(S)$ is the minimum number of recombination events required to explain the sequences, if we disallow ``double-hits''
An efficient algorithm for calculating $R_{min}(S)$ for lots of leaves and long sequences does not exist. (``parsimony wrt \# recombination events'').
	\item Another approach would be to jointly estimate all parameters of interest and $R(S)$, or calculate a posterior probability of $R(S)$. (typically Bayesian inference).
\end{itemize}

\myNewSlide
\section*{Calculating $R_{min}(S)$ under the infinite-sites model}
\begin{itemize}
	\item Seems like it should not be too hard (and it isn't for few leaves) - connected to integer programming approaches
	\item Tightly connected to the ``perfect phylogeny'' and ``compatibility'' literature in phylogenetics
	\item First studied by \citet{HudsonK1985}
	\item Several lower bounds on $R_{min}(S)$ published by \citet{MyersG2003,SongH2004}
	\item Exact calculation \citet{Hein1990,Hein1993,SongH2003}
\end{itemize}

\myNewSlide
\section*{Exact calculation of \citet{SongH2004}}
\begin{enumerate}
	\item Recombination events $\rightarrow$ SPR operations on rooted trees;
	\item Calculate the ``SPR-distance'' between trees (with some restriction induced by the ages of nodes);
	\item We can tell which trees are compatible with each parsimony-informative site
	\item Dynamic programming lets us walk from the beginning to the end of the sequences storing the path to that gives us the minimal number of recombinations and that is compatible with every site.
	\item Impractical for large trees (over 9 unique sequences) -- we have to be able to enumerate labelled histories
\end{enumerate}

\myNewSlide
\includepdf[pages={46}]{../nonfreeimages/joe/joe-coal.pdf}

\myNewSlide
\section*{An ARG (ancestral recombination graph)}
\begin{picture}(0,0)(0,0)
	\put(0,-180){\makebox(0,0)[l]{\includegraphics[scale=1.5]{../images/SongH2003Fig1.pdf}}}
	\put(-0,-50){\begin{tabular}{ccccc}
 & 1 & 2 & 3 & 4 \\
 \hline
$s_1$ & 0 & 0 & 0 & 0 \\
$s_2$ & 0 & 0 & 1 & 1 \\
$s_3$ & 0 & 1 & 0 & 1 \\
$s_4$ & 1 & 1 & 0 & 0 \\
$s_5$ & 1 & 1 & 1 & 1 \\
\end{tabular}
}
	\put(-20,-120){{\tiny Figure from Song and Hein 2003}}
\end{picture}


\myNewSlide
\section*{The trees embedded in the ARG}
\begin{picture}(0,0)(0,0)
	\put(-50,-250){\makebox(0,0)[l]{\includegraphics[scale=1.5]{../images/SongH2003Fig1.pdf}}}
	\put(-50,-0){\makebox(0,0)[l]{\includegraphics[scale=1.5]{../images/SongH2003Fig1.pdf}}}
\end{picture}

\myNewSlide
\section*{$R(S)$ score for the ARG}
\begin{picture}(0,0)(0,0)
	\put(-120,-250){\makebox(0,0)[l]{\includegraphics[scale=1.5]{../images/SongH2003Fig1.pdf}}}
	\put(-50,-0){\makebox(0,0)[l]{\includegraphics[scale=1.5]{../images/SongH2003Fig1.pdf}}}
	\put(100,-120){\makebox(0,0)[l]{\includegraphics[scale=1.5]{../images/SongH2003Fig3.pdf}}}
\end{picture}


\myNewSlide
\bibliography{phylo}

\end{document}
