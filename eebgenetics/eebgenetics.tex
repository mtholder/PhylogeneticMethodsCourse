\documentclass[landscape]{foils} 
\input{../common-preamble-start}
\input{../preamble.tex}
\usepackage{url}
\usepackage{hyperref}
\hypersetup{backref,  linkcolor=blue, citecolor=red, colorlinks=true, hyperindex=true}

\usepackage{pdfpages}
\usepackage{bm}
\usepackage{ifsym}
\usepackage{amssymb}
\newcommand{\hilite}[1]{{\color{red} \bf #1}}
\newcommand{\disruption}{\theta}
\begin{document}
\pagecolor{white}
\unitlength=1mm
\begin{center}
{\Large many slides have been borrowed from Joe Felsenstein -- Thanks, Joe!}
\end{center}


\myNewSlide
\includepdf[pages={14}]{../nonfreeimages/joe/joe-coal.pdf}
\includepdf[pages={17}]{../nonfreeimages/joe/joe-coal.pdf}
\includepdf[pages={24}]{../nonfreeimages/joe/joe-coal.pdf}
\includepdf[pages={23}]{../nonfreeimages/joe/joe-coal.pdf}
\includepdf[pages={25}]{../nonfreeimages/joe/joe-coal.pdf}
%\includepdf[pages={30}]{../nonfreeimages/joe/joe-coal.pdf}
\includepdf[pages={31}]{../nonfreeimages/joe/joe-coal.pdf}
\includepdf[pages={36}]{../nonfreeimages/joe/joe-coal.pdf}

\myNewSlide
\section*{Inconsistency of estimation from concatenated gene sequences}
\large
\citet{DegnanR2006} show that the most likely topology for a gene tree
is not necessarily the tree that agrees with the 
phylogenetic tree.

For some phylogenetic shapes (e.g. imbalanced trees
with short internal nodes) there exists (at least) one other tree shape
that has a higher probability of agreeing with a gene tree.

Argues for explicitly considering the coalescent process in phylogenetic inference.


\myNewSlide
\section*{Likelihood-based inference using the coalescent}
$S$ is set of sequences.\par
$G$ is a genealogy.\par
$\mu$ is the mutation rate.\par
$N_e$ is the effective population size.
$$ \mbox{Likelihood}(\mu,N_e) = \Pr(S|\mu,N_e) $$
$$ \Pr(S|\mu,N_e) = \int \Pr(S|G',\mu)\Pr(G'|N_e)dG'$$
It is important to integrate over all $G'$, because we can't estimate $G$ with much precision.

\includepdf[pages={45}]{../nonfreeimages/joe/joe-coal.pdf}


\myNewSlide
\section*{Recombination means that the genealogies for linked sites are neither identical nor independent}
\begin{picture}(0,0)(0,0)
	\put(0,-180){\makebox(0,0)[l]{\includegraphics[scale=1.5]{../images/KuhnerYF2000.pdf}}}
	\put(-20,-120){{\tiny Figure from Kuhner, Yamato, Felsentein 2000}}
\end{picture}

\myNewSlide
Accounting for recombination becomes important for estimating any parameter from coalescent trees,
whenever your sequences so long enough that $$\Pr(\mbox{recombination before coalescence})$$ is not tiny.

\begin{itemize}
	\item $R_{min}(S)$ is the minimum number of recombination events required to explain the sequences, if we disallow ``double-hits''
An efficient algorithm for calculating $R_{min}(S)$ for lots of leaves and long sequences does not exist. (``parsimony wrt \# recombination events'').
	\item Another approach would be to jointly estimate all parameters of interest and $R(S)$, or calculate a posterior probability of $R(S)$. (typically Bayesian inference).
\end{itemize}

\myNewSlide
\section*{Calculating $R_{min}(S)$ under the infinite-sites model}
\begin{itemize}
	\item Seems like it should not be too hard (and it isn't for few leaves) - connected to integer programming approaches
	\item Tightly connected to the ``perfect phylogeny'' and ``compatibility'' literature in phylogenetics
	\item First studied by \citet{HudsonK1985}
	\item Several lower bounds on $R_{min}(S)$ published by \citet{MyersG2003,SongH2004}. Violation of infinite-sites assumption studied in \citet{LiuF2008}
	\item Exact calculation \citet{Hein1990,Hein1993,SongH2003}
\end{itemize}

\myNewSlide
\section*{Exact calculation of \citet{SongH2004}}
\normalsize
\begin{enumerate}
	\item Recombination events $\rightarrow$ SPR operations on rooted trees;
	\item Calculate the ``SPR-distance'' between trees (with some restriction induced by the ages of nodes);
	\item We can tell which trees are compatible with each parsimony-informative site
	\item Dynamic programming lets us walk from the beginning to the end of the sequences storing the path to that gives us the minimal number of recombinations and that is compatible with every site.
	\item Impractical for large trees (over 9 unique sequences) -- we have to be able to enumerate ordered trees (aka ``labelled histories'')
\end{enumerate}

\myNewSlide
\includepdf[pages={46}]{../nonfreeimages/joe/joe-coal.pdf}

\myNewSlide
\section*{An ARG (ancestral recombination graph)}
\begin{picture}(0,0)(0,0)
	\put(0,-180){\makebox(0,0)[l]{\includegraphics[scale=1.5]{../images/SongH2003Fig1.pdf}}}
	\put(-0,-50){\begin{tabular}{ccccc}
 & 1 & 2 & 3 & 4 \\
 \hline
$s_1$ & 0 & 0 & 0 & 0 \\
$s_2$ & 0 & 0 & 1 & 1 \\
$s_3$ & 0 & 1 & 0 & 1 \\
$s_4$ & 1 & 1 & 0 & 0 \\
$s_5$ & 1 & 1 & 1 & 1 \\
\end{tabular}
}
	\put(-20,-120){{\tiny Figure from Song and Hein 2003}}
\end{picture}


\myNewSlide
\section*{The trees embedded in the ARG}
\begin{picture}(0,0)(0,0)
	\put(-50,-250){\makebox(0,0)[l]{\includegraphics[scale=1.5]{../images/SongH2003Fig1.pdf}}}
	\put(-50,-0){\makebox(0,0)[l]{\includegraphics[scale=1.5]{../images/SongH2003Fig1.pdf}}}
\end{picture}

\myNewSlide
\section*{$R(S)$ score for the ARG}
\begin{picture}(0,0)(0,0)
	\put(-120,-250){\makebox(0,0)[l]{\includegraphics[scale=1.5]{../images/SongH2003Fig1.pdf}}}
	\put(-50,-0){\makebox(0,0)[l]{\includegraphics[scale=1.5]{../images/SongH2003Fig1.pdf}}}
	\put(100,-120){\makebox(0,0)[l]{\includegraphics[scale=1.5]{../images/SongH2003Fig3.pdf}}}
\end{picture}


\myNewSlide
\section*{The dynamic programming bit}
\begin{picture}(0,0)(0,0)
	\put(-50,11){\makebox(0,0)[l]{\includegraphics[scale=1.5]{../images/SongH2003Fig1.pdf}}}
	\put(40,-100){\begin{tabular}{ccccccccc}
	& 1 & & 2 & & 3 & & 4 \\
\hline
	$T_i$ & {\color{red}0} & {\color{red}$\rightarrow$} & {\color{red}0} & $\rightarrow$ & $\infty$& $\rightarrow$  & $\infty$ \\
	& & $\searrow\mkern-18mu\nearrow$ & & {\color{red}$\searrow$}$\mkern-18mu\nearrow$ & & $\searrow\mkern-18mu\nearrow$\\
	$T_j$ & $\infty$ & $\rightarrow$ & $\infty$ & & {\color{red}1} & $\rightarrow$ & $\infty$ \\
	& & $\searrow\mkern-18mu\nearrow$ & & $\searrow$$\mkern-18mu\nearrow$ & & {\color{red}$\searrow$}$\mkern-18mu\nearrow$\\
	$T_k$ & $\infty$ & $\rightarrow$ & $\infty$ & $\rightarrow$ & $\infty$ &&  {\color{red}2} \\
	$\vdots$ &  & $\searrow\mkern-18mu\nearrow$ & & $\searrow\mkern-18mu\nearrow$ & & $\searrow\mkern-18mu\nearrow$\\
	$T_{57}$ & $0$ & $\rightarrow$ & $\infty$ & $\rightarrow$ & $\infty$ & $\rightarrow$ &  $\infty$ \\
	$\vdots$ &  & $\searrow\mkern-18mu\nearrow$ & & $\searrow\mkern-18mu\nearrow$ & & $\searrow\mkern-18mu\nearrow$\\
	$T_{85}$ & $\infty$ & $\rightarrow$ & $\infty$ & $\rightarrow$ & $3$ & $\rightarrow$&  $3$ \\
	$\vdots$ &  & $\searrow\mkern-18mu\nearrow$ & & $\searrow\mkern-18mu\nearrow$ & & $\searrow\mkern-18mu\nearrow$\\
	\end{tabular}}
\end{picture}



\myNewSlide
\section*{The need to consider constraints in SPR moves on ordered trees}
\begin{picture}(0,0)(0,0)
	\put(30,-20){\makebox(0,0)[l]{\includegraphics[scale=1.5]{../images/SongH2003Fig6.pdf}}}
\end{picture}
\myNewSlide
\section*{The need to consider constraints in rooted SPR moves (continued)}
\begin{picture}(0,0)(0,0)
	\put(0,-20){\makebox(0,0)[l]{\includegraphics[scale=1.5]{../images/SongH2003Fig7.pdf}}}
\end{picture}

\myNewSlide
\section*{Faster (but still slow) calculation of $R_{min}(S)$}
\large
\citet{LyngsoSH2005} developed a branch and bound algorithm that move backward in time rejecting partial trees that are incompatible with the data.
\begin{compactitem}
	\item Avoid considering recombination in ``subsumed prefix'' and ``subsumed suffix''
\end{compactitem}

\myNewSlide
\section*{Subsumed prefix and suffix shown in red}
\begin{picture}(0,0)(0,0)
	\put(30,-120){\makebox(0,0)[l]{\includegraphics[scale=1.0]{../images/PrefixSuffixARG.pdf}}}
\end{picture}


\myNewSlide
\section*{Faster (but still slow) calculation of $R_{min}(S)$}
\large
\citet{LyngsoSH2005} developed a branch and bound algorithm that move backward in time rejecting partial trees that are incompatible with the data.
\begin{compactitem}
	\item Avoid considering recombination in ``subsumed prefix'' and ``subsumed suffix''
	\item Consider only a restricted set of events that is plausibly on the minimal path,
	\item implemented in {\tt beagle} and a quicker heuristic in {\tt kwarg} (see \href{http://www.stats.ox.ac.uk/~lyngsoe/section26/}{http://www.stats.ox.ac.uk/~lyngsoe/section26} )
\end{compactitem}

\myNewSlide
\section*{Decomposing theory for phylogenetic networks}
\large
\citet{GusfieldBBS2007}:
\begin{compactitem}
	\item prove that we can use the incompatibility graph to generate a fully-decomposed phylogenetic network (in polynomial time - perl scripts at \url{http://csiflabs.cs.ucdavis.edu/~gusfield/software.html}
	\item disprove previous conjectures that this network can be used to find the minimum \# of recombination nodes {\em in general},
	\item fully-decomposed phylogenetic network is helpful if the network is a {\em galled} tree,
	\item conduct simulations to show that cases in which the fully decomposed network does not help us find the min.~\# recombination nodes are rare.
\end{compactitem}

\myNewSlide
\begin{picture}(0,0)(0,0)
	\put(0,-50){\makebox(0,0)[l]{\includegraphics[scale=1.5]{../images/GusfieldBBS2007.pdf}}}
\put(-0,-0){M:}
\put(-0,-50){\begin{tabular}{cccccc}
 & 1 & 2 & 3 & 4 & 5\\
 \hline
$a$ & 0 & 0 & 0 & 1 & 0 \\
$b$ & 1 & 0 & 0 & 1 & 0 \\
$c$ & 0 & 0 & 1 & 0 & 0 \\
$d$ & 1 & 0 & 1 & 0 & 0 \\
$e$ & 0 & 1 & 1 & 0 & 0 \\
$f$ & 0 & 1 & 1 & 0 & 1 \\
$g$ & 0 & 0 & 1 & 0 & 1 \\
\end{tabular}
}
\put(-0,-150){\small from Gusfield, {\em et al} 2007}
\end{picture}


\myNewSlide
\section*{Calculating $R_{min}(S)$}
\large
\begin{compactitem}
	\item Not fully using the likelihood --  branch length information $\therefore$  not misled by singleton sequencing errors (\citet{LyngsoSH2008} implement parsimony-based approximations for the likelihood on ARGs, but these are not practical yet).
	\item Infinite-sites assumption
	\item Many approaches (Gusfield's) need to be extended to deal with missing data and three-state characters.
\end{compactitem}


\myNewSlide
\section*{Estimating ARGs  (usually done in a Bayesian context)}
\begin{compactitem}
	\item Change points between trees (marginalizing over branch lengths). {\tt cBrother} \citep{SuchardWDS2002,SuchardWDS2003,FangDMSD2007} 
	\item Change points between trees  and substitution process. {\tt DualBrother} \citep{MininDFS2005}
	\item ARGs for horizontal transmission in phylogenetic contexts. {\tt SMARTIE} \citep{BloomquistS2010}
	\item ML estimates conditional the number of breakpoints by iteratively examining low numbers using a GA: {\tt GARD} \citep{KosakovskyPondPGWF2006}; {\tt SCUEAL} \citep{KosakovsyPondEtAl2009}
	\item Using approximate SPR-distance to specify a prior on the next site's tree:  \href{http://www.biomcmc.org/software/biomc2}{biomc2} \citep{OliveiraMartinsLK2008,OliveiraMartinsK2010} (still marginalizes over branch lengths).
\end{compactitem}

\myNewSlide
\section*{General impressions/questions}
\begin{compactitem}
	\item Still difficult in on a chromosome scale (even for small \# of leaves).
	\item \citet{McVeanC2005} use ``Sequentially Markov Coalescent'' as an approximation -- prohibit coalescence between chromosomes that don't have any overlap in ancestral material
	\item To what extent can we model sequencing/assembly error?
	\item Does marginalizing over branch lengths lead to significant artifacts? -- if not biomc2 looks promising for sampling from the posterior.
\end{compactitem}
	


\small

\myNewSlide
\bibliography{phylo}

\end{document}
