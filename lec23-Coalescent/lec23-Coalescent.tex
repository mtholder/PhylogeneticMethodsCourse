\documentclass[landscape]{foils} 
\input{../common-preamble-start}
\input{../preamble.tex}
\usepackage{url}
\usepackage{hyperref}
\hypersetup{backref,  linkcolor=blue, citecolor=red, colorlinks=true, hyperindex=true}

\usepackage{pdfpages}
\usepackage{bm}
\usepackage{ifsym}
\newcommand{\hilite}[1]{{\color{red} \bf #1}}
\newcommand{\disruption}{\theta}
\begin{document}
\pagecolor{white}
\unitlength=1mm
\begin{center}
{\Large Some of these slides have been borrowed from Dr.\ Paul Lewis, Dr.\ Joe Felsenstein. Thanks!}
\vskip 15mm
\large Paul has many great tools for teaching phylogenetics at his web site: \\
\url{http://hydrodictyon.eeb.uconn.edu/people/plewis}
\end{center}


\myNewSlide
\includepdf[pages={2-39}]{../nonfreeimages/joe/joe-coal.pdf}
\includepdf[pages={46}]{../nonfreeimages/joe/joe-coal.pdf}
\section*{Inconsistency of estimation from concatenated gene sequences}
\citet{DegnanR2006} show that the most likely topology for a gene tree
is not necessarily the tree that agrees with the 
phylogenetic tree.

For some phylogenetic shapes (e.g. imbalanced trees
with short internal nodes) there exists (at least) one other tree shape
that has a higher probability of agreeing with a gene tree.

Argues for explicitly considering the coalescent process in phylogenetic inference.
\includepdf[pages={40-45}]{../nonfreeimages/joe/joe-coal.pdf}
%\includepdf[pages={48-50}]{../nonfreeimages/joe/joe-coal.pdf}
%\includepdf[pages={55}]{../nonfreeimages/joe/joe-coal.pdf}
\includepdf[pages={51-54}]{../nonfreeimages/joe/joe-coal.pdf}
\includepdf[pages={56-57}]{../nonfreeimages/joe/joe-coal.pdf}

\myNewSlide
\section*{``Skyline'' and ``Skyride'' plots in BEAST}
\begin{picture}(0,0)(0,0)
	\put(-50,-50){\makebox(0,0)[l]{\includegraphics[scale=1.5]{../images/MininBS2008.pdf}}}
	\put(-20,-120){{\tiny Figure from Minin, Bloomquist, and Suchard 2008}}
\end{picture}




\myNewSlide
\section*{BEST \citet{LiuP2007,EdwardsLP2007} }
\begin{compactitem}
	\item $X$ -- sequence data
	\item $G$ -- a genealogy (gene tree -- with branch lengths)
	\item $S$ -- a species tree
	\item $\bm{\theta}$ -- demographic parameters
	\item $\bm{\Lambda}$ -- parameters of molecular sequence evolution
\end{compactitem}

\begin{eqnarray*}
	\Pr(S,\bm{\theta}|X) & =  & \frac{\Pr(S,\bm{\theta})\Pr(X|S,\bm{\theta})}{\Pr(X)} \\
				& =  & \Pr(S)\Pr(\bm{\theta})\int\Pr(X|G)\Pr(G|S,\bm{\theta})dG \\
				& \propto  & \Pr(S)\Pr(\bm{\theta})\int\left[\int \Pr(X|G, \bm{\Lambda}) \Pr(\bm{\Lambda}) d\bm{\Lambda}\right]\Pr(G|S,\bm{\theta})dG 
\end{eqnarray*}

\myNewSlide
\section*{BEST -- importance sampling}
\begin{compactenum}
	\item Generate a collection of gene trees, $\bm{G}$, using an approximation of the coalescent prior
	\item Sample from the distribution of the species trees conditional on the gene trees, $\bm{G}$.
	\item Use ``importance weights'' to correct the sample for the fact that an approximate prior was used
\end{compactenum}

\myNewSlide
\section*{BEST -- importance sampling}
\begin{compactenum}
	\item Generate a collection of gene trees, $\bm{G}$, using an approximation of the coalescent prior
		\begin{compactenum}
		\item Use a tweaked version of MrBayes to sample $N$ sets of gene trees, $\bm{G}$, from \[\Pr^{\dag}(\bm{G}|X) = \frac{\Pr^{\dag}(\bm{G})\Pr(X|\bm{G})}{\Pr^{\dag}(X)}\]
			\item $Pr^{\dag}(\bm{G})$ is an approximate prior on gene trees from using a ``maximal'' species tree.
		\end{compactenum}
	\item Sample from the distribution of the species trees conditional on the gene trees, $\bm{G}$.
	\item Use ``importance weights'' to correct the sample for the fact that an approximate prior was used
\end{compactenum}

\myNewSlide
\section*{BEST -- importance sampling}
\begin{compactenum}
	\item Generate a collection of gene trees, $\bm{G}$, using an approximation of the coalescent prior
	\item Sample from the distribution of the species trees conditional on the gene trees, $\bm{G}$.
		\begin{compactenum}
		\item From each set of gene trees (${G}_j$ for $1\leq j\leq N$) generate $k$ species trees using coalescent theory:
			 \[\Pr(S_i|\bm{G}_j) = \frac{\Pr(S_i)\Pr(\bm{G}_j|S_i)}{\Pr(\bm{G}_j)}\]
		\end{compactenum}
	\item Use ``importance weights'' to correct the sample for the fact that an approximate prior was used
\end{compactenum}

\myNewSlide
\section*{BEST -- importance sampling}
\begin{compactenum}
	\item Generate a collection of gene trees, $\bm{G}$, using an approximation of the coalescent prior
	\item Sample from the distribution of the species trees conditional on the gene trees, $\bm{G}$.
	\item Use ``importance weights'' to correct the sample for the fact that an approximate prior was used
		\begin{compactenum}
		\item Estimate $\widehat{\Pr}(\bm{G}_j)$ by using the harmonic mean estimator from the MCMC in step 2.
		\item Compute a normalization factor \[\beta = \sum_{j=1}^N \frac{\widehat{\Pr}(\bm{G}_j)}{\Pr(\bm{G}_j)} \]
		  \item Reweight all sampled species trees by \[ \frac{\widehat{\Pr}(\bm{G}_j)}{\Pr(\bm{G}_j)}\beta\]
		\end{compactenum}
\end{compactenum}

\myNewSlide
\section*{BEST -- conclusions}
\begin{compactenum}
	\item very expensive computationally (long MrBayes runs are needed)
	\item should correctly deal with the variability in gene tree caused by the coalescent process.
\end{compactenum}

\myNewSlide
\section*{$\ast$BEST}
Similar model to BEST, but {\em much} more efficient implementation.

Both will be very sensitive to migration, but they represent the state-of-the-art for estimating species trees from gene trees.

\myNewSlide
\section*{Gene tree in a species tree w/ variable population size}
\begin{picture}(0,0)(0,0)
	\put(-50,-50){\makebox(0,0)[l]{\includegraphics[scale=1.5]{../images/HeledD2010Fig1.pdf}}}
	\put(-20,-120){{\tiny Figure from Heled and Drummond 2010}}
\end{picture}


\myNewSlide
\section*{Multiple gene tree in a species tree w/ variable population size}
\begin{picture}(0,0)(0,0)
	\put(-0,-130){\makebox(0,0)[l]{\includegraphics[scale=1.2]{../images/HeledD2010Fig9.pdf}}}
	\put(-20,-120){{\tiny Figure from Heled and Drummond 2010}}
\end{picture}


\myNewSlide
\bibliography{phylo}

\end{document}