\documentclass[landscape]{foils} 
\input{../common-preamble-start}
\input{../preamble.tex}
\usepackage{url}
\usepackage{hyperref}
\hypersetup{backref,  linkcolor=blue, citecolor=red, colorlinks=true, hyperindex=true}

\usepackage{pdfpages}
\begin{document}
\pagecolor{white}
\unitlength=1mm



\myNewSlide
\section*{Consistency Index (CI)}
\begin{itemize}
	\item minimum number of changes divided by the number required on the tree.
	\item CI=1 if there is no homoplasy
	\item negatively correlated with the number of species sampled
\end{itemize}

\myNewSlide
\section*{Retention Index (RI)}
 \[\mbox{RI} = \frac{\mbox{MaxSteps}-\mbox{ObsSteps}}{\mbox{MaxSteps}-\mbox{MinSteps}}\] 
\begin{itemize}
	\item defined to be 0 for parsimony uninformative characters
	\item RI=1 if the character fits perfectly
	\item RI=0 if the tree fits the character as poorly as possible
\end{itemize}

\myNewSlide
\section*{Qualitative description of parsimony}
\begin{itemize}
	\item Enables estimation of ancestral sequences.
	\item Even though parsimony always seeks to minimizes the number of changes, it can perform well even when changes are not rare. 
	\item Does not ``prefer'' to put changes on one branch over another
	\item Hard to characterize statistically
	\begin{compactitem}
		\item the set of conditions in which parsimony is guaranteed to work well is very restrictive (low probability of change and not too much branch length heterogeneity);
	   \item Parsimony often performs well in simulation studies (even when outside the zones in which it is guaranteed to work); 
	   	\item Estimates of the tree can be extremely biased.
	\end{compactitem}	   
\end{itemize}

\myNewSlide
\section*{Long branch attraction}
\begin{picture}(0,0)(0,0)  
\put(20,-60){\makebox(0,0)[l]{\includegraphics[scale=1.1]{../images/fels_tree.pdf}}}
\put(100, -0){\normalsize Felsenstein, J. 1978. Cases in which}
\put(100, -10){\normalsize  parsimony or compatibility methods will be}
\put(100, -20){\normalsize positively misleading. {\em Systematic Zoology}}
\put(100, -30){\normalsize {\bf 27}: 401-410.}
\put(20, -50){\small 1.0}
\put(68, -50){\small 1.0}
\put(44, -120){\small 0.01}
\put(55, -115){\small 0.01}
\put(30, -115){\small 0.01}
\end{picture}

\myNewSlide
\section*{Long branch attraction}
\begin{picture}(0,0)(0,0)  
\put(20,-60){\makebox(0,0)[l]{\includegraphics[scale=1.1]{../images/fels_tree.pdf}}}
\put(38,-103){\makebox(0,0)[l]{\includegraphics[scale=1]{../images/lightning.pdf}}}
\put(100, -0){\normalsize Felsenstein, J. 1978. Cases in which}
\put(100, -10){\normalsize  parsimony or compatibility methods will be}
\put(100, -20){\normalsize positively misleading. {\em Systematic Zoology}}
\put(100, -30){\normalsize {\bf 27}: 401-410.}
\put(100, -60){\normalsize The probability of a parsimony informative}
\put(100, -70){\normalsize site due to inheritance is very low,}
 \put(100, -80){\normalsize (roughly 0.0003).}
\put(16, -0){\large A}
\put(75, -0){\large G}
\put(35, -123){\large A}
\put(56, -123){\large G}
\put(20, -50){\small 1.0}
\put(68, -50){\small 1.0}
\put(44, -120){\small 0.01}
\put(55, -115){\small 0.01}
\put(30, -115){\small 0.01}
\end{picture}

\myNewSlide
\section*{Long branch attraction}
\begin{picture}(0,0)(0,0)  
\put(20,-60){\makebox(0,0)[l]{\includegraphics[scale=1.1]{../images/fels_tree.pdf}}}
\put(50,-50){\rotatebox{90}{\includegraphics{../images/lightning.pdf}}}
\put(10,-50){\rotatebox{90}{\includegraphics{../images/lightning.pdf}}}
\put(100, -0){\normalsize Felsenstein, J. 1978. Cases in which}
\put(100, -10){\normalsize  parsimony or compatibility methods will be}
\put(100, -20){\normalsize positively misleading. {\em Systematic Zoology}}
\put(100, -30){\normalsize {\bf 27}: 401-410.}
\put(100, -60){\normalsize The probability of a parsimony informative}
\put(100, -70){\normalsize site due to inheritance is very low,}
\put(100, -80){\normalsize (roughly 0.0003).}
\put(100, -100){\normalsize The probability of a misleading parsimony}
\put(100, -110){\normalsize informative site due to parallelism is much}
\put(100, -120){\normalsize higher (roughly 0.008).}
\put(16, -0){\large A}
\put(75, -0){\large A}
\put(35, -123){\large G}
\put(56, -123){\large G}
\put(20, -50){\small 1.0}
\put(68, -50){\small 1.0}
\put(44, -120){\small 0.01}
\put(55, -115){\small 0.01}
\put(30, -115){\small 0.01}
\end{picture}

\myNewSlide
\section*{Long branch attraction}
Parsimony is almost guaranteed to get this tree wrong.\\
\begin{picture}(0,0)(0,0)  
\put(20,-60){\makebox(0,0)[l]{\includegraphics[scale=1.1]{../images/fels_tree.pdf}}}
\put(120,-60){\makebox(0,0)[l]{\includegraphics[scale=1.1]{../images/inferred_fels.pdf}}}
\put(16, -0){\large 1}
\put(75, -0){\large 3}
\put(35, -123){\large 2}
\put(56, -123){\large 4}
\put(42, -133){\large True}
\put(115, -20){\large 1}
\put(200, -20){\large 3}
\put(150, -103){\large 2}
\put(165, -103){\large 4}
\put(152, -123){\large Inferred}
\end{picture}

\myNewSlide
\section*{Inconsistency}
\begin{itemize}	
	\item Statistical Consistency (roughly speaking) is converging to the true answer as the amount of data goes to $\infty$.
	\item Parsimony based tree inference is {\em not} consistent for some tree shapes.  In fact it can be ``positively misleading'':
	 \begin{itemize}	
		\item ``Felsenstein zone'' tree
		\item Many clocklike trees with short internal branch lengths and long terminal branches (Penny {\em et al.}, 1989, Huelsenbeck and Lander, 2003).
	\end{itemize}
	\item Methods for assessing confidence (e.g. bootstrapping) will indicate that you should be very confident in the wrong answer.
\end{itemize}


\myNewSlide
If the data is generated such that:

\begin{eqnarray*}
\Pr
\left(
\begin{array}{c}
 A \\
 A \\
 G \\   
 G 
\end{array}
\right)\approx 0.0003 &\mbox{and} & \Pr
\left(
\begin{array}{c}
 A \\
 G \\
 G \\   
 A 
\end{array}
\right) \approx 0.008
\end{eqnarray*}
then how can we hope to infer the tree ((1,2),3,4) ?



\myNewSlide

Looking at the data in ``bird's eye'' view (using Mesquite):\\
\begin{picture}(0,0)(100,0)  
\put(0,-50){\makebox(0,0)[l]{\includegraphics[scale=2]{../images/birdseyefelsenstein.pdf}}}
\end{picture}

\myNewSlide

Looking at the data in ``bird's eye'' view (using Mesquite):\\
\begin{picture}(0,0)(100,0)  
\put(0,-50){\makebox(0,0)[l]{\includegraphics[scale=2]{../images/birdseyefelsenstein.pdf}}}
\end{picture}

\vskip 8cm 
We see that sequences 1 and 4 are clearly very different.\\
Perhaps we can estimate the tree if we use the branch length information from the sequences...



\myNewSlide

\section*{Distance-based approaches to inferring trees}
\begin{itemize}
	\item Convert the raw data (sequences) to a pairwise distances
	\item Try to find a tree that explains these distances.
	\item {\em Not} simply clustering the most similar sequences.
\end{itemize}



\myNewSlide

\begin{tabular}{lcccccccccc}
 &1&2&3&4&5&6&7&8&9&10\\
 Species 1\hskip 2mm& C & G  & A & C & C & A & G & G & T & A\\
 Species 2\hskip 2mm& C & G  & A & C & C & A & G & G & T & A\\
 Species 3\hskip 2mm& C & G  & G & T & C & C & G & G & T & A\\
 Species 4\hskip 2mm& C & G  & G & C & C & A & T & G & T & A \\
\end{tabular}\par
Can be converted to a distance matrix:\par
\begin{tabular}{c|cccc|}
& Species 1& Species 2 & Species 3 &Species 4\\
\hline Species 1\hskip 2mm& 0 & 0 & 0.3 & 0.2 \\
Species 2\hskip 2mm& 0 & 0 &  0.3 & 0.2 \\
Species 3\hskip 2mm& 0.3 & 0.3 & 0 &0.3 \\
Species 4\hskip 2mm& 0.2 & 0.2 & 0.3 &0\\ \hline
\end{tabular}

\myNewSlide
Note that the distance matrix is symmetric.\par
\begin{tabular}{c|cccc|}
& Species 1& Species 2 & Species 3 &Species 4\\
\hline Species 1\hskip 2mm& 0 & 0 & 0.3 & 0.2 \\
Species 2\hskip 2mm& 0 & 0 &  0.3 & 0.2 \\
Species 3\hskip 2mm& 0.3 & 0.3 & 0 &0.3 \\
Species 4\hskip 2mm& 0.2 & 0.2 & 0.3 &0\\ \hline
\end{tabular}

\myNewSlide
\ldots so we can just use the lower triangle.\par
\begin{tabular}{c|ccc|}
& Species 1& Species 2 & Species 3 \\
\hline Species 2\hskip 2mm& 0 &   &    \\
Species 3\hskip 2mm& 0.3 & 0.3 &   \\
Species 4\hskip 2mm& 0.2 & 0.2 & 0.3 \\
\hline
\end{tabular}
\par
Can we find a tree that would predict these observed character divergences?
\myNewSlide
\begin{tabular}{c|ccc|}
& Species 1& Species 2 & Species 3 \\
\hline Species 2\hskip 2mm& 0 &   &    \\
Species 3\hskip 2mm& 0.3 & 0.3 &   \\
Species 4\hskip 2mm& 0.2 & 0.2 & 0.3 \\ \hline
\end{tabular}\par
Can we find a tree that would predict these observed character divergences?
\begin{picture}(0,0)(0,0)  
\put(0,-50){\makebox(0,0)[l]{\includegraphics[scale=1.2]{../images/simple_unrooted.pdf}}}
\put(-15,-30){Sp. 1}
\put(-15,-80){Sp. 2}
\put(65,-30){Sp. 3}
\put(65,-80){Sp. 4}
\normalsize
\put(0,-48){\bf 0.0}
\put(15,-66){\bf 0.0}
\put(22,-45){\bf 0.1}
\put(55,-45){\bf 0.2}
\put(58,-65){\bf 0.1}
\end{picture}

\myNewSlide
\begin{picture}(0,0)(0,0)  
\put(50,-30){\makebox(0,0)[l]{\includegraphics[scale=1.2]{../images/path_lengths.pdf}}}
\put(80,-10){$a$}
\put(80,-50){$b$}
\put(140,-10){$c$}
\put(140,-50){$d$}
\put(110,-25){$i$}
\put(150, -120){\begin{tabular}{c|ccc|}
& 1& 2 & 3 \\
\hline 2\hskip 2mm& $d_{12}$ &   &    \\
3\hskip 2mm& $d_{13}$ & $d_{23}$ &   \\
4\hskip 2mm& $d_{14}$ & $d_{24}$ &$d_{34}$ \\ \hline
\end{tabular}\par
 }
\put(185, -90){data}
\put(37, -90){parameters}
\put(10, -100){$p_{12} =  a+b$}
\put(10, -110){$p_{13}  = a+i+c$}
\put(10, -120){$p_{14}  = a+i+d$}
\put(10, -130){$p_{23} =  b+i+c$}
\put(10, -140){$p_{23}  = b+i+d$}
\put(10, -150){$p_{34}  = c+d$}
\end{picture}

\myNewSlide
\large
If our pairwise distance measurements were error-free estimates of the {\em evolutionary
distance} between the sequences, then we could always infer the tree from the distances.

The evolutionary distance is the number of mutations that have occurred along the path
that connects two tips. 

We hope the distances that we measure can produce good estimates of the evolutionary
distance, but we know that they cannot be perfect.

\myNewSlide
\section*{Intuition of sequence divergence vs evolutionary distance}
\large
\begin{picture}(0,0)(10,0)  
\put(20,-70){\makebox(0,0)[l]{\includegraphics[scale=1.]{../images/distance_correction.pdf}}}
\put(-10,-70){ $p$-dist}
\put(115, -133){Evolutionary distance}
\put(230, -138){$\infty$}
\put(188, -10){This can't be right!}
\end{picture}

\myNewSlide
\section*{Sequence divergence vs evolutionary distance}
\begin{picture}(0,0)(0,0)  
\put(20,-70){\makebox(0,0)[l]{\includegraphics[scale=1.]{../images/jc_distance_correction.pdf}}}
\put(-10,-70){ $p$-dist}
\put(115, -133){Evolutionary distance}
\put(230, -138){$\infty$}
\put(172, -30){the $p$-dist}
\put(172, -40){``levels off''}
\end{picture}


\myNewSlide
\section*{``Multiple hits'' problem (also known as saturation)}
\begin{itemize}
	\item Levelling off of sequence divergence vs time plot is caused by multiple substitutions affecting the same site in the DNA.
	\item At large distances the ``raw'' sequence divergence (also known as the $p$-distance or Hamming distance) is a poor estimate of the true evolutionary distance.
	 \item Large $p$-distances respond more to model-based correction -- and there is a larger error associated with the correction.
\end{itemize}

\myNewSlide
\begin{picture}(0,0)(50,0)  
\put(20,-70){\makebox(0,0)[l]{\includegraphics[scale=1.]{../images/sim20BaseSeq.pdf}}}
\end{picture}




\myNewSlide
\section*{Distance corrections}
\begin{compactitem}
	\item applied to distances before tree estimation,
	\item converts raw distances to an estimate of the evolutionary distance
\end{compactitem}
\[d = -\frac{3}{4}\;\ln\left(\frac{4c}{3}-1\right) \]

\begin{picture}(0,0)(0,0)  
\put(150, -60){\begin{tabular}{c|ccc|}
& 1& 2 & 3 \\
\hline 2\hskip 2mm& $d_{12}$ &   &    \\
3\hskip 2mm& $d_{13}$ & $d_{23}$ &   \\
4\hskip 2mm& $d_{14}$ & $d_{24}$ &$d_{34}$ \\ \hline
\end{tabular}\par
 }
\put(150, -20){corrected distances}

\put(0, -60){\begin{tabular}{c|ccc|}
& 1& 2 & 3 \\
\hline 2\hskip 2mm& $c_{12}$ &   &    \\
3\hskip 2mm& $c_{13}$ & $c_{23}$ &   \\
4\hskip 2mm& $c_{14}$ & $c_{24}$ &$c_{34}$ \\ \hline
\end{tabular}\par
 }
\put(0, -20){``raw'' $p$-distances}
\end{picture}



\myNewSlide
\[d = -\frac{3}{4}\;\ln\left(1-\frac{4c}{3}\right) \]

\begin{picture}(0,0)(0,0)  
\put(150, -60){\begin{tabular}{c|ccc|}
& 1& 2 & 3 \\
\hline 2\hskip 2mm& $0$ &   &    \\
3\hskip 2mm& $0.383$ & $0.383$ &   \\
4\hskip 2mm& $0.233$ & $0.233$ &$0.383$ \\ \hline
\end{tabular}\par
 }
\put(150, -20){corrected distances}

\put(0, -60){\begin{tabular}{c|ccc|}
& 1& 2 & 3 \\
\hline 2\hskip 2mm& $0.0$ &   &    \\
3\hskip 2mm& $0.3$ & $0.3$ &   \\
4\hskip 2mm& $0.2$ & $0.2$ &$0.3$ \\ \hline
\end{tabular}\par
 }
\put(0, -20){``raw'' $p$-distances}
\end{picture}



\myNewSlide
\section*{Least Squares Branch Lengths}
\Large
\[\mbox{Sum of Squares} = \sum_{i}\sum_{j}\frac{(p_{ij}-d_{ij})^{2}}{\sigma_{ij}^{k}} \]
\begin{itemize}
	\item minimize discrepancy between path lengths and observed distances
	\item ${\sigma_{ij}^{k}} $ is used to ``downweight'' distance estimates with high variance
\end{itemize}


\myNewSlide
\section*{Least Squares Branch Lengths}
\Large
\[\mbox{Sum of Squares} = \sum_{i}\sum_{j}\frac{(p_{ij}-d_{ij})^{2}}{\sigma_{ij}^{k}} \]
\begin{itemize}
	\item in unweighted least-squares (Cavalli-Sforza \& Edwards, 1967): $k=0$ 
	\item in the method Fitch-Margoliash (1967):  $k=2$ and $\sigma_{ij} = d_{ij}$ 
\end{itemize}

\myNewSlide
\section*{Poor fit using arbitrary branch lengths}
\large
\begin{tabular}{|c|c|c|c|}
\hline Species& $d_{ij}$ & $p_{ij}$ & $(p-d)^{2}$\\
\hline Hu-Ch & 0.09267 & 0.2 & 0.01152 \\
\hline Hu-Go & 0.10928  & 0.3 & 0.03637 \\
\hline Hu-Or & 0.17848  & 0.4 & 0.04907 \\
\hline Hu-Gi & 0.20420  & 0.4 & 0.03834 \\
\hline Ch-Go & 0.11440  & 0.3 & 0.03445 \\
\hline Ch-Or & 0.19413  & 0.4 & 0.04238 \\
\hline Ch-Gi & 0.21591  & 0.4 & 0.03389 \\
\hline Go-Or & 0.18836  & 0.3 & 0.01246 \\
\hline Go-Gi & 0.21592  & 0.3 & 0.00707 \\
\hline Or-Gi & 0.21466  & 0.2 & 0.00021  \\
\hline &   &  S.S.  & {\bf 0.26577 } \\
\hline
\end{tabular}

\begin{picture}(0,0)(0,0)  
\put(151,90){\makebox(0,0)[l]{\includegraphics[scale=1.]{../images/primate_tree_shape.pdf}}}
\put(141,80){Hu}
\put(170,50){Ch}
\put(185,125){Go}
\put(232,80){Or}
\put(203,50){Gi}
\put(157,85){\normalsize 0.1} %hu
\put(165,65){\normalsize 0.1} %ch
\put(175,93){\normalsize 0.1} %hu-ch
\put(197,93){\normalsize 0.1} %or-gi
\put(193,109){\normalsize 0.1} %go
\put(215,85){\normalsize 0.1} %or
\put(209,65){\normalsize 0.1} %gi
\end{picture}

\myNewSlide
\section*{Optimizing branch lengths yields the least-squares score}
\normalsize
\begin{tabular}{|c|c|c|c|}
\hline Species& $d_{ij}$ & $p_{ij}$ & $(p-d)^{2}$\\
\hline Hu-Ch & 0.09267 & 0.09267  & 0.000000000\\
\hline Hu-Go & 0.10928  & 0.10643  & 0.000008123\\
\hline Hu-Or & 0.17848  & 0.18026  & 0.000003168\\
\hline Hu-Gi & 0.20420  & 0.20528  & 0.000001166\\
\hline Ch-Go & 0.11440  & 0.11726  & 0.000008180\\
\hline Ch-Or & 0.19413  & 0.19109  & 0.000009242\\
\hline Ch-Gi & 0.21591  & 0.21611  & 0.000000040\\
\hline Go-Or & 0.18836  & 0.18963  & 0.000001613\\
\hline Go-Gi & 0.21592  & 0.21465  & 0.000001613\\
\hline Or-Gi & 0.21466  & 0.21466  & 0.000000000 \\
\hline &   &  S.S.  & {\bf 0.000033144 } \\
\hline
\end{tabular}

\begin{picture}(0,0)(0,0)  
\put(161,70){\makebox(0,0)[l]{\includegraphics[scale=1.]{../images/primate_tree_shape.pdf}}}
\put(151,60){Hu}
\put(180,30){Ch}
\put(195,105){Go}
\put(242,60){Or}
\put(213,30){Gi}
\put(164,63){\small 0.04092} %hu
\put(165,45){\small 0.05175} %ch
\put(177,73){\small 0.00761} %hu-ch
\put(207,73){\small 0.03691} %or-gi
\put(203,89){\small 0.05790} %go
\put(222,63){\small 0.09482} %or
\put(219,45){\small 0.11984} %gi
\end{picture}

\myNewSlide
\section*{Least squares as an optimality criterion}
\begin{picture}(0,0)(0,0)  
\put(121,-70){\makebox(0,0)[l]{\includegraphics[scale=1.]{../images/primate_tree_shape.pdf}}}
\put(111,-80){Hu}
\put(140,-110){\color{red} Ch}
\put(155,-35){\color{red} Go}
\put(203,-80){Or}
\put(173,-110){Gi}
\put(124,-77){\small 0.04092} %hu
\put(125,-95){\small 0.05175} %ch
\put(137,-67){\small 0.00761} %hu-ch
\put(167,-67){\small 0.03691} %or-gi
\put(163,-51){\small 0.05790} %go
\put(182,-77){\small 0.09482} %or
\put(179,-95){\small 0.11984} %gi
\put(11,-70){\makebox(0,0)[l]{\includegraphics[scale=1.]{../images/primate_tree_shape.pdf}}}
\put(1,-80){Hu}
\put(30,-110){\color{red} Go}
\put(45,-35){\color{red} Ch}
\put(93,-80){Or}
\put(63,-110){Gi}
\put(14,-77){\small 0.04742} %hu
\put(15,-95){\small 0.05175} %go
\put(25,-67){\small -0.00701} %hu-go
\put(57,-67){\small 0.04178} %or-gi
\put(53,-51){\small 0.05591} %ch
\put(72,-77){\small 0.09482} %or
\put(69,-95){\small 0.11984} %gi
\put(25, -15){\large SS = 0.00034}
\put(135, -15){\large SS = 0.0003314}
\put(145, -25){\large (best tree)}
\end{picture}

\myNewSlide
\section*{Minimum evolution optimality criterion}
\begin{picture}(0,0)(0,0)  
\put(121,-70){\makebox(0,0)[l]{\includegraphics[scale=1.]{../images/primate_tree_shape.pdf}}}
\put(111,-80){Hu}
\put(140,-110){\color{red} Ch}
\put(155,-35){\color{red} Go}
\put(203,-80){Or}
\put(173,-110){Gi}
\put(124,-77){\small 0.04092} %hu
\put(125,-95){\small 0.05175} %ch
\put(137,-67){\small 0.00761} %hu-ch
\put(167,-67){\small 0.03691} %or-gi
\put(163,-51){\small 0.05790} %go
\put(182,-77){\small 0.09482} %or
\put(179,-95){\small 0.11984} %gi
\put(11,-70){\makebox(0,0)[l]{\includegraphics[scale=1.]{../images/primate_tree_shape.pdf}}}
\put(1,-80){Hu}
\put(30,-110){\color{red} Go}
\put(45,-35){\color{red} Ch}
\put(93,-80){Or}
\put(63,-110){Gi}
\put(14,-77){\small 0.04742} %hu
\put(15,-95){\small 0.05175} %go
\put(25,-67){\small -0.00701} %hu-go
\put(57,-67){\small 0.04178} %or-gi
\put(53,-51){\small 0.05591} %ch
\put(72,-77){\small 0.09482} %or
\put(69,-95){\small 0.11984} %gi
\put(20, -5){\large Sum of branch lengths}
\put(35, -15){\large =0.41152}
\put(130, -5){\large Sum of branch lengths}
\put(145, -15){\large =0.40975}
\put(145, -25){\large (best tree)}
\put(-20, -125){\large We still use least squares branch lengths when we use Minimum Evolution }
\end{picture}

\myNewSlide
\section*{Huson and Steel -- distances that perfectly mislead}
\large
\citet{HusonS2004} point out problems when our pairwise distances have errors (do not reflect true evolutionary distances).
Consider:
\begin{table}[htdp]
\begin{center}
\begin{tabular}{|c|ccccccc|p{1cm}|c|cccc|}
\hline
& & & & & & & & & &\multicolumn{4}{c|}{Taxa} \\
Taxon & \multicolumn{7}{c|}{Characters} & & Taxa & A& B & C &D \\
\hline
A & A & A & C & A & A & C & C & & A & - & 6 & 6 & 5 \\
B & A & C & A & C & C & A & A & & B & 6 & - & 5 & 6\\
C & C & A & G & G & G & A & A & & C & 6 & 5  & - & 6\\
D & C & G & A & A & A & G & G & & D & 5 & 6 & 6 & - \\
\hline
\end{tabular}
\end{center}
\label{default}
\end{table}%
Homoplasy-free on tree $AB|CD$, but additive on tree $AD|BC$  (and not additive on any other tree).

\myNewSlide
\section*{Huson and Steel -- distances that perfectly mislead}
Clearly, the previous matrix was contrived and not typical of realistic data. \\
Would we ever expect to see additive distances on the {\bf wrong} tree as the result of a reasonable evolutionary process?

Yes.  

\citet{HusonS2004} show that under the equal-input model(more on this later), the {\bf uncorrected} distances can be additive on the wrong tree leading to long-branch attraction.
The result holds even if the number of characters $\rightarrow \infty$

\myNewSlide
\section*{Failure to correct distance sufficiently leads to poor performance}
 ``Under-correcting''  will underestimate long evolutionary distances more than short distances\\
\begin{picture}(0,0)(0,0)  
\put(70,-70){\makebox(0,0)[l]{\includegraphics[scale=1.]{../images/felsenstein_undercorrected.pdf}}}
\end{picture}

\myNewSlide
\section*{Failure to correct distance sufficiently leads to poor performance}
 The result is the classic ``long-branch attraction'' phenomenon.\\
\begin{picture}(0,0)(0,0)  
\put(80,-70){\makebox(0,0)[l]{\includegraphics[scale=1.]{../images/fels_inferred.pdf}}}
\end{picture}

\myNewSlide
\section*{Distance methods -- summary}
We can:
\begin{compactitem}
	\item summarize a dataset as a matrix of distances or dissimilarities.
	\item correct these distances for unseen character state changes.
	\item estimate a tree by finding the tree with path lengths that are ``closest'' to the corrected distances. 
\end{compactitem}



\myNewSlide
{
\setlength{\unitlength}{.06cm}
\begin{picture}(100,100)(-100,-100)
	\thicklines
	\put(-105,3){A}
	\put(-105,-99){B}
	\put(70,3){C}
	\put(70,-99){D}
	\put(-75,-17){$a$}
	\put(-70,-72){$b$}
	\put(40,-17){$c$}
	\put(33,-74){$d$}
	\put(-15,-40){$i$}
	\put(-95,0){\line(1,-1){45}}
	\put(-95,-90){\line(1,1){45}}
	\put(-50,-45){\line(1,0){70}}
	\put(20,-45){\line(1,1){45}}
	\put(20,-45){\line(1,-1){45}}
\put(160,-45){\begin{tabular}{c|ccc}
 & A & B & C\\
 \hline
 B & $d_{AB}$ & & \\
 C & $d_{AC}$ & $d_{BC}$ & \\
 D & $d_{AD}$ & $d_{BD}$ & $d_{CD}$ \\
\end{tabular}
}
\end{picture}\\
}
If the tree above is correct then:
\begin{eqnarray*}
	p_{AB} & = & a + b \\
	p_{AC} & = & a + i + c \\
	p_{AD} & = & a + i + d \\
	p_{BC} & = & b + i + c \\
	p_{BD} & = & b + i + d \\
	p_{CD} & = & c + d \\
\end{eqnarray*}


\myNewSlide
\begin{figure}
\begin{center}
\setlength{\unitlength}{.06cm}
\begin{picture}(200,100)(0,-100)
	\thicklines
	\put(-105,3){A}
	\put(-105,-99){B}
	\put(70,3){C}
	\put(70,-99){D}
	\put(-75,-17){$a$}
	\put(-70,-72){$b$}
	\put(40,-17){$c$}
	\put(33,-74){$d$}
	\put(-15,-40){$i$}
	\put(-95,0){\line(1,-1){45}}
	\put(-95,-90){\line(1,1){45}}
	\put(-50,-45){\line(1,0){70}}
	\put(20,-45){\line(1,1){45}}
	\put(20,-45){\line(1,-1){45}}

	\put(-95,15){\color{darkgreen}\line(1,-1){45}}
	\put(20,-30){\color{darkgreen}\line(1,1){45}}
	\put(-50,-30){\color{darkgreen}\line(1,0){70}}

\put(160,-45){\begin{tabular}{c|ccc}
 & A & B & C\\
 \hline
 B & $d_{AB}$ & & \\
 C & {\color{darkgreen}$d_{AC}$} & $d_{BC}$ & \\
 D & $d_{AD}$ & $d_{BD}$ & $d_{CD}$ \\
\end{tabular}
}
\end{picture}\\
\end{center}
\end{figure}
\vskip 2cm
\begin{center}

{\color{darkgreen}$d_{AC}$}

\end{center}


\myNewSlide
\begin{figure}
\begin{center}
\setlength{\unitlength}{.06cm}
\begin{picture}(200,100)(0,-100)
	\thicklines
	\put(-105,3){A}
	\put(-105,-99){B}
	\put(70,3){C}
	\put(70,-99){D}
	\put(-75,-17){$a$}
	\put(-70,-72){$b$}
	\put(40,-17){$c$}
	\put(33,-74){$d$}
	\put(-15,-40){$i$}
	\put(-95,0){\line(1,-1){45}}
	\put(-95,-90){\line(1,1){45}}
	\put(-50,-45){\line(1,0){70}}
	\put(20,-45){\line(1,1){45}}
	\put(20,-45){\line(1,-1){45}}

	\put(-95,15){\color{darkgreen}\line(1,-1){45}}
	\put(20,-30){\color{darkgreen}\line(1,1){45}}
	\put(-50,-30){\color{darkgreen}\line(1,0){70}}

	\put(-95,-105){\color{darkgreen}\line(1,1){45}}
	\put(20,-60){\color{darkgreen}\line(1,-1){45}}
	\put(-50,-60){\color{darkgreen}\line(1,0){70}}

\put(160,-45){\begin{tabular}{c|ccc}
 & A & B & C\\
 \hline
 B & $d_{AB}$ & & \\
 C & {\color{darkgreen}$d_{AC}$} & $d_{BC}$ & \\
 D & $d_{AD}$ & {\color{darkgreen}$d_{BD}$} & $d_{CD}$ \\
\end{tabular}
}
\end{picture}\\
\end{center}
\end{figure}
\vskip 2cm
\begin{center}

{\color{darkgreen}$d_{AC} + d_{BD}$}

\end{center}

\myNewSlide
\begin{figure}
\begin{center}
\setlength{\unitlength}{.06cm}
\begin{picture}(200,100)(0,-100)
	\thicklines
	\put(-105,3){A}
	\put(-105,-99){B}
	\put(70,3){C}
	\put(70,-99){D}
	\put(-75,-17){$a$}
	\put(-70,-72){$b$}
	\put(40,-17){$c$}
	\put(33,-74){$d$}
	\put(-15,-40){$i$}
	\put(-95,0){\line(1,-1){45}}
	\put(-95,-90){\line(1,1){45}}
	\put(-50,-45){\line(1,0){70}}
	\put(20,-45){\line(1,1){45}}
	\put(20,-45){\line(1,-1){45}}

	\put(-95,15){\color{darkgreen}\line(1,-1){45}}
	\put(20,-30){\color{darkgreen}\line(1,1){45}}
	\put(-50,-30){\color{darkgreen}\line(1,0){70}}

	\put(-95,-105){\color{darkgreen}\line(1,1){45}}
	\put(20,-60){\color{darkgreen}\line(1,-1){45}}
	\put(-50,-60){\color{darkgreen}\line(1,0){70}}

	\put(-110,-90){\color{red}\line(1,1){45}}
	\put(-110,0){\color{red}\line(1,-1){45}}

\put(160,-45){\begin{tabular}{c|ccc}
 & A & B & C\\
 \hline
 B & {\color{red}$d_{AB}$} & & \\
 C & {\color{darkgreen}$d_{AC}$} & $d_{BC}$ & \\
 D & $d_{AD}$ & {\color{darkgreen}$d_{BD}$} & $d_{CD}$ \\
\end{tabular}
}
\end{picture}\\
\end{center}
\end{figure}
\vskip 2cm
\begin{center}

{\color{darkgreen}$d_{AC} + d_{BD}$}
\par
{\color{red}$d_{AB}$}
\end{center}



\myNewSlide
\begin{figure}
\begin{center}
\setlength{\unitlength}{.06cm}
\begin{picture}(200,100)(0,-100)
	\thicklines
	\put(-105,3){A}
	\put(-105,-99){B}
	\put(70,3){C}
	\put(70,-99){D}
	\put(-75,-17){$a$}
	\put(-70,-72){$b$}
	\put(40,-17){$c$}
	\put(33,-74){$d$}
	\put(-15,-40){$i$}
	\put(-95,0){\line(1,-1){45}}
	\put(-95,-90){\line(1,1){45}}
	\put(-50,-45){\line(1,0){70}}
	\put(20,-45){\line(1,1){45}}
	\put(20,-45){\line(1,-1){45}}

	%\put(-95,15){\color{darkgreen}\line(1,-1){45}}
	\put(20,-30){\color{darkgreen}\line(1,1){45}}
	\put(-50,-30){\color{darkgreen}\line(1,0){70}}

	%\put(-95,-105){\color{darkgreen}\line(1,1){45}}
	\put(20,-60){\color{darkgreen}\line(1,-1){45}}
	\put(-50,-60){\color{darkgreen}\line(1,0){70}}

	%\put(-110,-90){\color{red}\line(1,1){45}}
	%\put(-110,0){\color{red}\line(1,-1){45}}

\put(160,-45){\begin{tabular}{c|ccc}
 & A & B & C\\
 \hline
 B & {\color{red}$d_{AB}$} & & \\
 C & {\color{darkgreen}$d_{AC}$} & $d_{BC}$ & \\
 D & $d_{AD}$ & {\color{darkgreen}$d_{BD}$} & $d_{CD}$ \\
\end{tabular}
}
\end{picture}\\
\end{center}
\end{figure}
\vskip 2cm
\begin{center}

{\color{darkgreen}$d_{AC} + d_{BD}$} {\color{red}$- d_{AB}$}
\par
\end{center}


\myNewSlide
\begin{figure}
\begin{center}
\setlength{\unitlength}{.06cm}
\begin{picture}(200,100)(0,-100)
	\thicklines
	\put(-105,3){A}
	\put(-105,-99){B}
	\put(70,3){C}
	\put(70,-99){D}
	\put(-75,-17){$a$}
	\put(-70,-72){$b$}
	\put(40,-17){$c$}
	\put(33,-74){$d$}
	\put(-15,-40){$i$}
	\put(-95,0){\line(1,-1){45}}
	\put(-95,-90){\line(1,1){45}}
	\put(-50,-45){\line(1,0){70}}
	\put(20,-45){\line(1,1){45}}
	\put(20,-45){\line(1,-1){45}}

	%\put(-95,15){\color{darkgreen}\line(1,-1){45}}
	\put(20,-30){\color{darkgreen}\line(1,1){45}}
	\put(-50,-30){\color{darkgreen}\line(1,0){70}}

	%\put(-95,-105){\color{darkgreen}\line(1,1){45}}
	\put(20,-60){\color{darkgreen}\line(1,-1){45}}
	\put(-50,-60){\color{darkgreen}\line(1,0){70}}

	\put(35,-45){\color{red}\line(1,1){45}}
	\put(35,-45){\color{red}\line(1,-1){45}}

\put(160,-45){\begin{tabular}{c|ccc}
 & A & B & C\\
 \hline
 B & {\color{red}$d_{AB}$} & & \\
 C & {\color{darkgreen}$d_{AC}$} & $d_{BC}$ & \\
 D & $d_{AD}$ & {\color{darkgreen}$d_{BD}$} & {\color{red}$d_{CD}$} \\
\end{tabular}
}
\end{picture}\\
\end{center}
\end{figure}
\vskip 2cm
\begin{center}

{\color{darkgreen}$d_{AC} + d_{BD}$} {\color{red}$- d_{AB}$}
\par
{\color{red}$d_{CD}$}
\end{center}


\myNewSlide
\begin{figure}
\begin{center}
\setlength{\unitlength}{.06cm}
\begin{picture}(200,100)(0,-100)
	\thicklines
	\put(-105,3){A}
	\put(-105,-99){B}
	\put(70,3){C}
	\put(70,-99){D}
	\put(-75,-17){$a$}
	\put(-70,-72){$b$}
	\put(40,-17){$c$}
	\put(33,-74){$d$}
	\put(-15,-40){$i$}
	\put(-95,0){\line(1,-1){45}}
	\put(-95,-90){\line(1,1){45}}
	\put(-50,-45){\line(1,0){70}}
	\put(20,-45){\line(1,1){45}}
	\put(20,-45){\line(1,-1){45}}

	%\put(-95,15){\color{darkgreen}\line(1,-1){45}}
	%\put(20,-30){\color{darkgreen}\line(1,1){45}}
	\put(-50,-30){\color{darkgreen}\line(1,0){70}}

	%\put(-95,-105){\color{darkgreen}\line(1,1){45}}
	%\put(20,-60){\color{darkgreen}\line(1,-1){45}}
	\put(-50,-60){\color{darkgreen}\line(1,0){70}}

	%\put(35,-45){\color{red}\line(1,1){45}}
	%\put(35,-45){\color{red}\line(1,-1){45}}

\put(160,-45){\begin{tabular}{c|ccc}
 & A & B & C\\
 \hline
 B & {\color{red}$d_{AB}$} & & \\
 C & {\color{darkgreen}$d_{AC}$} & $d_{BC}$ & \\
 D & $d_{AD}$ & {\color{darkgreen}$d_{BD}$} & {\color{red}$d_{CD}$} \\
\end{tabular}
}
\end{picture}\\
\end{center}
\end{figure}
\vskip 2cm
\begin{center}

{\color{darkgreen}$d_{AC} + d_{BD}$} {\color{red}$- d_{AB}-d_{CD}$}
\par
\end{center}




\myNewSlide
\begin{figure}
\begin{center}
\setlength{\unitlength}{.06cm}
\begin{picture}(200,100)(0,-100)
	\thicklines
	\put(-105,3){A}
	\put(-105,-99){B}
	\put(70,3){C}
	\put(70,-99){D}
	\put(-75,-17){$a$}
	\put(-70,-72){$b$}
	\put(40,-17){$c$}
	\put(33,-74){$d$}
	\put(-15,-40){$i$}
	\put(-95,0){\line(1,-1){45}}
	\put(-95,-90){\line(1,1){45}}
	\put(-50,-45){\color{darkgreen}\line(1,0){70}}
	\put(20,-45){\line(1,1){45}}
	\put(20,-45){\line(1,-1){45}}

\put(160,-45){\begin{tabular}{c|ccc}
 & A & B & C\\
 \hline
 B & {\color{red}$d_{AB}$} & & \\
 C & {\color{darkgreen}$d_{AC}$} & $d_{BC}$ & \\
 D & $d_{AD}$ & {\color{darkgreen}$d_{BD}$} & {\color{red}$d_{CD}$} \\
\end{tabular}
}
\end{picture}\\
\end{center}
\end{figure}
\vskip 2cm
\begin{center}
\Large
{$i^{\dag} = \frac{{\color{darkgreen}d_{AC} + d_{BD}} {\color{red}- d_{AB}-d_{CD}}}{2}$}
\par
\end{center}
\normalsize


\myNewSlide
\Large
Note that our estimate \\
\[i^{\dag} = \frac{{\color{darkgreen}d_{AC} + d_{BD}} {\color{red}- d_{AB}-d_{CD}}}{2}\] 
 does not use all of our data.  $d_{BC}$ and $d_{AD}$ are ignored!

We could have used $d_{BC} + d_{AD}$ instead of $d_{AC} + d_{BD}$ (you can see this by going through
the previous slides after rotating the internal branch).

\[{i^{\ast} = \frac{d_{BC} + d_{AD} {\color{red}- d_{AB}-d_{CD}}}{2}}\] 

\myNewSlide
A better estimate than either $i$ or $i^{\ast}$ would be the average of both of them:
\[i^{\prime} = \frac{d_{BC} + d_{AD}  + {\color{darkgreen}d_{AC} + d_{BD}}}{4} {\color{red} - \frac{d_{AB}-d_{CD}}{2}}\] 



\myNewSlide
\begin{figure}
\normalsize
\begin{center}
\setlength{\unitlength}{.04cm}
\begin{picture}(200,250)(70,-250)
	\thicklines
	{\color{red}
	\put(-105,3){A}
	\put(-110,-99){B}
	\put(70,3){C}
	\put(70,-99){D}
	\put(-75,-17){$\nu_a$}
	\put(-65,-72){$\nu_b$}
	\put(40,-17){$\nu_c$}
	\put(28,-74){$\nu_d$}
	\put(-15,-40){$\nu_i$}
	\put(-95,0){\line(1,-1){45}}
	\put(-95,-90){\line(1,1){45}}
	\put(-50,-45){\line(1,0){70}}
	\put(20,-45){\line(1,1){45}}
	\put(20,-45){\line(1,-1){45}}
}
	\put(105,3){A}
	\put(105,-99){C}
	\put(280,3){B}
	\put(280,-99){D}
	\put(135,-17){$\nu_a$}
	\put(130,-72){$\nu_c$}
	\put(250,-17){$\nu_b$}
	\put(243,-74){$\nu_d$}
	\put(195,-40){$\nu_i$}
	\put(120,0){\line(1,-1){45}}
	\put(120,-90){\line(1,1){45}}
	\put(165,-45){\line(1,0){70}}
	\put(235,-45){\line(1,1){45}}
	\put(235,-45){\line(1,-1){45}}

	{\color{green}
	\put(325,3){A}
	\put(320,-99){D}
	\put(500,3){C}
	\put(500,-99){B}
	\put(365,-17){$\nu_a$}
	\put(365,-72){$\nu_d$}
	\put(470,-17){$\nu_c$}
	\put(460,-74){$\nu_b$}
	\put(415,-40){$\nu_i$}
	\put(335,0){\line(1,-1){45}}
	\put(335,-90){\line(1,1){45}}
	\put(380,-45){\line(1,0){70}}
	\put(450,-45){\line(1,1){45}}
	\put(450,-45){\line(1,-1){45}}
	}
	\put(-180,-195){\begin{tabular}{p{3.75cm}|p{5.95cm}p{0.2cm}p{6cm}p{0.2cm}p{8cm}|}
\hline
{\color{black}$d_{AB} + d_{CD}$} & {\color{red}$\nu_a+\nu_b+\nu_c+\nu_d$} & & {\color{black}$\nu_a+\nu_b+\nu_c+\nu_d+2\nu_i$} & & {\color{green}$\nu_a+\nu_b+\nu_c+\nu_d+2\nu_i$} \\
\hline
{\color{black}$d_{AC} + d_{BD}$}& {\color{red}$\nu_a+\nu_b+\nu_c+\nu_d+2\nu_i$} & & {\color{black}$\nu_a+\nu_b+\nu_c+\nu_d$} & & {\color{green}$\nu_a+\nu_b+\nu_c+\nu_d+2\nu_i$} \\
\hline
{\color{black}$d_{AD} + d_{BC}$} & {\color{red}$\nu_a+\nu_b+\nu_c+\nu_d+2\nu_i$} & & {\color{black}$\nu_a+\nu_b+\nu_c+\nu_d+2\nu_i$} & & {\color{green}$\nu_a+\nu_b+\nu_c+\nu_d$} \\
\hline
\end{tabular}
}
\end{picture}
\end{center}
\end{figure}

The four point condition of \citet{Buneman1971}.

This assumes additivity of distances.

\myNewSlide
\normalsize
\begin{figure}
\begin{center}
\setlength{\unitlength}{.04cm}
\begin{picture}(200,250)(70,-250)
	\thicklines
	\put(105,3){A}
	\put(105,-99){C}
	\put(280,3){B}
	\put(280,-99){D}
	\put(135,-17){$\nu_a$}
	\put(130,-72){$\nu_c$}
	\put(250,-17){$\nu_b$}
	\put(243,-74){$\nu_d$}
	\put(195,-40){$\nu_i$}
	\put(120,0){\line(1,-1){45}}
	\put(120,-90){\line(1,1){45}}
	\put(165,-45){\line(1,0){70}}
	\put(235,-45){\line(1,1){45}}
	\put(235,-45){\line(1,-1){45}}

	\put(-180,-175){\begin{tabular}{p{4cm}|p{5cm}p{0.3cm}p{12cm}p{0.8cm}p{8cm}|}
\hline
{\color{black}$d_{AB} + d_{CD}$} &  & & {\color{black}$\nu_a+\nu_b+\nu_c+\nu_d+2\nu_i + \epsilon_{AB} + \epsilon_{CD}$} & & \\
\hline
{\color{black}$d_{AC} + d_{BD}$}& & & {\color{black}$\nu_a+\nu_b+\nu_c+\nu_d+ \epsilon_{AC}+ \epsilon_{BD}$} & & \\
\hline
{\color{black}$d_{AD} + d_{BC}$} & & & {\color{black}$\nu_a+\nu_b+\nu_c+\nu_d+2\nu_i+ \epsilon_{AD}+ \epsilon_{BC}$} & & \\
\hline
\end{tabular}
}
\end{picture}
\end{center}
\end{figure}
\large
If $|\epsilon_{ij}| < \frac{\nu_i}{2}$ then $d_{AC} + d_{BD}$ will still be the smallest sum -- So Buneman's method will get the tree correct.

Worst case: $\epsilon_{AC} = \epsilon_{BD} = \frac{\nu_i}{2}$ and $\epsilon_{AB} = \epsilon_{CD} = -\frac{\nu_i}{2}$ then 

\[d_{AC} + d_{BD} = \nu_a+\nu_b+\nu_c+\nu_d+\nu_i = d_{AB} + d_{CD}\]

\myNewSlide




\large
Both Buneman's four-point condition and  Hennigian logic, return the tree given perfectly clean data. But what does ``perfectly clean data'' mean?
\begin{compactenum}
	\item Hennigian analysis $\rightarrow$ no homoplasy. The infinite alleles model.
	\item Buneman's four-point test $\rightarrow$ no multiple hits to the same site. The infinite sites model.
\end{compactenum}


\myNewSlide
\section*{The guiding principle of distance-based methods}
If our data are true measures of evolutionary distances (and the distance along each branch is always $>0$)
then:
\begin{compactenum}
	\item The distances will be additive on the true tree.
	\item The distances will {\bf not} be additive on any other tree.
\end{compactenum}
This is the basis of Buneman's method and the motivation for minimizing the sum-of-squared error (least squares) too choose among trees.



\myNewSlide
\section*{Balanced minimum evolution}
\large
The logic behind Buneman's four-point condition has been extend to trees of more than 4 taxa by \citet{Pauplin2000} and \citet{SempleS2004}.

\cite{Pauplin2000} showed that you can calculate a tree length from the pairwise distances without calculating branch lengths.
The key is weighting the distances:
\[l = \sum_i^N\sum_{j=i+1}^N w_{ij} d_{ij}\] 
where:
\[w_{ij} = \frac{1}{2^{n(i,j)}}\]
and $n(i,j)$ is the number of nodes on the path from $i$ to $j$.

\section*{Balanced minimum evolution}

``Balanced Minimum Evolution'' \citet{DesperG2002,DesperG2004} -- fitting the branch lengths using the estimators of \citet{Pauplin2000} and preferring the tree with the smallest tree length 

BME = a form of weighted least squares in which distances are down-weighted by an exponential function of the topological distances between the leaves.

\citet{DesperG2005}: neighbor-joining is star decomposition (more on this later) under BME. See \citet{GascuelS2006}

\myNewSlide
\section*{FastME}
Software by \citet{DesperG2004} which implements searching under the balanced minimum evolution criterion. 

It is extremely fast and is more accurate than neighbor-joining (based on simulation studies).

\myNewSlide
\section*{Distance methods: pros}
\begin{itemize}
	\item Fast  -- the new FastTree method \citet{PriceDA2009} can calculate a tree in less time than it takes to calculate a full distance matrix!
	\item Can use models to correct for unobserved differences
	\item Works well for closely related sequences
	\item Works well for clock-like sequences
\end{itemize}

\myNewSlide
\section*{Distance methods: cons}
\begin{itemize}
	\item Do not use all of the information in sequences
	\item Do not reconstruct character histories, so they not enforce all logical constraints
\end{itemize}

\begin{picture}(0,0)(0,0)  
\put(100,-50){\makebox(0,0)[l]{\includegraphics[scale=1.2]{../images/simple_unrooted.pdf}}}
\put(90,-30){A}
\put(90,-80){G}
\put(165,-30){A}
\put(165,-80){G}
\end{picture}




\myNewSlide

\myNewSlide
\section*{Neighbor-joining}
\citet{SaitouN1987}.  $r$ is the number of leaves remaining.
Start with $r=N$.
\begin{compactitem}
	\item[1.] choose the pair of leaves $x$ and $y$ that minimize $Q(x,y)$:
	\[Q(i,j) = (r-2)d_{ij} -\sum_{k=1}^r d_{ik}-\sum_{k=1}^r d_{jk}\]
	\item[2.] Join $x$ and $y$ with at a new node $z$.  Take $x$ and $y$ out of the leaf set and distance matrix, and add the new node $z$ as a leaf.
\end{compactitem}

\myNewSlide
\section*{Neighbor-joining (continued)}
\begin{compactitem}
	\item[3.] Set the branch length from $x$ to $z$ using:
	\[d_{xz} = \frac{d_{xy}}{2} + \left(\frac{1}{2(r-2)}\right)\left(\sum_{k=1}^r d_{xk}-\sum_{k=1}^r d_{yk}\right)\]
	(the length of the branch from $y$ to $z$ is set with a similar formula).
	\item[4.] Update the distance matrix, by adding (for any other taxon $k$) the distance:
	\[d_{zk} = \frac{d_{xk} + d_{yk} - d_{xz} - d_{yz}}{2}\]
	\item[5.] return to step 1 until you are down to a trivial tree.
\end{compactitem}




\myNewSlide
\section*{Neighbor-joining (example)}
\begin{table}[htdp]
\begin{center}
\begin{tabular}{|c|c|c|c|c|c|c|}
\hline
 & A & B & C & D & E & F  \\ 
\hline
A &      - &         &         &       & & \\ 
B & 0.258 &        - &         &       & & \\ 
C  & 0.274 &  0.204   &      - &       & & \\ 
D  & 0.302  & 0.248 & 0.278  &       -  & &\\
E  &  0.288  &  0.224 &  0.252 &  0.268 &        -  & \\
F  &  0.250  &  0.160 &  0.226 &  0.210 &  0.194 &        - \\
\hline
\end{tabular}
\end{center}
\end{table}

\myNewSlide
\section*{Neighbor-joining (example)}
\begin{table}[htdp]
\begin{center}
\begin{tabular}{|r|c|c|c|c|c|c|c|}
\hline
$\sum_{k} d_{ik}$ &  & A & B & C & D & E & F  \\ 
\hline
1.372 & A & 0.0 & 0.258 & 0.274 & 0.302 & 0.288 & 0.25 \\
1.094 & B & 0.258 & 0.0 & 0.204 & 0.248 & 0.224 & 0.16 \\
1.234 & C & 0.274 & 0.204 & 0.0 & 0.278 & 0.252 & 0.226 \\
1.306 & D & 0.302 & 0.248 & 0.278 & 0.0 & 0.268 & 0.21 \\
1.226 & E & 0.288 & 0.224 & 0.252 & 0.268 & 0.0 & 0.194 \\
1.040 & F & 0.25 & 0.16 & 0.226 & 0.21 & 0.194 & 0.0 \\
\hline
\end{tabular}
\end{center}
\end{table}%


\myNewSlide
\begin{table}[htdp]
\begin{center}
\begin{tabular}{|r|c|}
\hline
$Q(A,B)$ & -1.434 \\
$Q(A,C)$ & -1.510 \\
$Q(A,D)$ & -1.470 \\
$Q(A,E)$ & -1.446 \\
$Q(A,F)$ & -1.412 \\
{\color{red} $Q(B,C)$ } &{\color{red} -1.512 }\\
$Q(B,D)$ & -1.408 \\
$Q(B,E)$ & -1.424 \\
$Q(B,F)$ & -1.494 \\
$Q(C,D)$ & -1.428 \\
$Q(C,E)$ & -1.452 \\
$Q(C,F)$ & -1.370 \\
$Q(D,E)$ & -1.460 \\
$Q(D,F)$ & -1.506 \\
$Q(E,F)$ & -1.490 \\
\hline
\end{tabular}
\end{center}
\end{table}%




\myNewSlide
\section*{Neighbor-joining (example)}
\begin{table}[htdp]
\begin{center}
\begin{tabular}{|c|c|c|c|c|c|}
\hline
 & A  & D & E & F & (B,C)  \\ 
\hline
A & 0.0 & 0.302 & 0.288 & 0.25 & 0.164 \\
D & 0.302 & 0.0 & 0.268 & 0.21 & 0.161 \\
E & 0.288 & 0.268 & 0.0 & 0.194 & 0.136 \\
F & 0.25 & 0.21 & 0.194 & 0.0 & 0.091 \\
(B,C) & 0.164 & 0.161 & 0.136 & 0.091 & 0.0\\
\hline
\end{tabular}
\end{center}
\end{table}%



\myNewSlide
\section*{Neighbor-joining (example)}
\begin{table}[htdp]
\begin{center}
\begin{tabular}{|c|c|c|c|c|c|c|}
\hline
$\sum_{k} d_{ik}$ &  & A & D & E & F & (B,C)  \\ 
\hline
1.004000 & A & 0.0 & 0.302 & 0.288 & 0.25 & 0.164 \\
0.941000 & D & 0.302 & 0.0 & 0.268 & 0.21 & 0.161 \\
0.886000 & E & 0.288 & 0.268 & 0.0 & 0.194 & 0.136 \\
0.745000 & F & 0.25 & 0.21 & 0.194 & 0.0 & 0.091 \\
0.552000 & (B,C) & 0.164 & 0.161 & 0.136 & 0.091 & 0.0 \\
\hline
\end{tabular}
\end{center}
\end{table}%

\myNewSlide
\section*{Neighbor-joining (example)}
\begin{table}[htdp]
\begin{center}
\begin{tabular}{|r|c|}
\hline
$Q(A,D)$ & -1.039000 \\
$Q(A,E)$ & -1.026000 \\
$Q(A,F)$ & -0.999000 \\
{\color{red} $Q(A,(B,C))$} & {\color{red} -1.064000} \\
$Q(D,E)$ & -1.023000 \\
$Q(D,F)$ & -1.056000 \\
$Q(D,(B,C))$ & -1.010000 \\
$Q(E,F)$ & -1.049000 \\
$Q(E,(B,C))$ & -1.030000 \\
$Q(F,(B,C))$ & -1.024000 \\
\hline
\end{tabular}
\end{center}
\end{table}%



\myNewSlide
\section*{Neighbor-joining (example)}
\begin{table}[htdp]
\begin{center}
\begin{tabular}{|c|c|c|c|c|}
\hline
   & D & E & F & (A,(B,C))  \\ 
\hline
D & 0.0 & 0.268 & 0.21 & 0.1495 \\
E & 0.268 & 0.0 & 0.194 & 0.13 \\
F & 0.21 & 0.194 & 0.0 & 0.0885 \\
(A,(B,C)) & 0.1495 & 0.13 & 0.0885 & 0.0 \\
\hline
\end{tabular}
\end{center}
\end{table}%


\myNewSlide
\section*{Neighbor-joining (example)}
\begin{table}[htdp]
\begin{center}
\begin{tabular}{|c|c|c|c|c|c|}
\hline
$\sum_{k} d_{ik}$ &  & D & E & F & (A,(B,C))  \\
\hline
0.627500 & D & 0.0 & 0.268 & 0.21 & 0.1495 \\
0.592000 & E & 0.268 & 0.0 & 0.194 & 0.13 \\
0.492500 & F & 0.21 & 0.194 & 0.0 & 0.0885 \\
0.368000 & (A,(B,C)) & 0.1495 & 0.13 & 0.0885 & 0.0 \\
\hline
\end{tabular}
\end{center}
\end{table}%


\myNewSlide
\section*{Neighbor-joining (example)}
\begin{table}[htdp]
\begin{center}
\begin{tabular}{|r|c|}
\hline
$Q(D,E)$ & -0.683500 \\
{\color{red}$Q(D,F)$} & {\color{red}-0.700000} \\
$Q(D,(A,(B,C)))$ & -0.696500 \\
$Q(E,F)$ & -0.696500 \\
$Q(E,(A,(B,C)))$ & -0.700000 \\
$Q(F,(A,(B,C)))$ & -0.683500 \\
\hline
\end{tabular}
\end{center}
\end{table}%
\begin{center}
{\color{red}$((D,F),E,(A,(B,C)))$}
\end{center}


\myNewSlide
\section*{Neighbor-joining is special}
\citet{Bryant2005} discusses neighbor joining in the context of clustering methods that:
\begin{compactitem}
	\item Work on the distance (or dissimilarity) matrix as input.
	\item Repeatedly
	\begin{compactitem}
		\item select a pair of taxa to agglomerate (step 1 above)
		\item make the pair into a new group (step 2 above)
		\item estimate branch lengths (step 3 above)
		\item reduce the distance matrix (step 4 above)
	\end{compactitem}
\end{compactitem}

\myNewSlide
\section*{Neighbor-joining is special (cont)}
\citet{Bryant2005} shows that if you want your selection criterion to be:
\begin{compactitem}
	\item based solely on distances
	\item invariant to the ordering of the leaves (no {\em a priori} special taxa).
	\item work on linear combinations of distances (simple coefficients for weights, no fancy weighting schemes).
	\item statistically consistent
\end{compactitem}
then neighbor-joining's $Q$-criterion as a selection rule is the {\em only} choice.
	
\myNewSlide
\section*{Neighbor-joining is not perfect}
\begin{compactitem}
	\item BioNJ \citep{Gascuel1997} does a better job by using the variance and covariances in the reduction step.
	\item Weighbor \citep{BrunoSH2000} includes the variance information in the selection step.
	\item FastME \citep{DesperG2002,DesperG2004} does a better job of finding the BME tree (and seems to get the true tree right more often).
\end{compactitem}


\myNewSlide
\bibliography{phylo}
\end{document}     




\myNewSlide

\begin{figure}
\begin{center}
\setlength{\unitlength}{.06cm}
\begin{picture}(200,100)(0,-100)
	\thicklines
	\put(-105,3){A}
	\put(-105,-99){B}
	\put(70,3){C}
	\put(70,-99){D}
	\put(-75,-17){$a$}
	\put(-70,-72){$b$}
	\put(40,-17){$c$}
	\put(33,-74){$d$}
	\put(-15,-40){$i$}
	\put(-95,0){\line(1,-1){45}}
	\put(-95,-90){\line(1,1){45}}
	\put(-50,-45){\line(1,0){70}}
	\put(20,-45){\line(1,1){45}}
	\put(20,-45){\line(1,-1){45}}
\put(160,-45){\begin{tabular}{c|ccc}
 & A & B & C\\
 \hline
 B & $d_{AB}$ & & \\
 C & $d_{AC}$ & $d_{BC}$ & \\
 D & $d_{AD}$ & $d_{BD}$ & $d_{CD}$ \\
\end{tabular}
}
\end{picture}\\
\end{center}
\end{figure}

If the tree above is correct then:
\begin{eqnarray*}
	p_{AB} & = & a + b \\
	p_{AC} & = & a + i + c \\
	p_{AD} & = & a + i + d \\
	p_{BC} & = & b + i + c \\
	p_{BD} & = & b + i + d \\
	p_{CD} & = & c + d \\
\end{eqnarray*}

\myNewSlide
If our distance data reflects true ``evolutionary distance'' (amount of character change within a lineage),
then distances will be additive:
\begin{eqnarray*}
	d_{AB} = p_{AB} & = & a + b \\
	d_{AC} = p_{AC} & = & a + i + c \\
	d_{AD} = p_{AD} & = & a + i + d \\
	d_{BC} = p_{BC} & = & b + i + c \\
	d_{BD} = p_{BD} & = & b + i + d \\
	d_{CD} = p_{CD} & = & c + d \\
\end{eqnarray*}
and we simply have to find values for $a, b, c, d,$ and $i$  that match the data.


\myNewSlide
\begin{figure}
\begin{center}
\setlength{\unitlength}{.06cm}
\begin{picture}(200,100)(0,-100)
	\thicklines
	\put(-105,3){A}
	\put(-105,-99){B}
	\put(70,3){C}
	\put(70,-99){D}
	\put(-75,-17){$a$}
	\put(-70,-72){$b$}
	\put(40,-17){$c$}
	\put(33,-74){$d$}
	\put(-15,-40){$i$}
	\put(-95,0){\line(1,-1){45}}
	\put(-95,-90){\line(1,1){45}}
	\put(-50,-45){\line(1,0){70}}
	\put(20,-45){\line(1,1){45}}
	\put(20,-45){\line(1,-1){45}}

	\put(-95,15){\color{green}\line(1,-1){45}}
	\put(20,-30){\color{green}\line(1,1){45}}
	\put(-50,-30){\color{green}\line(1,0){70}}

\put(160,-45){\begin{tabular}{c|ccc}
 & A & B & C\\
 \hline
 B & $d_{AB}$ & & \\
 C & $d_{AC}$ & $d_{BC}$ & \\
 D & $d_{AD}$ & $d_{BD}$ & $d_{CD}$ \\
\end{tabular}
}
\end{picture}\\
\end{center}
\end{figure}
\vskip 2cm
\begin{center}

{\color{green}$d_{AC}$}

\end{center}


\myNewSlide
\begin{figure}
\begin{center}
\setlength{\unitlength}{.06cm}
\begin{picture}(200,100)(0,-100)
	\thicklines
	\put(-105,3){A}
	\put(-105,-99){B}
	\put(70,3){C}
	\put(70,-99){D}
	\put(-75,-17){$a$}
	\put(-70,-72){$b$}
	\put(40,-17){$c$}
	\put(33,-74){$d$}
	\put(-15,-40){$i$}
	\put(-95,0){\line(1,-1){45}}
	\put(-95,-90){\line(1,1){45}}
	\put(-50,-45){\line(1,0){70}}
	\put(20,-45){\line(1,1){45}}
	\put(20,-45){\line(1,-1){45}}

	\put(-95,15){\color{green}\line(1,-1){45}}
	\put(20,-30){\color{green}\line(1,1){45}}
	\put(-50,-30){\color{green}\line(1,0){70}}

	\put(-95,-105){\color{green}\line(1,1){45}}
	\put(20,-60){\color{green}\line(1,-1){45}}
	\put(-50,-60){\color{green}\line(1,0){70}}

\put(160,-45){\begin{tabular}{c|ccc}
 & A & B & C\\
 \hline
 B & $d_{AB}$ & & \\
 C & $d_{AC}$ & $d_{BC}$ & \\
 D & $d_{AD}$ & $d_{BD}$ & $d_{CD}$ \\
\end{tabular}
}
\end{picture}\\
\end{center}
\end{figure}
\vskip 2cm
\begin{center}

{\color{green}$d_{AC} + d_{BD}$}

\end{center}

\myNewSlide
\begin{figure}
\begin{center}
\setlength{\unitlength}{.06cm}
\begin{picture}(200,100)(0,-100)
	\thicklines
	\put(-105,3){A}
	\put(-105,-99){B}
	\put(70,3){C}
	\put(70,-99){D}
	\put(-75,-17){$a$}
	\put(-70,-72){$b$}
	\put(40,-17){$c$}
	\put(33,-74){$d$}
	\put(-15,-40){$i$}
	\put(-95,0){\line(1,-1){45}}
	\put(-95,-90){\line(1,1){45}}
	\put(-50,-45){\line(1,0){70}}
	\put(20,-45){\line(1,1){45}}
	\put(20,-45){\line(1,-1){45}}

	\put(-95,15){\color{green}\line(1,-1){45}}
	\put(20,-30){\color{green}\line(1,1){45}}
	\put(-50,-30){\color{green}\line(1,0){70}}

	\put(-95,-105){\color{green}\line(1,1){45}}
	\put(20,-60){\color{green}\line(1,-1){45}}
	\put(-50,-60){\color{green}\line(1,0){70}}

	\put(-110,-90){\color{red}\line(1,1){45}}
	\put(-110,0){\color{red}\line(1,-1){45}}

\put(160,-45){\begin{tabular}{c|ccc}
 & A & B & C\\
 \hline
 B & $d_{AB}$ & & \\
 C & $d_{AC}$ & $d_{BC}$ & \\
 D & $d_{AD}$ & $d_{BD}$ & $d_{CD}$ \\
\end{tabular}
}
\end{picture}\\
\end{center}
\end{figure}
\vskip 2cm
\begin{center}

{\color{green}$d_{AC} + d_{BD}$}
\par
{\color{red}$d_{AB}$}
\end{center}



\myNewSlide
\begin{figure}
\begin{center}
\setlength{\unitlength}{.06cm}
\begin{picture}(200,100)(0,-100)
	\thicklines
	\put(-105,3){A}
	\put(-105,-99){B}
	\put(70,3){C}
	\put(70,-99){D}
	\put(-75,-17){$a$}
	\put(-70,-72){$b$}
	\put(40,-17){$c$}
	\put(33,-74){$d$}
	\put(-15,-40){$i$}
	\put(-95,0){\line(1,-1){45}}
	\put(-95,-90){\line(1,1){45}}
	\put(-50,-45){\line(1,0){70}}
	\put(20,-45){\line(1,1){45}}
	\put(20,-45){\line(1,-1){45}}

	%\put(-95,15){\color{green}\line(1,-1){45}}
	\put(20,-30){\color{green}\line(1,1){45}}
	\put(-50,-30){\color{green}\line(1,0){70}}

	%\put(-95,-105){\color{green}\line(1,1){45}}
	\put(20,-60){\color{green}\line(1,-1){45}}
	\put(-50,-60){\color{green}\line(1,0){70}}

	%\put(-110,-90){\color{red}\line(1,1){45}}
	%\put(-110,0){\color{red}\line(1,-1){45}}

\put(160,-45){\begin{tabular}{c|ccc}
 & A & B & C\\
 \hline
 B & $d_{AB}$ & & \\
 C & $d_{AC}$ & $d_{BC}$ & \\
 D & $d_{AD}$ & $d_{BD}$ & $d_{CD}$ \\
\end{tabular}
}
\end{picture}\\
\end{center}
\end{figure}
\vskip 2cm
\begin{center}

{\color{green}$d_{AC} + d_{BD} - d_{AB}$}
\par
\end{center}


\myNewSlide
\begin{figure}
\begin{center}
\setlength{\unitlength}{.06cm}
\begin{picture}(200,100)(0,-100)
	\thicklines
	\put(-105,3){A}
	\put(-105,-99){B}
	\put(70,3){C}
	\put(70,-99){D}
	\put(-75,-17){$a$}
	\put(-70,-72){$b$}
	\put(40,-17){$c$}
	\put(33,-74){$d$}
	\put(-15,-40){$i$}
	\put(-95,0){\line(1,-1){45}}
	\put(-95,-90){\line(1,1){45}}
	\put(-50,-45){\line(1,0){70}}
	\put(20,-45){\line(1,1){45}}
	\put(20,-45){\line(1,-1){45}}

	%\put(-95,15){\color{green}\line(1,-1){45}}
	\put(20,-30){\color{green}\line(1,1){45}}
	\put(-50,-30){\color{green}\line(1,0){70}}

	%\put(-95,-105){\color{green}\line(1,1){45}}
	\put(20,-60){\color{green}\line(1,-1){45}}
	\put(-50,-60){\color{green}\line(1,0){70}}

	\put(35,-45){\color{red}\line(1,1){45}}
	\put(35,-45){\color{red}\line(1,-1){45}}

\put(160,-45){\begin{tabular}{c|ccc}
 & A & B & C\\
 \hline
 B & $d_{AB}$ & & \\
 C & $d_{AC}$ & $d_{BC}$ & \\
 D & $d_{AD}$ & $d_{BD}$ & $d_{CD}$ \\
\end{tabular}
}
\end{picture}\\
\end{center}
\end{figure}
\vskip 2cm
\begin{center}

{\color{green}$d_{AC} + d_{BD} - d_{AB}$}
\par
{\color{red}$d_{CD}$}
\end{center}


\myNewSlide
\begin{figure}
\begin{center}
\setlength{\unitlength}{.06cm}
\begin{picture}(200,100)(0,-100)
	\thicklines
	\put(-105,3){A}
	\put(-105,-99){B}
	\put(70,3){C}
	\put(70,-99){D}
	\put(-75,-17){$a$}
	\put(-70,-72){$b$}
	\put(40,-17){$c$}
	\put(33,-74){$d$}
	\put(-15,-40){$i$}
	\put(-95,0){\line(1,-1){45}}
	\put(-95,-90){\line(1,1){45}}
	\put(-50,-45){\line(1,0){70}}
	\put(20,-45){\line(1,1){45}}
	\put(20,-45){\line(1,-1){45}}

	%\put(-95,15){\color{green}\line(1,-1){45}}
	%\put(20,-30){\color{green}\line(1,1){45}}
	\put(-50,-30){\color{green}\line(1,0){70}}

	%\put(-95,-105){\color{green}\line(1,1){45}}
	%\put(20,-60){\color{green}\line(1,-1){45}}
	\put(-50,-60){\color{green}\line(1,0){70}}

	%\put(35,-45){\color{red}\line(1,1){45}}
	%\put(35,-45){\color{red}\line(1,-1){45}}

\put(160,-45){\begin{tabular}{c|ccc}
 & A & B & C\\
 \hline
 B & $d_{AB}$ & & \\
 C & $d_{AC}$ & $d_{BC}$ & \\
 D & $d_{AD}$ & $d_{BD}$ & $d_{CD}$ \\
\end{tabular}
}
\end{picture}\\
\end{center}
\end{figure}
\vskip 2cm
\begin{center}

{\color{green}$d_{AC} + d_{BD} - d_{AB}-d_{CD}$}
\par
\end{center}




\myNewSlide
\begin{figure}
\begin{center}
\setlength{\unitlength}{.06cm}
\begin{picture}(200,100)(0,-100)
	\thicklines
	\put(-105,3){A}
	\put(-105,-99){B}
	\put(70,3){C}
	\put(70,-99){D}
	\put(-75,-17){$a$}
	\put(-70,-72){$b$}
	\put(40,-17){$c$}
	\put(33,-74){$d$}
	\put(-15,-40){$i$}
	\put(-95,0){\line(1,-1){45}}
	\put(-95,-90){\line(1,1){45}}
	\put(-50,-45){\color{green}\line(1,0){70}}
	\put(20,-45){\line(1,1){45}}
	\put(20,-45){\line(1,-1){45}}

	%\put(-95,15){\color{green}\line(1,-1){45}}
	%\put(20,-30){\color{green}\line(1,1){45}}
	%\put(-50,-30){\color{green}\line(1,0){70}}

	%\put(-95,-105){\color{green}\line(1,1){45}}
	%\put(20,-60){\color{green}\line(1,-1){45}}
	%\put(-50,-60){\color{green}\line(1,0){70}}

	%\put(35,-45){\color{red}\line(1,1){45}}
	%\put(35,-45){\color{red}\line(1,-1){45}}

\put(160,-45){\begin{tabular}{c|ccc}
 & A & B & C\\
 \hline
 B & $d_{AB}$ & & \\
 C & $d_{AC}$ & $d_{BC}$ & \\
 D & $d_{AD}$ & $d_{BD}$ & $d_{CD}$ \\
\end{tabular}
}
\end{picture}\\
\end{center}
\end{figure}
\vskip 2cm
\begin{center}
\Large
{\color{green}$i = \frac{d_{AC} + d_{BD} - d_{AB}-d_{CD}}{2}$}
\par
\end{center}
\normalsize


\myNewSlide
\begin{figure}
\begin{center}
\setlength{\unitlength}{.04cm}
\begin{picture}(200,250)(70,-250)
	\thicklines
	{\color{red}
	\put(-105,3){A}
	\put(-110,-99){B}
	\put(70,3){C}
	\put(70,-99){D}
	\put(-75,-17){$\nu_a$}
	\put(-65,-72){$\nu_b$}
	\put(40,-17){$\nu_c$}
	\put(28,-74){$\nu_d$}
	\put(-15,-40){$\nu_i$}
	\put(-95,0){\line(1,-1){45}}
	\put(-95,-90){\line(1,1){45}}
	\put(-50,-45){\line(1,0){70}}
	\put(20,-45){\line(1,1){45}}
	\put(20,-45){\line(1,-1){45}}
}
	\put(105,3){A}
	\put(105,-99){C}
	\put(280,3){B}
	\put(280,-99){D}
	\put(135,-17){$\nu_a$}
	\put(130,-72){$\nu_c$}
	\put(250,-17){$\nu_b$}
	\put(243,-74){$\nu_d$}
	\put(195,-40){$\nu_i$}
	\put(120,0){\line(1,-1){45}}
	\put(120,-90){\line(1,1){45}}
	\put(165,-45){\line(1,0){70}}
	\put(235,-45){\line(1,1){45}}
	\put(235,-45){\line(1,-1){45}}

	{\color{green}
	\put(325,3){A}
	\put(320,-99){D}
	\put(500,3){C}
	\put(500,-99){B}
	\put(365,-17){$\nu_a$}
	\put(365,-72){$\nu_d$}
	\put(470,-17){$\nu_c$}
	\put(460,-74){$\nu_b$}
	\put(415,-40){$\nu_i$}
	\put(335,0){\line(1,-1){45}}
	\put(335,-90){\line(1,1){45}}
	\put(380,-45){\line(1,0){70}}
	\put(450,-45){\line(1,1){45}}
	\put(450,-45){\line(1,-1){45}}
	}
	\put(-180,-195){\begin{tabular}{p{3.75cm}|p{5.95cm}p{0.2cm}p{6cm}p{0.2cm}p{8cm}|}
\hline
{\color{black}$d_{AB} + d_{CD}$} & {\color{red}$\nu_a+\nu_b+\nu_c+\nu_d$} & & {\color{black}$\nu_a+\nu_b+\nu_c+\nu_d+2\nu_i$} & & {\color{green}$\nu_a+\nu_b+\nu_c+\nu_d+2\nu_i$} \\
\hline
{\color{black}$d_{AC} + d_{BD}$}& {\color{red}$\nu_a+\nu_b+\nu_c+\nu_d+2\nu_i$} & & {\color{black}$\nu_a+\nu_b+\nu_c+\nu_d$} & & {\color{green}$\nu_a+\nu_b+\nu_c+\nu_d+2\nu_i$} \\
\hline
{\color{black}$d_{AD} + d_{BC}$} & {\color{red}$\nu_a+\nu_b+\nu_c+\nu_d+2\nu_i$} & & {\color{black}$\nu_a+\nu_b+\nu_c+\nu_d+2\nu_i$} & & {\color{green}$\nu_a+\nu_b+\nu_c+\nu_d$} \\
\hline
\end{tabular}
}
\end{picture}
\end{center}
\end{figure}

The four point condition of \citet{Buneman1971}.

This assumes additivity of distances.

\myNewSlide
\begin{figure}
\begin{center}
\setlength{\unitlength}{.04cm}
\begin{picture}(200,250)(70,-250)
	\thicklines
	\put(105,3){A}
	\put(105,-99){C}
	\put(280,3){B}
	\put(280,-99){D}
	\put(135,-17){$\nu_a$}
	\put(130,-72){$\nu_c$}
	\put(250,-17){$\nu_b$}
	\put(243,-74){$\nu_d$}
	\put(195,-40){$\nu_i$}
	\put(120,0){\line(1,-1){45}}
	\put(120,-90){\line(1,1){45}}
	\put(165,-45){\line(1,0){70}}
	\put(235,-45){\line(1,1){45}}
	\put(235,-45){\line(1,-1){45}}

	\put(-180,-175){\begin{tabular}{p{4cm}|p{5cm}p{0.3cm}p{12cm}p{0.8cm}p{8cm}|}
\hline
{\color{black}$d_{AB} + d_{CD}$} &  & & {\color{black}$\nu_a+\nu_b+\nu_c+\nu_d+2\nu_i + \epsilon_{AB} + \epsilon_{CD}$} & & \\
\hline
{\color{black}$d_{AC} + d_{BD}$}& & & {\color{black}$\nu_a+\nu_b+\nu_c+\nu_d+ \epsilon_{AC}+ \epsilon_{BD}$} & & \\
\hline
{\color{black}$d_{AD} + d_{BC}$} & & & {\color{black}$\nu_a+\nu_b+\nu_c+\nu_d+2\nu_i+ \epsilon_{AD}+ \epsilon_{BC}$} & & \\
\hline
\end{tabular}
}
\end{picture}
\end{center}
\end{figure}
\large
If $|\epsilon_{ij}| < \frac{\nu_i}{2}$ then $d_{AC} + d_{BD}$ will still be the smallest sum -- So Buneman's method will get the tree correct.

Worst case: $\epsilon_{AC} = \epsilon_{BD} = \frac{\nu_i}{2}$ and $\epsilon_{AB} = \epsilon_{CD} = -\frac{\nu_i}{2}$ then 

\[d_{AC} + d_{BD} = \nu_a+\nu_b+\nu_c+\nu_d+\nu_i = d_{AB} + d_{CD}\]
\myNewSlide
\begin{picture}(0,0)(0,0)
	\put(-30,-70){\makebox(0,0)[l]{\includegraphics{../images/sim20BaseSeq.pdf}}}
\end{picture}

\myNewSlide
\includepdf[pages={10-15}]{../nonfreeimages/pol/pol-distances.pdf} 



