\documentclass[landscape]{foils} 
\input{../common-preamble-start}
\input{../preamble.tex}
\usepackage{url}
\usepackage{hyperref}
\hypersetup{backref,  linkcolor=blue, citecolor=red, colorlinks=true, hyperindex=true}

\usepackage{pdfpages}
\usepackage{bm}
\usepackage{ifsym}
\newcommand{\hilite}[1]{{\color{red} \bf #1}}
\newcommand{\disruption}{\theta}
\begin{document}
\pagecolor{white}
\unitlength=1mm
\begin{center}
{\Large Some of these slides have been borrowed from Dr.\ Paul Lewis, Dr.\ Joe Felsenstein. Thanks!}
\vskip 15mm
\large Paul has many great tools for teaching phylogenetics at his web site: \\
\url{http://hydrodictyon.eeb.uconn.edu/people/plewis}
\end{center}


\myNewSlide
\section*{Key Innovations}


\myNewSlide
\includepdf[pages={5}]{../nonfreeimages/pol/pol-keyinnov.pdf}

\myNewSlide
\begin{picture}(0,0)(0,0)
	\put(0,-100){\makebox(0,0)[l]{\includegraphics{../images/twosp.pdf}}}
	\put(60,-19){$\Pr(1,1) = 1.0$}
\end{picture}

\myNewSlide
\begin{picture}(0,0)(0,0)
	\put(0,-100){\makebox(0,0)[l]{\includegraphics{../images/threesp_symm.pdf}}}
	\put(80,-25){$\Pr(1,2) = 0.5$}
	\put(80,-79){$\Pr(2,1) = 0.5$}
\end{picture}

\myNewSlide
\begin{picture}(0,0)(0,0)
	\put(0,-100){\makebox(0,0)[l]{\includegraphics{../images/threesp.pdf}}}
	\put(80,-25){$\Pr(1,2) = 1$}
\end{picture}

\myNewSlide
\begin{picture}(0,0)(0,0)
	\put(0,-100){\makebox(0,0)[l]{\includegraphics{../images/foursp.pdf}}}
	\put(120,-25){$\Pr(1,3) = \frac{1}{3}$}
	\put(120,-79){$\Pr(1,3) = \frac{1}{3}$}
	\put(120,-139){$\Pr(2,2) = \frac{1}{3}$}
\end{picture}

\myNewSlide
\begin{picture}(0,0)(0,0)
	\put(0,-100){\makebox(0,0)[l]{\includegraphics{../images/foursp_poly.pdf}}}
	\put(120,-25){$\Pr(1,3) + \Pr(3,1) = \frac{2}{3}$}
	\put(120,-79){$\Pr(2,2) = \frac{1}{3}$}
\end{picture}

\myNewSlide
\begin{picture}(0,0)(0,0)
	\put(0,-100){\makebox(0,0)[l]{\includegraphics{../images/fivesp.pdf}}}
	\put(120,-25){$\Pr(1,4) + \Pr(4,1) = \frac{1}{2}$}
	\put(120,-89){$\Pr(2,3) + \Pr(3,2) = \frac{1}{2}$}
\end{picture}

\myNewSlide
\begin{picture}(0,0)(0,0)
	\put(0,-80){\makebox(0,0)[l]{\includegraphics{../images/sixsp.pdf}}}
	\put(120,-20){$\Pr(1,5) + \Pr(5,1) = \frac{2}{5}$}
	\put(120,-75){$\Pr(2,4) + \Pr(4,2) = \frac{2}{5}$}
	\put(120,-130){$\Pr(3,3) = \frac{1}{5}$}
\end{picture}

\myNewSlide
\begin{picture}(0,0)(0,0)
	\put(0,-120){\makebox(0,0)[l]{\includegraphics{../images/sevensp.pdf}}}
	\put(120,-20){$\Pr(1,6) + \Pr(6,1) = \frac{1}{3}$}
	\put(120,-75){$\Pr(2,5) + \Pr(5,2) = \frac{1}{3}$}
	\put(120,-130){$\Pr(3,4) + \Pr(4,3) = \frac{1}{3}$}
\end{picture}

\myNewSlide
\begin{picture}(0,0)(0,0)
	\put(20,-90){\makebox(0,0)[l]{\includegraphics[scale=0.8]{../images/eightsp.pdf}}}
	\put(120,-15){$\Pr(1,7) + \Pr(7,1) = \frac{2}{7}$}
	\put(120,-65){$\Pr(2,6) + \Pr(6,2) = \frac{2}{7}$}
	\put(120,-115){$\Pr(3,5) + \Pr(5,3) = \frac{2}{7}$}
	\put(120,-160){$\Pr(4,4) = \frac{1}{7}$}
\end{picture}

\myNewSlide
\begin{picture}(0,0)(0,0)
	\put(20,-90){\makebox(0,0)[l]{\includegraphics[scale=0.8]{../images/ninesp.pdf}}}
	\put(120,-15){$\Pr(1,8)  + \Pr(8,1) = \frac{1}{4}$}
	\put(120,-65){$\Pr(2,7)  + \Pr(7,2) = \frac{1}{4}$}
	\put(120,-120){$\Pr(3,6)  + \Pr(6,3) = \frac{1}{4}$}
	\put(120,-170){$\Pr(4,5) + \Pr(5,4) = \frac{1}{4}$}
\end{picture}

\myNewSlide
\section*{Key Innovations -- clade size comparison}
If you have an {\em a priori} reason to expect one state to lead to more species, then
you can conduct the test as a one-tailed test.

This (roughly) divides the probabilities by one half.

\[\Pr(x,y) = \frac{1}{x+y-1}\]

You have the pair of sister clades have a total of 46 species; 43 are in one clade and 
three are in another.
What is the probability of seeing this much imbalance in clade size even if the character
does not affect clade size?

\myNewSlide
\includepdf[pages={8-17}]{../nonfreeimages/pol/pol-keyinnov.pdf}

\myNewSlide
\section*{Potential weakness of Ree's approach}
\large
\begin{compactitem}
	\item When testing for character correlations, stochastic character mapping can be weaker than Pagel's method because the model used to infer the mapping assumes independence.
	\item In Ree's approach the stochastic character mapping is done using a model that assumes that state changes are independent of the probability of cladogenesis.
	\item A potentially more powerful approach is to use a model that allows speciation and extinction rates to vary depending on a character state.
	\item As is often the case: being less powerful may make Ree's approach more robust!
\end{compactitem}


\myNewSlide
\section*{BiSSE model of \citet{MaddisonMO2007}}
Calculate the probability of tree shape and character distribution:
\[\Pr(X,T,{\bm \nu}|\theta)\]
rather than:
\[\Pr(X,|T,{\bm \nu}, \theta)\]
which is done by assuming that the evolution of a character is independent of tree shape.

\myNewSlide
\section*{BiSSE model of \citet{MaddisonMO2007}}
\begin{compactitem}
	\item[] $\mu_0$ the extinction rate of a species that displays character state 0
	\item[] $\mu_1$ the extinction rate of a species that displays character state 1
	\item[] $\lambda_0$ the speciation rate of a species that displays character state 0
	\item[] $\lambda_1$ the speciation rate of a species that displays character state 1
	\item[] $q_{01}$ the rate of $0\rightarrow 1 $ transitions.
	\item[] $q_{10}$ the rate of $1\rightarrow 0 $ transitions.
\end{compactitem}


\myNewSlide
\section*{BiSSE model of \citet{MaddisonMO2007}}
Sweep tip-to-root. $D_{N0}(t+\Delta t)$ is the probability of an species with character state 0 at time $t+\Delta t$ being the ancestor of a particular clade of $N$ taxa at time 0. \\
$D_{N0}(t+\Delta t)$ is $(1-\mu_0\Delta t)$ times the sum of:
\begin{table}[htdp]
\begin{center}
\begin{tabular}{|l|l|}
\hline
 $\Pr($No changes in $\Delta t)$ & $(1-q_{01}\Delta t)(1-\lambda_0\Delta t)D_{N0}(t)$  \\
 $\Pr($state change in $\Delta t)$ & $(q_{01}\Delta t)(1-\lambda_0\Delta t)D_{N1}(t)$  \\
 $\Pr($Spec. + extinct. in $\Delta t)$ & $(1-q_{01}\Delta t)(\lambda_0\Delta t)E_0(t)D_{N0}(t)$  \\
 $\Pr($Spec. + extinct. in $\Delta t)$ & $(1-q_{01}\Delta t)(\lambda_0\Delta t)E_0(t)D_{N0}(t)$  \\
\hline
\end{tabular}
\end{center}
\label{default}
\end{table}%

\myNewSlide
\section*{BiSSE model of \citet{MaddisonMO2007}}
$E_{0}(t+\Delta t)$ is the probability of an species with character state 0 at time $t+\Delta t$ giving rise to no descendants at time 0. \\
$E_{0}(t+\Delta t)$ is  the sum of:
\begin{table}[htdp]
\begin{center}
\begin{tabular}{|l|l|}
\hline
 $\Pr($Extinction in $\Delta t)$ & $\mu_0\Delta t$  \\
 $\Pr($No changes in $\Delta t)$ & $(1-\mu_0\Delta t)(1-q_{01}\Delta t)(1-\lambda_0\Delta t)E_{0}(t)$  \\
 $\Pr($State change in $\Delta t)$ & $(1-\mu_0\Delta t)(q_{01}\Delta t)(1-\lambda_0\Delta t)E_{1}(t)$  \\
 $\Pr($Spec. in $\Delta t)$ &  $(1-\mu_0\Delta t)(1-q_{01}\Delta t)(\lambda_0\Delta t)E_{0}(t)^2$  \\
\hline
\end{tabular}
\end{center}
\label{default}
\end{table}%

\myNewSlide
\section*{BiSSE model of \citet{MaddisonMO2007}}
Initial conditions:
\begin{compactitem}
	\item[] A $D_{10}(0) = 1$ term for every tip that has state 0
	\item[] A $D_{11}(0) = 1$ term for every tip that has state 1
	\item[] $E_0(0) = E_1(0) = 0$
\end{compactitem}

You can use likelihood ratios to test if $\lambda_0 = \lambda_1$ and $\mu_0=\mu_1$.

Implemented (by Peter Midford) in Mesquite.

\myNewSlide
\section*{Continuous characters}

\myNewSlide
\includepdf[pages={3-4}]{../nonfreeimages/pol/pol-indcontrasts.pdf}

\myNewSlide
\includepdf[pages={2}]{../nonfreeimages/pol/pol-indcontrasts.pdf}

\myNewSlide
\includepdf[pages={11-15}]{../nonfreeimages/joe/joe-brownian.pdf}

\myNewSlide
\includepdf[pages={5-14}]{../nonfreeimages/pol/pol-indcontrasts.pdf}



\myNewSlide
\bibliography{phylo}

\end{document}