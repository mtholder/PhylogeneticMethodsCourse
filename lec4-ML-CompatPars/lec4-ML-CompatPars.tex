\documentclass[landscape]{foils}
%\newif\ifpdf
%\ifx\pdfoutput\undefined
%\pdffalse % we are not running PDFLaTeX
%\else
%\pdfoutput=1 % we are running PDFLaTeX
%\pdftrue
%\fi

%\ifpdf
\usepackage[pdftex]{graphicx}
%\else
%\usepackage{graphicx}
%\fi

%\ifpdf
\DeclareGraphicsExtensions{.pdf, .jpg, .tif, .png}
%\else
%\DeclareGraphicsExtensions{.eps, .jpg}
%\fi

%\usepackage{pslatex}
\usepackage{tabularx,dcolumn, graphicx, amsfonts,amsmath}  
\usepackage[sectionbib]{natbib}
\bibliographystyle{apalike}
\usepackage{picinpar}
\usepackage{multirow}
\usepackage{rotating}
\usepackage{paralist} %compactenum
\setlength{\voffset}{-0.5in}
%\setlength{\hoffset}{-0.5in}
%\setlength{\textwidth}{10.5in}
\setlength{\textheight}{7in}
\setlength{\parindent}{0pt}
%\pagestyle{empty}
%\renewcommand{\baselinestretch}{2.0}
\DeclareMathSymbol{\expect}{\mathalpha}{AMSb}{'105}
\def\p{\rm p}
\def\pp{\rm P}
% this are commands that come with the color package
\usepackage{color}
\usepackage{fancyhdr}


\pagestyle{empty}
%define colors
\definecolor{mediumblue}{rgb}{0.0509,0.35,0.568}
\definecolor{blue}{rgb}{0.0109,0.15,0.468}
\definecolor{black}{rgb}{0.04,0.06,0.2}
\definecolor{darkblue}{rgb}{0.03,0.1,0.2}
\definecolor{darkgreen}{rgb}{0.03,0.5,0.2}
\definecolor{lightblue}{rgb}{0.85,0.9333,0.95}
\definecolor{lightblue2}{rgb}{0.270588, 0.45098, 0.701961}
\definecolor{white}{rgb}{1.0,1.0,1.0}
\definecolor{yellow}{rgb}{0.961,0.972,0.047}
\definecolor{red}{rgb}{0.9,0.1,0.1}
\definecolor{orange}{rgb}{1.0,0.4,0.0}
\definecolor{grey}{rgb}{0.5,0.5,0.5}
\definecolor{violet}{rgb}{0.619608, 0.286275, 0.631373}
\definecolor{mybackgroundcolor}{rgb}{1.0,1.0,1.0}

%\definecolor{light}{rgb}{.5,0.5,0.0}
\definecolor{light}{rgb}{.3,0.3,0.3}

% sets backgroundcolor for whole document 
\pagecolor{mybackgroundcolor}
% sets text color
%\color{black}
% see below for an example how to change just a few words
% using \textcolor{color}{text}

\font \courier=pcrb scaled 2000
\newcommand{\notetoself}[1]{{\textsf{\textsc{\color{red} #1}}}\\}

\newcommand{\answer}[1]{{\sf \color{red} #1}}

\usepackage{pdfpages}

\newcommand{\section}{\secdef \newsection\newsection}
%\renewcommand{\labelitemi}{\includegraphics[width=5mm]{images/bullet.pdf}}
\newcommand{\newsection}[1]{%
{
	\par\flushleft\large\sf\bfseries \vskip -2cm #1\\\rule[0.7\baselineskip]{\textwidth}{0.5mm}\par}}

\newcommand{\subsection}{\secdef \test\test}
\newcommand{\test}[1]{%
	{\par\flushleft\normalsize\sf\bfseries #1: }}
\newcommand{\M}{\mathcal{M}}
\newcommand{\prob}{{\rm Prob~}}
\def\showy#1{{\normalsize\sf\bfseries #1}}
\def\donotuse#1{}

\newcommand{\entrylabel}[1]{\mbox{#1}\hfil}
\newenvironment{entry}
	{\begin{list}{}%
		{\renewcommand{\makelabel}{\entrylabel}%
		\setlength{\labelwidth}{35pt}%
		\setlength{\leftmargin}{\labelwidth+\labelsep}%
	}%
	{\end{list}}}

\newcommand{\poltext}{{\copyright\ 2002--2010 by Paul O. Lewis -- Modified by  Mark Holder with permission from Paul Lewis}}

\newcommand{\pol}{{\footnotesize \poltext}}
\newcommand{\myBackground}{\begin{picture}(0,0)(0,0)  \put(-40,-70){\makebox(0,0)[l]{\includegraphics[width=33cm]{images/baby_blue.jpg}}} \end{picture}}
\newcommand{\myFooter}{}
%\begin{picture}(0,0)(0,0)
%	\put(0,-185){\pol}
%\end{picture}}
\newcommand{\myNewSlide}{\newpage\myFooter} % \myBackground}

\usepackage{url}
\usepackage{hyperref}
\hypersetup{backref,  linkcolor=blue, citecolor=red, colorlinks=true, hyperindex=true}

\begin{document}
\pagecolor{white}
\unitlength=1mm

\myNewSlide
\begin{table}[htdp]
\begin{center}
\label{coloredPerfect}
\begin{tabular}{|c|c|c|c|c|c|c|c|c|c|c|}
\hline 
 & \multicolumn{10}{c|}{Character \#} \\ 
Taxon &\color{blue} 1 & \color{blue} 2 & \color{blue} 3 & \color{blue} 4 & \color{blue} 5 & \color{green} 6 & \color{green} 7 & \color{green} 8 & \color{green} 9 & \color{red} 10  \\ 
\hline 
A &    \color{blue} 0 & \color{blue} 0 & \color{blue} 0 & \color{blue} 0 & \color{blue} 0 & \color{green} 0 & \color{green} 0 & \color{green} 0 & \color{green} 0 & \color{red} 0 \\
B &    \color{blue} 1 & \color{blue} 0 & \color{blue} 0 & \color{blue} 0 & \color{blue} 0 & \color{green} 1 & \color{green} 1 & \color{green} 1 & \color{green} 1 & \color{red} 1 \\
C &    \color{blue} 0 & \color{blue} 1 & \color{blue} 1 & \color{blue} 1 & \color{blue} 0 & \color{green} 1 & \color{green} 1 & \color{green} 1 & \color{green} 1 & \color{red} 1 \\
D &    \color{blue} 0 & \color{blue} 0 & \color{blue} 0 & \color{blue} 0 & \color{blue} 1 & \color{green} 1 & \color{green} 1 & \color{green} 1 & \color{green} 1 & \color{red} 0 \\
\hline 
\end{tabular}
\end{center}
\end{table}

\myNewSlide
\begin{center}
\begin{picture}(-90,0)(20,20)
	\thicklines
	\put(-26,17){B} 
	\put(14,-53){C} 
	\put(-46,-93){D}
	\put(-50,20){\line(1,0){20}} 
	\put(-50,-50){\line(1,0){60}} 
	\put(-50,-50){\line(0,1){70}} 
	\put(-70,-10){\line(1,0){20}} 
	\put(-70,-90){\line(1,0){20}} 
	\put(-70,-90){\line(0,1){80}} 
	\put(-150,-50){\line(1,0){80}}
	\put(-150,-120){\line(1,0){10}} 
	\put(-150,-120){\line(0,1){70}} 
	\put(-136,-123){A} 
	\put(50,7){B C} 
	\put(-40,17){\color{blue} \line(0,1){6}}
	\put(-42,7){\color{blue}1}
	\put(-33,-53){\color{blue} \line(0,1){6}}
	\put(-24,-53){\color{blue} \line(0,1){6}}
	\put(-15,-53){\color{blue} \line(0,1){6}}
	\put(-35,-63){\color{blue}2 3 4}
	\put(-60,-93){\color{blue} \line(0,1){6}}
	\put(-62,-103){\color{blue}5}
	\put(-60,-13){\color{red} \line(0,1){6}}
	\put(-66,-23){\color{red}10}
	\put(-123,-53){\color{green} \line(0,1){6}}
	\put(-113,-53){\color{green} \line(0,1){6}}
	\put(-104,-53){\color{green} \line(0,1){6}}
	\put(-95,-53){\color{green} \line(0,1){6}}
	\put(-125,-63){\color{green}6 7 8 9}
	\put(55,10){\oval(30,20)} 
	\put(55,-10){D} 
	\put(65,5){\oval(60,40)} 
	\put(55,-27){A}
\end{picture}
\end{center}

\myNewSlide
Interestingly, without polarization Hennig's method
can infer unrooted trees.
We can get the tree topology, but be unable to tell 
paraphyletic from monophyletic groups.

The outgroup method amounts to inferring an unrooted
tree and then rooting the tree on the branch that
leads to an outgroup.

\myNewSlide
\begin{center}
\begin{picture}(-90,0)(20,20)
	\thicklines
	\put(-26,17){B} 
	\put(14,-53){C} 
	\put(-99,17){D}
	\put(-50,20){\line(1,0){20}} 
	\put(-50,-50){\line(1,0){60}} 
	\put(-50,-50){\line(0,1){70}} 
	\put(-70,-10){\line(1,0){20}} 
	\put(-90,20){\line(1,0){20}} 
	\put(-70,-50){\line(0,1){70}} 
	\put(-140,-50){\line(1,0){70}}
	\put(-150,-55){A} 
	\put(50,7){B}
	\put(50,-20){A}
 	\put(35,-5){\color{red}\line(1,0){70}} 
	\put(80,7){C} 
	 \put(80,-20){D} 
	\put(-40,17){\color{blue} \line(0,1){6}}
	\put(-42,7){\color{blue}1}
	\put(-33,-53){\color{blue} \line(0,1){6}}
	\put(-24,-53){\color{blue} \line(0,1){6}}
	\put(-15,-53){\color{blue} \line(0,1){6}}
	\put(-35,-63){\color{blue}2 3 4}
	\put(-80,17){\color{blue} \line(0,1){6}}
	\put(-82,7){\color{blue}5}
	\put(-60,-13){\color{red} \line(0,1){6}}
	\put(-66,-23){\color{red}10}
	\put(-123,-53){\color{green} \line(0,1){6}}
	\put(-113,-53){\color{green} \line(0,1){6}}
	\put(-104,-53){\color{green} \line(0,1){6}}
	\put(-95,-53){\color{green} \line(0,1){6}}
	\put(-125,-63){\color{green}6 7 8 9}
\end{picture}
\end{center}


\myNewSlide
\section*{Inadequacy of logic}
Unfortunately, though Hennigian logic is valid we quickly find that we
do not have a reliable method of generating accurate homology statements.

The logic is valid, but we don't know that the premises are true.  

In fact, we almost always find that it is impossible for all of our premises to be true.

\myNewSlide
\section*{Character conflict}
\begin{table}[htdp]
\begin{center}
\begin{tabular}{l|l}
 &  \\
\hline
{\em Homo sapiens} & {\tt A{\color{red} G}TTCAAG{\color{green} T}} \\
{\em Rana catesbiana}  & {\tt A{\color{red} A}TTCAAG{\color{green} T}} \\
{\em Drosophila melanogaster}  & {\tt A{\color{red} G}TTCAAG{\color{green} C}} \\
{\em C. elegans}  & {\tt A{\color{red} A}TTCAAG{\color{green} C}} \\
\end{tabular}
\end{center}
\label{default}
\end{table}
The red character implies that either ({\em Homo + Drosophila}) is a group (if G is derived) and/or
({\em Rana + C. elegans}) is a group.\\
The green character implies that either ({\em Homo + Rana}) is a group (if T is derived) and/or
({\em Drosophila + C. elegans}) is a group.\\
The green and red character cannot both be correct.


\myNewSlide
\begin{table}[htdp]
\begin{center}
\label{coloredHomoplasy}
\begin{tabular}{|c|c|c|c|c|c|c|c|c|c|c|c|c|c|}
\hline 
 & \multicolumn{12}{c|}{Character \#} \\ 
Taxon &\color{blue} 1 & \color{blue} 2 & \color{blue} 3 & \color{blue} 4 & \color{blue} 5 & \color{green} 6 & \color{green} 7 & \color{green} 8 & \color{green} 9 & \color{red} 10 & \color{red} 11 &  \color{red} 12   \\ 
\hline 
A & \color{blue} 0 & \color{blue} 0 & \color{blue} 0 & \color{blue} 0 & \color{blue} 0 & \color{green} 0 & \color{green} 0 & \color{green} 0 & \color{green} 0 & \color{red} 0 & \color{red} 0 & \color{red} 0  \\
B & \color{blue} 1 & \color{blue} 0 & \color{blue} 0 & \color{blue} 0 & \color{blue} 0 & \color{green} 1 & \color{green} 1 & \color{green} 1 & \color{green} 1 & \color{red} 1 & \color{red} 1 & \color{red} 1 \\
C &    \color{blue} 0 & \color{blue} 1 & \color{blue} 1 & \color{blue} 1 & \color{blue} 0 & \color{green} 1 & \color{green} 1 & \color{green} 1 & \color{green} 1 & \color{red} 1 & \color{red} 1 & \color{red} 0\\
D &    \color{blue} 0 & \color{blue} 0 & \color{blue} 0 & \color{blue} 0 & \color{blue} 1 & \color{green} 1 & \color{green} 1 & \color{green} 1 & \color{green} 1 & \color{red} 0 & \color{red} 0 & \color{red} 1\\
\hline 
\end{tabular}
\end{center}
\end{table}



\myNewSlide
\begin{figure}
\begin{center}
\label{mpnesting}
\begin{picture}(0,0)(20,20)
	\thicklines
	\put(15,-35){C} 
	\put(-45,-35){B} 
	\put(-45,-75){D} 
	\put(19,-32){{\color{blue}\oval(40,40)}} 
	\put(19,-32){{\color{blue}\oval(30,30)}} 
	\put(19,-32){{\color{blue}\oval(20,20)}} 
	\put(-41,-72){{\color{blue}\oval(20,20)}} 
	\put(-41,-32){{\color{blue}\oval(20,20)}} 
	\put(-40,-52){{\color{red}\oval(30,70)}} 
	\put(-10,-32){{\color{red}\oval(120,45)}} 
	\put(-10,-32){{\color{red}\oval(125,52)}} 
	\put(15,-47){{\color{green}\oval(180,95)}} 
	\put(15,-47){{\color{green}\oval(185,102)}} 
	\put(15,-47){{\color{green}\oval(190,109)}} 
	\put(15,-47){{\color{green}\oval(195,116)}} 
	\put(135,-30){A} 
\end{picture}
\end{center}
\vskip 4.1cm
\end{figure}


\myNewSlide
\section*{Character conflict}
Two characters are compatible if they can both be mapped on the 
same tree so that all of the character states displayed could 
be homologous.

Incompatible characters are evidence of {\em homoplasy} in the data

Homoplasy literally means the ``same change'' has occurred more than 
once in the evolutionary history of the group.

The presence of homoplasy undermines Hennigian analyses.


\myNewSlide
\begin{picture}(0,0)(0,0) 
	\put(35,-113){\makebox(0,0)[l]{\includegraphics[scale=1.]{../images/all_matrices.pdf}}}
	\put(-20,-50){white = space}
	\put(-15,-60){of all possible}	
	\put(-15,-70){matrices}	
\end{picture}
\myNewSlide
\begin{picture}(0,0)(0,0) 
	\put(35,-113){\makebox(0,0)[l]{\includegraphics[scale=1.]{../images/110_matrices.pdf}}}
	\put(-23,-50){blue = space}
	\put(-23,-60){of matrices with}	
	\put(-23,-70){the pattern:}	
	\put(-10,-85){{\tt A B C D}}	
	\put(-10,-95){{\tt - * * -}}	
\end{picture}
\myNewSlide
\begin{picture}(0,0)(0,0) 
	\put(35,-113){\makebox(0,0)[l]{\includegraphics[scale=1.]{../images/101_matrices.pdf}}}
	\put(-23,-50){red = space}
	\put(-23,-60){of matrices with}	
	\put(-23,-70){the pattern:}	
	\put(-10,-85){{\tt A B C D}}	
	\put(-10,-95){{\tt - * - *}}	
\end{picture}

\myNewSlide
\begin{picture}(0,0)(0,0) 
	\put(35,-113){\makebox(0,0)[l]{\includegraphics[scale=1.]{../images/011_matrices.pdf}}}
	\put(-23,-50){yellow = space}
	\put(-23,-60){of matrices with}	
	\put(-23,-70){the pattern:}	
	\put(-10,-85){{\tt A B C D}}	
	\put(-10,-95){{\tt - - * *}}	
\end{picture}

\myNewSlide
\begin{picture}(0,0)(0,0) 
	\put(35,-113){\makebox(0,0)[l]{\includegraphics[scale=1.]{../images/all_matrices_partitioned.pdf}}}
	\put(-23,-50){all eight}
	\put(-23,-60){categories of}	
	\put(-23,-70){matrices}	
\end{picture}

\myNewSlide
\begin{picture}(0,0)(0,0) 
	\put(35,-113){\makebox(0,0)[l]{\includegraphics[scale=1.]{../images/treeAB_matrices.pdf}}}
	\put(-23,-50){blue = space}
	\put(-23,-60){of matrices }	
	\put(-23,-70){compatible}	
	\put(-23,-80){with tree:}	
	\put(-20,-95){{\tt (A,(B,C),D)}}	
\end{picture}

\myNewSlide
\begin{picture}(0,0)(0,0) 
	\put(35,-113){\makebox(0,0)[l]{\includegraphics[scale=1.]{../images/treeAC_matrices.pdf}}}
	\put(-23,-50){blue = space}
	\put(-23,-60){of matrices }	
	\put(-23,-70){compatible}	
	\put(-23,-80){with tree:}	
	\put(-20,-95){{\tt (A,C,(B,D))}}	
\end{picture}

\myNewSlide
\begin{picture}(0,0)(0,0) 
	\put(35,-113){\makebox(0,0)[l]{\includegraphics[scale=1.]{../images/treeBC_matrices.pdf}}}
	\put(-23,-50){blue = space}
	\put(-23,-60){of matrices }	
	\put(-23,-70){compatible}	
	\put(-23,-80){with tree:}	
	\put(-20,-95){{\tt (A,B,(C,D))}}	
\end{picture}

\myNewSlide
\begin{picture}(0,0)(0,0) 
	\put(35,-113){\makebox(0,0)[l]{\includegraphics[scale=1.]{../images/inference_from_matrices.pdf}}}
	\put(-25,-20){Hennigian:}
	\put(-25,-30){grey = any tree}
	\put(-25,-40){blue = B+C}
	\put(-25,-50){red = B+D}
	\put(-25,-60){yellow = C+D}
	\put(-25,-70){white = no tree }
	\put(-20,-80){(conflicting }
	\put(-20,-90){characters) }
\end{picture}

\myNewSlide
\begin{picture}(0,0)(20,20)
	\put(35,-173){\makebox(0,0)[l]{\includegraphics[scale=1.5]{../images/3trees_to_data.pdf}}}
\end{picture}

\myNewSlide
What can we do if our data end up in the white (character conflict) or grey (uninformative characters only) zone?

\begin{itemize}
	\item can we detect character conflict?
	\item is there a logic-based solution to the problem of character conflict?
\end{itemize}

\myNewSlide
\section*{Detecting character conflict in binary characters}
Consider the four possible combinations of states in a two-character matrix.

The characters are incompatible {\em iff} (when you look across all taxa) you see all four state combinations.

\begin{table}[htdp]
\begin{center}
\begin{tabular}{|lc|c|c|}
\hline
& & \multicolumn{2}{c|}{Char 1} \\
& & 0 & 1\\
\hline
\multirow{2}{*}{Char 2} & 0 & $\times$ & $\times$ \\
& 1 & $\times$ & $\times$ \\
\hline
\end{tabular}
\end{center}
\label{default}
\end{table}

\myNewSlide
What can we do if our data end up in the white (character conflict) or grey (uninformative characters only) zone?

\begin{itemize}
	\item {\color{grey} Can we detect character conflict? Yes}
	\item Is there a logic-based solution to the problem of character conflict?
	\begin{itemize}
		\item recoding characters? 
	   \item ``reciprocal illumination''?
	\end{itemize}	
\end{itemize}

\myNewSlide
What can we do if our data end up in the white (character conflict) or grey (uninformative characters only) zone?

\begin{itemize}
	\item {\color{grey} Can we detect character conflict? Yes}
	\item Is there a logic-based solution to the problem of character conflict? No, nothing purely based on logic (and the suggestions for culling data to make matrices suitable for logical inference can lead to unsatisfyingly subjecive analyses).
	\item What can we do?
\end{itemize}
We must have an ``error model''

\myNewSlide
\section*{Statistical inference}
There are many ways to derive estimators, we are going 
to talk about maximum likelihood estimation:
\Large
\begin{eqnarray*}
	\theta & \in & \Theta \\
	X & \in & \mathcal{X} \\
	x & \sim & \Pr(X=x|\theta)\\
	\mathcal{L}(\theta) & = & \Pr(X=x|\theta)\\ 
	\hat{\theta} & = & \arg\max \mathcal{L}(\theta)
\end{eqnarray*}



\myNewSlide
\huge
\begin{picture}(0,0)(20,20)
	\put(35,-153){\makebox(0,0)[l]{\includegraphics[scale=1.5]{../images/tree_to_data0.pdf}}}
	\put(150,-35){$\Theta$}
\end{picture}

\myNewSlide
\begin{picture}(0,0)(20,20)
	\put(35,-153){\makebox(0,0)[l]{\includegraphics[scale=1.5]{../images/tree_to_data0_5.pdf}}}
	\put(150,-35){$\theta\in\Theta$}
\end{picture}

\myNewSlide
\begin{picture}(0,0)(20,20)
	\put(35,-173){\makebox(0,0)[l]{\includegraphics[scale=1.5]{../images/data_of_tree_to_data.pdf}}}
	\put(170,-55){$\mathcal{X}$}
\end{picture}


\myNewSlide
\begin{picture}(0,0)(20,20)
	\put(35,-173){\makebox(0,0)[l]{\includegraphics[scale=1.5]{../images/tree_to_data1_5.pdf}}}
	\put(170,-55){$\Pr(X=x|\theta)$}
\end{picture}


\myNewSlide
\begin{picture}(0,0)(20,20)
	\put(35,-173){\makebox(0,0)[l]{\includegraphics[scale=1.5]{../images/tree_to_data2.pdf}}}
	\put(160,-55){$x \sim \Pr(X=x|\theta)$}
\end{picture}

\myNewSlide
\begin{picture}(0,0)(20,20)
	\put(35,-153){\makebox(0,0)[l]{\includegraphics[scale=1.5]{../images/tree_to_data2_5.pdf}}}
	\put(60,-70){$x$ represents}
\end{picture}

\myNewSlide
\begin{picture}(0,0)(20,20)
	\put(35,-153){\makebox(0,0)[l]{\includegraphics[scale=1.5]{../images/tree_to_data3.pdf}}}
	\put(20,10){$\theta_1$}
	\put(20,-60){$\theta_2$}
	\put(20,-130){$\theta_3$}
\end{picture}

\myNewSlide
\begin{picture}(0,0)(20,20)
	\put(35,-153){\makebox(0,0)[l]{\includegraphics[scale=1.5]{../images/tree_to_data3_5.pdf}}}
	\put(20,10){$\theta_1$}
	\put(20,-60){$\theta_2$}
	\put(20,-130){$\theta_3$}
	\put(120,10){$\Pr(x|\theta_1) = 0.00024$}
	\put(115,-30){\large $0.00024$}
\end{picture}

\myNewSlide
\begin{picture}(0,0)(20,20)
	\put(35,-153){\makebox(0,0)[l]{\includegraphics[scale=1.5]{../images/tree_to_data3_5.pdf}}}
	\put(20,10){$\theta_1$}
	\put(20,-60){$\theta_2$}
	\put(20,-130){$\theta_3$}
	\put(120,10){$\Pr(x|\theta_2) = 0.0002$}
	\put(115,-30){\large $0.00024$}
	\put(97,-70){\large $0.0002$}
\end{picture}

\myNewSlide
\begin{picture}(0,0)(20,20)
	\put(35,-153){\makebox(0,0)[l]{\includegraphics[scale=1.5]{../images/tree_to_data3_5.pdf}}}
	\put(20,10){$\theta_1$}
	\put(20,-60){$\theta_2$}
	\put(20,-130){$\theta_3$}
	\put(120,10){$\Pr(x|\theta_3) = 0.00022$}
	\put(115,-30){\large $0.00024$}
	\put(97,-70){\large $0.0002$}
	\put(115,-110){\large $0.00022$}
\end{picture}

\myNewSlide
\begin{picture}(0,0)(20,20)
	\put(35,-153){\makebox(0,0)[l]{\includegraphics[scale=1.5]{../images/tree_to_data4.pdf}}}
	\put(120,-5){$\hat{\theta} = \arg\max \mathcal{L}(\theta)$}
\end{picture}


\myNewSlide
\section*{ML Estimation}
\large
\begin{itemize}
	\item Flexible form of inference
	\item Requires a model: $\Pr(X=x|\theta)$
\end{itemize}
Under mild conditions, ML estimation is asymptotically:
\begin{itemize}
	\item not very biased,
	\item efficient
\end{itemize}
How can we come up with a model?
\end{document}









\myNewSlide
\bibliography{../phylo848}
\end{document}
