\documentclass[landscape]{foils} 
\input{../common-preamble-start}
\input{../preamble.tex}
\usepackage{url}
\usepackage{hyperref}
\hypersetup{backref,  linkcolor=blue, citecolor=red, colorlinks=true, hyperindex=true}


\begin{document}
\pagecolor{white}
\unitlength=1mm
\begin{center}
{\Large Some of these slides have been borrowed from Dr.\ Paul Lewis, Dr.\ Joe Felsenstein. Thanks!}
\vskip 15mm
\large Paul has many great tools for teaching phylogenetics at his web site: \\
\url{http://hydrodictyon.eeb.uconn.edu/people/plewis}
\end{center}

\myNewSlide
\section*{Tree Searching}
\large
Parsimony and ML give us ways to deciding whether one tree is fits our data better than another tree, but \ldots\\
\begin{center}
\Large How do we find the best tree?\\
(or one that is good enough)
\end{center}

\myNewSlide
\begin{picture}(0,0)(0,0)
	\put(-30,-70){\makebox(0,0)[l]{\includegraphics{../nonfreeimages/pol/POL_exhaustive_1.pdf}}}
\end{picture}
\myNewSlide
\begin{picture}(0,0)(0,0)
	\put(-30,-70){\makebox(0,0)[l]{\includegraphics{../nonfreeimages/pol/POL_exhaustive_2.pdf}}}
\end{picture}
\myNewSlide
\begin{picture}(0,0)(0,0)
	\put(-30,-70){\makebox(0,0)[l]{\includegraphics{../nonfreeimages/pol/POL_exhaustive_3.pdf}}}
\end{picture}

\myNewSlide
\normalsize
\begin{tabular}{r| r | c}
Tips & Number of unrooted (binary) trees\\
\hline
4 & 3 &\\
5 & 15 &\\
6 & 105 &\\
7 & 945 &\\
8 & 10,395 &\\
9 & 135,135 &\\
10 & 2,027,025 &\\
\hline
11 & 34,459,425 &\\
12 & 654,729,075 &\\
13 & 13,749,310,575 &\\
14 & 316,234,143,225 &\\
15 & 7,905,853,580,625 &\\
16 & 213,458,046,676,875 &\\
17 & 6,190,283,353,629,375 &\\
18 & 191,898,783,962,510,625 &\\
19 & 6,332,659,870,762,850,625 &\\
20 & 22,164,309,5476,699,771,875 &\\
\hline
21 & 8,200,794,532,637,891,559,375 &\\
22 & 319,830,986,772,877,770,815,625 &\\
23 & 13,113,070,457,687,988,603,440,625 & $>21$ moles of trees\\
\hline
24 & 563,862,029,680,583,509,947,946,875 &\\
25 & 25,373,791,335,626,257,947,657,609,375 &\\
\end{tabular}

\myNewSlide
For $N$ taxa:
\begin{eqnarray*}
\#\mbox{ unrooted, binary trees}	& = & \prod_{i=3}^{N-1} (2i-3) \\
	& = & \prod_{i=4}^{N} (2i-5) \\
\#\mbox{ rooted, binary trees}	& = & \prod_{i=3}^{N} (2i-3) \\
& = & (2N-3) (\#\mbox{ unrooted, binary trees})\\
\end{eqnarray*}
	
\myNewSlide
\section*{Star decomposition}
\begin{picture}(0,0)(0,0)
\put(0,-50){\makebox(0,0)[l]{\includegraphics{../images/star_decomp1.pdf}}}
\end{picture}

\myNewSlide
\section*{Star decomposition}
\begin{picture}(0,0)(0,0)
\put(0,-50){\makebox(0,0)[l]{\includegraphics{../images/star_decomp2.pdf}}}
\end{picture}

\myNewSlide
\section*{Star decomposition}
\begin{picture}(0,0)(0,0)
\put(0,-50){\makebox(0,0)[l]{\includegraphics{../images/star_decomp3.pdf}}}
\end{picture}

\myNewSlide
\section*{Star decomposition}
\begin{itemize}
	\item Very greedy. Once a pair of species are joined, they will not be separated.
	\item Neighbor-joining is star decomposition under the balanced minimum evolution criterion
\end{itemize}

\myNewSlide
\begin{picture}(0,0)(0,0)
\put(-30,-90){\makebox(0,0)[l]{\includegraphics{../images/first_100_num_first_step_star_decomp.pdf}}}
\end{picture}

\myNewSlide

\begin{picture}(0,0)(0,0)
\put(-30,-90){\makebox(0,0)[l]{\includegraphics{../images/first_100_num_star_decomp.pdf}}}
\end{picture}

\myNewSlide

\begin{picture}(0,0)(0,0)
\put(-30,-90){\makebox(0,0)[l]{\includegraphics{../images/num_star_decomp_loglog.pdf}}}
\end{picture}

\myNewSlide
\section*{Star decomposition}Number of trees scored for $N$ taxa (if we decrease $i$ in the summation):
	\begin{eqnarray*}
	\#\mbox{ trees scored}	& = & 3 \sum_{i=N}^{5} {i \choose 2} \\
		& = & 3\sum_{i=N}^{5} \frac{i(i-1)}{2} 
	\end{eqnarray*}
		Thus, star decomposition is $O(N^3)$. 
		
		For N=10:
		\[158 = 45 + 36 + 28 + 21 + 15 + 10 + 3\]

\myNewSlide
\section*{Star decomposition can fail even when there is no homoplasy}
\begin{center}\begin{table}[htdp]
\begin{center}
\begin{tabular}{|c|c|}
\hline
A &	0000{\color{red}00} \\
B &	0000{\color{red} 00} \\
C &	1100{\color{red} 00} \\
D &	1100{\color{red} 00} \\
E &	1010{\color{red} 11} \\
F &	1011{\color{red} 11} \\ 
G &	1011{\color{red} ??} \\ 
\hline
\end{tabular}
\end{center}
\end{table}%
\end{center}
Star decomposition (under parsimony) will join E and F in the first step based on characters 5 and 6 (shown in red), even though (((A,B),(C,D)),E,(F,G)) shows no homoplasy.
\myNewSlide
\section*{Stepwise addition}
\begin{picture}(0,0)(0,0)
\put(-10,-50){\makebox(0,0)[l]{\includegraphics{../images/stepwise1.pdf}}}
\end{picture}

\myNewSlide
\section*{Stepwise addition}
\begin{picture}(0,0)(0,0)
\put(-10,-50){\makebox(0,0)[l]{\includegraphics{../images/stepwise2.pdf}}}
\end{picture}

\myNewSlide
\section*{Stepwise addition}
\begin{picture}(0,0)(0,0)
\put(-10,-50){\makebox(0,0)[l]{\includegraphics{../images/stepwise3.pdf}}}
\end{picture}

\myNewSlide
\section*{Stepwise addition}
\begin{itemize}
	\item Greedy, but can introduce a new taxon on the path between taxa that have already been joined.
	\item The tree can depend on the input order of the taxa
	\item Number of trees scored for $N$ taxa :
	\begin{eqnarray*}
	 \#\mbox{ trees scored}	& = & \sum_{i=3}^{N-1} {(2i-3)} \\
		& = & (N-1)(N-3) 
	\end{eqnarray*}
		Thus, stepwise addition is $O(N^2)$. For N=10:
		\[63 = 3 + 5 + 7 + 9 + 11 + 13 + 15\]
\end{itemize}


\myNewSlide
\section*{Branch and bound}
\begin{picture}(0,0)(0,0)
\put(-10,-70){\makebox(0,0)[l]{\includegraphics{../images/bandb-1.pdf}}}
\put(80, 0){\normalsize Construct and score an initial estimate of the tree}
\end{picture}

\myNewSlide
\section*{Branch and bound}
\begin{picture}(0,0)(0,0)
\put(-10,-70){\makebox(0,0)[l]{\includegraphics{../images/bandb-2.pdf}}}
\put(40, -15){\normalsize Initial tree}
\put(80, 0){\normalsize Start building a new tree of three taxa}
\end{picture}

\myNewSlide
\section*{Branch and bound}
\begin{picture}(0,0)(0,0)
\put(-10,-70){\makebox(0,0)[l]{\includegraphics{../images/bandb-2-5.pdf}}}
\put(40, -15){\normalsize Initial tree}
\put(80, 0){\normalsize Consider all positions for the next taxon}
\end{picture}

\myNewSlide
\section*{Branch and bound}
\begin{picture}(0,0)(0,0)
\put(-10,-70){\makebox(0,0)[l]{\includegraphics{../images/bandb-3.pdf}}}
\put(40, -15){\normalsize Initial tree}
\put(80, 0){\normalsize Score all of these trees}
\end{picture}

\myNewSlide
\section*{Branch and bound}
\begin{picture}(0,0)(0,0)
\put(-10,-70){\makebox(0,0)[l]{\includegraphics{../images/bandb-4.pdf}}}
\put(40, -15){\normalsize Initial tree}
\put(70, 0){\normalsize Consider adding the next taxon to all viable backbone trees}
\put(105, -41){\normalsize Reject for exceeding}
\put(105, -51){\normalsize the bound}
\put(105, -80){\normalsize Reject for exceeding}
\put(105, -90){\normalsize the bound}
\end{picture}

\myNewSlide
\section*{Branch and bound}
\begin{itemize}
	\item Guaranteed to find all of the optimal trees.
	\item Relies on the fact that the score always gets worse as you add taxa.
	\item Can be as fast as stepwise addition (if you have lots of very clean data).
	\item Can be as slow as an exhaustive search (if you have little data and/or data with lots of conflict).
	\item Bound can be tightened by adding a lower bound on the number of steps to be added when you add more taxa to the tree (based on new forms of character conflict in unattached taxa).
\end{itemize}

\myNewSlide
\section*{Trying to improve a tree}
Neither stepwise addition is guaranteed to return the best tree(s), but branch-and-bound (or exhaustive searching) is frequently infeasible.

Heuristic hill-climbing searches can work quite well:
\begin{compactenum}
	\item Start with a tree
	\item Score the tree
	\item Consider a new tree within the neighborhood of the current tree:
	\begin{compactenum}
		\item Score the new tree.
		\item If the new tree has a better tree, use it as the ``current tree''
		\item Stop if there are no other trees within the neighborhood to consider.
	\end{compactenum}
\end{compactenum}
These are {\bf not} guaranteed to find even one of the optimal trees.

The most common way to explore the neighborhood of a tree is to swap the branches of the tree to construct similar trees.

\myNewSlide
\includepdf[pages={2-19}]{../nonfreeimages/joe/joe_week2.pdf} 

\myNewSlide
\section*{Nearest Neighbor Interchanges searches}
\begin{compactenum}
	\item Consider the two possible NNI neighbors ``around'' each internal edge in the tree
	\item Return the tree set of trees that is at least as good as all of the NNI neighbors.
	\item Number of rearrangements scored {\em per tree} $N$ taxa :
		\[\#\mbox{ rearrangements scored}	= 2\times(N-3) \]
		But, there is no upper bound on the number of trees encountered on the path from the initial tree to the final tree.
\end{compactenum}

\myNewSlide
\section*{Schoenberg graph -- edges connect NNI neighbors}
\begin{picture}(0,0)(0,0)
\put(-30,-50){\makebox(0,0)[l]{\includegraphics{../images/schoenberg.pdf}}}
\end{picture}

\myNewSlide
\section*{Tree ``Islands'' possible}
An $Op-L$ tree island \citep[{\em sensu}][]{Maddison1991}:
A set of trees with score $\leq L$ that are connected to each other by $Op$ operations such that you can get from any tree in the set to any other tree by repeated $Op$ changes and all intermediate trees along the path are also members of the set.

The following Schoenberg graph shows the scores of the 15 trees on the following dataset (contrived data by POL):
\begin{center}\begin{verbatim}
A	ACGCAGGT
B	ATGGTGAT
C	GCTCACGG
D	ACTGTCGT
E	GTTCTGAG
\end{verbatim}\end{center}

\myNewSlide
\section*{Schoenberg graph with parsimony scores}
\begin{picture}(0,0)(0,0)
\put(-30,-50){\makebox(0,0)[l]{\includegraphics{../images/schoenberg_scored.pdf}}}
\end{picture}

\myNewSlide
\section*{Schoenberg graph showing the single NNI-15 island}
\begin{picture}(0,0)(0,0)
\put(-10,-70){\makebox(0,0)[l]{\includegraphics{../images/schoenberg_15_island.pdf}}}
\end{picture}

\myNewSlide
\section*{Schoenberg graph showing the single NNI-14 island}
\begin{picture}(0,0)(0,0)
\put(-10,-70){\makebox(0,0)[l]{\includegraphics{../images/schoenberg_14_island.pdf}}}
\end{picture}

\myNewSlide
\section*{Schoenberg graph showing the both NNI-13 islands}
\begin{picture}(0,0)(0,0)
\put(-10,-70){\makebox(0,0)[l]{\includegraphics{../images/schoenberg_13_island.pdf}}}
\end{picture}

\myNewSlide
\section*{Tree Islands implications}
\begin{compactenum}
	\item Islands can be larger than 1 tree -- so we must consider ties if we want to find all trees that optimize the score.
	\item There can be more than 1 island for good scores -- so even swapping to completion on all optimal trees found in a search is guaranteed to succeed.
	\item The delimitation of an island depends on tree changing operation used.
\end{compactenum}

\myNewSlide
\section*{Heuristics explore ``Tree Space''}
\begin{picture}(0,0)(0,0)
\put(-10,-50){\makebox(0,0)[l]{\includegraphics{../images/landscape.pdf}}}
\put(125, -20){\normalsize Most commonly used methods}
\put(125, -30){\normalsize are ``hill-climbers.''}
\put(125, -50){\normalsize Multiple optima found by}
\put(125, -60){\normalsize repeating searches from}
\put(125, -70){\normalsize different origins.}
\put(125, -90){\normalsize Severity of the problem}
\put(125, -100){\normalsize of multiple optima}
\put(125, -110){\normalsize depends on step size.}
\end{picture}


\myNewSlide
\section*{Subtree Pruning Regrafting (SPR) and Tree Bisection Reconnection (TBR)}
\begin{picture}(0,0)(0,0)
\put(-10,-50){\makebox(0,0)[l]{\includegraphics[scale=0.9]{../images/sprtbr.pdf}}}
\put(5,-72){\normalsize SPR maintains}
\put(5,-82){\normalsize subtree rooting}
\put(90,-118){\normalsize TBR tries all}
\put(90,-128){\normalsize possible rootings}
\end{picture}

\myNewSlide
\section*{1-Edge-contract Refine}
\begin{picture}(0,0)(0,0)
\put(-10,-50){\makebox(0,0)[l]{\includegraphics[scale=0.9]{../images/1ecr.pdf}}}
\end{picture}

\myNewSlide
\section*{2-Edge-contract Refine}
\begin{picture}(0,0)(0,0)
\put(-10,-50){\makebox(0,0)[l]{\includegraphics[scale=0.9]{../images/2ecr.pdf}}}
\end{picture}

\myNewSlide
\section*{Many other heuristic strategies proposed}
\begin{itemize}
	\item Swapping need not include {\em all} neighbors (RAxML, {\tt reconlimit} in PAUP*)
	\item ``lazy'' scoring of swaps (RAxML)
	\item Ignoring (at some stage) interactions between different branch swaps (PHYML)
	\item Stochastic searches
	\begin{itemize}
		\item Genetic algorithms (GAML, MetaPIGA, GARLI)
		\item Simulated annealing
	\end{itemize}
	\item Divide and conquer methods (the sectortial searching of Goloboff, 1999; Rec-I-DCM3 Roshan {2004})
	\item Data perturbation methods ({\em e.g.} Kevin Nixon's  ``ratchet'')
\end{itemize}

\myNewSlide
\begin{picture}(0,0)(0,0)
\put(-30,-80){\makebox(0,0)[l]{\includegraphics[scale=1.0]{../images/gaVariablePop.pdf}}}
\end{picture}

\myNewSlide
\begin{picture}(0,0)(0,0)
\put(-30,-80){\makebox(0,0)[l]{\includegraphics[scale=1.0]{../images/gaScored.pdf}}}
\end{picture}

\myNewSlide
\begin{picture}(0,0)(0,0)
\put(-30,-80){\makebox(0,0)[l]{\includegraphics[scale=1.0]{../images/gaFitness.pdf}}}
\end{picture}

\myNewSlide
\begin{picture}(0,0)(0,0)
\put(-30,-80){\makebox(0,0)[l]{\includegraphics[scale=1.0]{../images/gaSelection.pdf}}}
\end{picture}

\myNewSlide
\begin{picture}(0,0)(0,0)
\put(-30,-80){\makebox(0,0)[l]{\includegraphics[scale=1.0]{../images/gaMutation.pdf}}}
\end{picture}


\myNewSlide
\section*{Divide-and-Conquer Methods}
The basic outline of a phylogenetic  Divide-and-Conquer approach is:
\begin{enumerate}
	\item {\bf Decompose} a starting tree into subsets of the taxa.
	\item {\bf Improve} the tree for each of the subsets of taxa.
	\item {\bf Merge} the resulting trees into a tree for the full set of taxa.
	\item {\bf Refine} the full tree (it will often have polytomies).
	\item {\bf Improve} the full tree using a simple (and fast) heuristic.
\end{enumerate}
Examples include Rec-I-DCM3 by Roshan {\em et al.}(2004).
See Goloboff and Pol ({\em Systematic Biology}, 2007) for a contrasting viewpoint about the relative efficiency of Rec-I-DCM3 compared to heuristics implemented in TNT.

\myNewSlide
\section*{Step 1: Leaf set decomposition}
In Rec-I-DCM3 Roshan {\em et al.} (2004):
\begin{itemize}
	\item A tree is divided (``decomposed'') into 4 trees around a central edge.  The edge is chosen such that it comes as close as possible to dividing the taxa into 2 equally-sized groups.
	\item The short quartet (taxa closest to this edge in each of the 4 directions) is selected.
	\item 4 sub-problems are produced.  Each contains 1 subtree connected to the central edge and all leaves that are a part of the short quartet.
\end{itemize}

\myNewSlide
\begin{picture}(0,0)(0,0)  \put(40,0){\makebox(-0,-30)[l]{\includegraphics[scale=1]{../images/dcmTree.pdf}}}
\end{picture}
\myNewSlide
\begin{picture}(0,0)(0,0)  \put(40,0){\makebox(-0,-30)[l]{\includegraphics[scale=1]{../images/centralEdge.pdf}}}
\end{picture}
\myNewSlide
\begin{picture}(0,0)(0,0)  \put(40,0){\makebox(-0,-30)[l]{\includegraphics[scale=1]{../images/shortQuartet.pdf}}}
\end{picture}

\myNewSlide
\begin{picture}(0,0)(0,0)  \put(40,0){\makebox(-0,-30)[l]{\includegraphics[scale=1]{../images/decomp0.pdf}}}
\end{picture}

\myNewSlide
\begin{picture}(0,0)(0,0)  \put(40,0){\makebox(-0,-30)[l]{\includegraphics[scale=1]{../images/decomp1.pdf}}}
\end{picture}

\myNewSlide
\begin{picture}(0,0)(0,0)  \put(40,0){\makebox(-0,-30)[l]{\includegraphics[scale=1]{../images/decomp2.pdf}}}
\end{picture}

\myNewSlide
\begin{picture}(0,0)(0,0)  \put(40,0){\makebox(-0,-30)[l]{\includegraphics[scale=1]{../images/decomp3.pdf}}}
\end{picture}

\myNewSlide
\section*{Step 2: Tree improvement}
Simply a tree search on a smaller tree

DCM is a ``meta-method'' that can be used with almost any type of large-scale tree inference. 

\myNewSlide
\section*{Step 3: Tree Merge (Supertree analysis)}
The step of ``glueing'' the trees for subproblems together is a supertree analysis.  


If there is no conflict between the input trees, the problem is trivial.

Roshan et recommend using a Strict Consensus Merger - collapse the minimal number of edges required to make 2 trees display the same tree (for the leaves that they have in common).

\myNewSlide
\begin{picture}(0,0)(0,0)  \put(0,0){\makebox(30,-160)[l]{\includegraphics[scale=0.8]{../images/scm.pdf}}}
\end{picture}

\myNewSlide
\begin{picture}(0,0)(0,0)  \put(0,0){\makebox(30,-160)[l]{\includegraphics[scale=0.8]{../images/scmResult.pdf}}}
\end{picture}


\myNewSlide
\begin{picture}(0,0)(0,0)  \put(0,0){\makebox(-0,-150)[l]{\includegraphics[scale=0.9]{../images/scmCollision.pdf}}}
\end{picture}

\myNewSlide
\section*{Step 4: Tree Refine}
Optional step - some tree searching methods require binary trees

{\bf \large Step 5: Tree Improve}\\\rule[0.7\baselineskip]{\textwidth}{0.5mm}\\
Another ``base method''  tree search (but with a large set of taxa, so the seach often has to be less thorough)

\vskip3cm

\myNewSlide
\begin{picture}(10,20)(10,20)  \put(-40,0){\makebox(-0,-150)[l]{\includegraphics[scale=1.75]{../images/schematic0.pdf}}}
\end{picture}

\myNewSlide
\begin{picture}(10,20)(10,20)  \put(-40,0){\makebox(-0,-150)[l]{\includegraphics[scale=1.75]{../images/schematic1.pdf}}}
\end{picture}

\myNewSlide
\begin{picture}(10,20)(10,20)  \put(-40,0){\makebox(-0,-150)[l]{\includegraphics[scale=1.75]{../images/schematic2.pdf}}}
\end{picture}

\myNewSlide
\begin{picture}(10,20)(10,20)  \put(-40,0){\makebox(-0,-150)[l]{\includegraphics[scale=1.75]{../images/schematic3.pdf}}}
\end{picture}

\myNewSlide
\begin{picture}(10,20)(10,20)  \put(-40,0){\makebox(-0,-150)[l]{\includegraphics[scale=1.75]{../images/schematic4.pdf}}}
\end{picture}

\myNewSlide
\begin{picture}(10,20)(10,20)  \put(-40,0){\makebox(-0,-150)[l]{\includegraphics[scale=1.75]{../images/schematic5.pdf}}}
\end{picture}

\myNewSlide
\section*{Recursion}
A {\em recursive} algorithm is one that calls (invokes) itself.

A definition of the function to compute the factorial is the classic example:
\begin{verbatim}
def factorial(n):
    if n == 1:
        return 1
    else:
        return n * factorial(n - 1)
\end{verbatim}

Recursion is often used when it is easy to perform a few tasks, but then you are faced with the same problem you originally faced, but on a smaller scale.\\ 
Recursive DCM3 arises from the recognition that, when we break our full set of taxa into subsets some of them may {\em still} be too large for thorough searching.  We can use another level of DCM to break them down into smaller problems. 


\myNewSlide
\begin{picture}(10,20)(10,20)  \put(-40,0){\makebox(-0,-150)[l]{\includegraphics[scale=1.75]{../images/recSchematic.pdf}}}
\end{picture}


\myNewSlide
\section*{Iteration }
Because the decompositions are sensitive to the starting tree, we may do a better job decomposing the tree into closely related subtrees if we have a better estimate of the tree. 

So we can simply repeat the whole recursive DCM process

\myNewSlide
\begin{picture}(-10,20)(-10,20)  \put(-40,0){\makebox(-0,-150)[l]{\includegraphics[scale=1.]{../images/recIterSchematic.pdf}}}
\end{picture}

\myNewSlide
\bibliography{phylo}
\end{document}     
     
