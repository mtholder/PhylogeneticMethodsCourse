\documentclass[landscape]{foils} 
\input{../common-preamble-start}
\input{../preamble.tex}
\usepackage{url}
\usepackage{hyperref}
\hypersetup{backref,  linkcolor=blue, citecolor=red, colorlinks=true, hyperindex=true}

\usepackage{pdfpages}

\newcommand{\disruption}{\theta}
\begin{document}
\pagecolor{white}
\unitlength=1mm
\begin{center}
{\Large Some of these slides have been borrowed from Dr.\ Paul Lewis, Dr.\ Joe Felsenstein. Thanks!}
\vskip 15mm
\large Paul has many great tools for teaching phylogenetics at his web site: \\
\url{http://hydrodictyon.eeb.uconn.edu/people/plewis}
\end{center}




\myNewSlide
\includepdf[pages={2-3}]{../nonfreeimages/pol/pol-09-model.pdf} 


\myNewSlide
\section*{The Poisson distribution}
Probability distribution on the number of events when:
\begin{compactenum}
	\item events are assumed to be independent,
	\item the {\em rate} of events some constant, $\mu$, and
	\item the process continues for some duration of time, $t$.
\end{compactenum}
The expectation of the number of events is $\nu = \mu t$.

Note that $\nu$ can be any non-negative number, but the Poisson is a discrete distribution -- it gives the probabilities of the number of events (and this number will always be a non-negative integer).

\myNewSlide
\section*{The Poisson distribution}
\Large
\[\Pr(k\mbox{ events}|\mbox{Expected \# is } \nu) = \frac{\nu^k e^{-\nu}}{k !}
\]

\[\Pr(0\mbox{ events}) = \frac{\nu^0 e^{-\nu}}{0 !} = e^{-\nu} = e^{-\mu t}
\]

\[\Pr(\geq 1\mbox{ events}) = 1 - e^{-\nu} = 1 - e^{-\mu t}
\]

\myNewSlide
\includepdf[pages={13}]{../nonfreeimages/pol/pol-09-model.pdf} 

\myNewSlide
\section*{Probability of T$\rightarrow$G over time $t$}
If $\mu$ is the rate of disruptions, and a branch is $t$ units of time long then:
Let's use $\disruption$ for the rate of any particular ``disruption.''

\[ \mu_{TA} = \mu_{TC} = \mu_{TG} = \mu_{TT} = \disruption \]
\[\mu = 4 \disruption\]

Furthermore, given that there is a disruption the chance of any particular change is
$\frac{1}{4}$

\myNewSlide
\section*{Probability of $T\rightarrow G$ over time $t$}
\begin{eqnarray*}
	\Pr(0\mbox{ disruptions} | t) &  = & e^{-\mu t} \\
	\Pr(\mbox{at least 1 disruption}| t) &  = & 1 - e^{-\mu t} \\
	\Pr(\mbox{last disruption leads to }G) &  = & 0.25 \\
	\Pr(T\rightarrow G | t) & = & 0.25\left(1 - e^{-\mu t}\right) \\
		 & = & 0.25\left(1 - e^{-4\disruption t}\right)
\end{eqnarray*}

\myNewSlide
\includepdf[pages={15}]{../nonfreeimages/pol/pol-09-model.pdf} 

\myNewSlide
\section*{JC transition probabilities}
\large
\begin{eqnarray*}
	\Pr(T\rightarrow A | t) & = & 0.25\left(1 - e^{-4\disruption t}\right) \\
	\Pr(T\rightarrow C | t) & = & 0.25\left(1 - e^{-4\disruption t}\right) \\
	\Pr(T\rightarrow G | t) & = & 0.25\left(1 - e^{-4\disruption t}\right) \\
	\Pr(T\rightarrow T | t) & = & 0.25\left(1 - e^{-4\disruption t}\right) \\
\end{eqnarray*}
but this only adds up to:
\[\left(1 - e^{-4\disruption t}\right)\]
instead of 1!

\myNewSlide
We left out the probability of no disruptions:$ e^{-4\disruption t}$

So:
\begin{eqnarray*}
	\Pr(T\rightarrow A | t) & = & 0.25\left(1 - e^{-4\disruption t}\right) \\
	\Pr(T\rightarrow C | t) & = & 0.25\left(1 - e^{-4\disruption t}\right) \\
	\Pr(T\rightarrow G | t) & = & 0.25\left(1 - e^{-4\disruption t}\right) \\
	\Pr(T\rightarrow T | t) & = & e^{-4\disruption t} + 0.25\left(1 - e^{-4\disruption t}\right) \\
		& = & 0.25 + 0.75e^{-4\disruption t} \\
\end{eqnarray*}

\myNewSlide
\section*{JC transition probabilities}
\begin{eqnarray*}
	\Pr(i\rightarrow j | t) & = & 0.25\left(1 - e^{-4\disruption t}\right) \\
	\Pr(i\rightarrow i | t) & = & 0.25 + 0.75e^{-4\disruption t} \\
\end{eqnarray*}
When $t = 0$, then $e^{-4\disruption t} = 1$, and:
\begin{eqnarray*}
	\Pr(i\rightarrow j | t) & = & 0 \\
	\Pr(i\rightarrow i | t) & = & 1 \\
\end{eqnarray*}

\myNewSlide
\section*{JC transition probabilities}
\begin{eqnarray*}
	\Pr(i\rightarrow j | t) & = & 0.25\left(1 - e^{-4\disruption t}\right) \\
	\Pr(i\rightarrow i | t) & = & 0.25 + 0.75e^{-4\disruption t} \\
\end{eqnarray*}
When $t = \infty$, then $e^{-4\disruption t} = 0$, and:
\begin{eqnarray*}
	\Pr(i\rightarrow j | t) & = & 0.25 \\
	\Pr(i\rightarrow i | t) & = & 0.25 \\
\end{eqnarray*}

\myNewSlide
\includepdf[pages={24-27}]{../nonfreeimages/pol/pol-09-model.pdf} 

\myNewSlide
\section*{JC instantaneous rate matrix - the Q matrix for JC}
The 1 parameter is $\alpha$ (sometimes parameterized in terms of $\mu$).
This is the rate of replacements (``disruptions'' that change the state):

\begin{table}[htdp]
\begin{center}
\begin{tabular}{cc|cccc}
& & \multicolumn{4}{c}{To State} \\
& & A & C & G & T \\
\hline
\multirow{2}{*}{From } & A &  $-3\alpha$ & $\alpha$ & $\alpha$ & $\alpha$    \\
\multirow{2}{*}{State } &C & $\alpha$ & $-3\alpha$ & $\alpha$ & $\alpha$    \\
 &G & $\alpha$ & $\alpha$ & $-3\alpha$ & $\alpha$    \\
 &T & $\alpha$ & $\alpha$ & $\alpha$ & $-3\alpha$    \\
\end{tabular}
\end{center}
\end{table}


\myNewSlide
\section*{Change probabilities}
We can calculate a transition probability matrix as a function of time by:
\[\mathbf{P}(t) = e^{\mathbf{Q}t}\]
The important thing to note is the rates ($\mathbf{Q}$ matrix) is multiplied by the time.

We can't separate rates and times since we always see the effect of their product.

Is a medium level of character divergence:
\begin{compactenum}
	\item medium rate of change and medium amount of time,
	\item high rate, but short time period,
	\item low rate, but a long time period?
\end{compactenum}
\myNewSlide
\section*{JC instantaneous rate matrix again}
What if you do not know the length of time for a branch in the tree?
We estimate branch lengths in terms of character divergence -- the product of rate and time.
What is important is that we know the relative rates of different types of substitutions, so JC can be expressed:
\begin{table}[htdp]
\begin{center}
\begin{tabular}{cc|cccc}
& & \multicolumn{4}{c}{To State} \\
& & A & C & G & T \\
\hline
\multirow{2}{*}{From } & A &  $-3$ & $1$ & $1$ & $1$    \\
\multirow{2}{*}{State } &C & $1$ & $-3$ & $1$ & $1$    \\
 &G & $1$ & $1$ & $-3$ & $1$ \\
 &T & $1$ & $1$ & $1$ & $-3$ \\
\end{tabular}
\end{center}
\end{table}

\myNewSlide
\section*{JC instantaneous rate matrix yet again}
We estimate branch lengths in terms of expected number of changes {\em per site}.
To do this we standardize the total rate of divergence in the Q matrix and estimate $\nu = \mu t = 3\alpha t$
for each branch.


\begin{table}[htdp]
\begin{center}
\begin{tabular}{cc|cccc}
& & \multicolumn{4}{c}{To State} \\
& & A & C & G & T \\
\hline
\multirow{2}{*}{From } & A &  $-1$ & $\frac{1}{3}$ & $\frac{1}{3}$ & $\frac{1}{3}$    \\
\multirow{2}{*}{State } &C & $\frac{1}{3}$ & $-1$ & $\frac{1}{3}$ & $\frac{1}{3}$    \\
 &G & $\frac{1}{3}$ & $\frac{1}{3}$ & $-1$ & $\frac{1}{3}$    \\
 &T & $\frac{1}{3}$ & $\frac{1}{3}$ & $\frac{1}{3}$ & $-1$    \\
\end{tabular}
\end{center}
\end{table}

\myNewSlide
\section*{\citet{Kimura1980} model or ``the K80 model''}
Transitions and transversions occur at different rates:

\begin{table}[htdp]
\begin{center}
\begin{tabular}{cc|cccc}
& & \multicolumn{4}{c}{To State} \\
& & A & C & G & T \\
\hline
\multirow{2}{*}{From } & A &  $-2\beta - \alpha$ & $\beta $ & $\alpha$ & $\beta $    \\
\multirow{2}{*}{State } &C & $\beta $ & $-2\beta - \alpha$ & $\beta $ & $\alpha$    \\
 &G & $\alpha$ & $\beta $ & $-2\beta - \alpha$ & $\beta $    \\
 &T & $\beta $ & $\alpha$ & $\beta $ & $-2\beta - \alpha$    \\
\end{tabular}
\end{center}
\end{table}

\myNewSlide
\section*{\citet{Kimura1980} model or ``the K80 model''. Reparameterized.}
Once again, we care only about the relative rates, so we can choose one rate to be frame of reference. This turns the 2 parameter model into a 1 parameter form:

\begin{table}[htdp]
\begin{center}
\begin{tabular}{cc|cccc}
& & \multicolumn{4}{c}{To State} \\
& & A & C & G & T \\
\hline
\multirow{2}{*}{From } & A &  $-(2+\kappa)\beta$ & $\beta $ & $\kappa\beta$ & $\beta $    \\
\multirow{2}{*}{State } &C & $\beta $ & $-(2+\kappa)\beta$ & $\beta $ & $\kappa\beta$    \\
 &G & $\kappa\beta$ & $\beta $ & $-(2+\kappa)\beta$ & $\beta $    \\
 &T & $\beta $ & $\kappa\beta$ & $\beta $ & $-(2+\kappa)\beta$    \\
\end{tabular}
\end{center}
\end{table}

\myNewSlide
\section*{\citet{Kimura1980} model or ``the K80 model''. Reparameterized again.}
\begin{table}[htdp]
\begin{center}
\begin{tabular}{cc|cccc}
& & \multicolumn{4}{c}{To State} \\
& & A & C & G & T \\
\hline
\multirow{2}{*}{From } & A &  $-2-\kappa$ & $1 $ & $\kappa$ & $1 $    \\
\multirow{2}{*}{State } &C & $1 $ & $-2-\kappa$ & $1 $ & $\kappa$    \\
 &G & $\kappa$ & $1 $ & $-2 - \kappa$ & $1 $    \\
 &T & $1 $ & $\kappa$ & $1 $ & $-2 - \kappa$    \\
\end{tabular}
\end{center}
\end{table}

\myNewSlide
Kappa is the transititon/transversion rate ratio:
\[\kappa = \frac{\alpha}{\beta}\]
(if $\kappa = 1$ then we are back to JC).

\myNewSlide
What is the instantaneous probability of an  particular transversion?
\begin{eqnarray*}
	\Pr(A\rightarrow C) & = & \Pr(A)\Pr(\mbox{change to } C) \\
& =  & \frac{1}{4}\left(\beta dt\right)
\end{eqnarray*}

\myNewSlide
What is the instantaneous probability of an  particular transition?
\begin{eqnarray*}
	\Pr(A\rightarrow G) & = & \Pr(A)\Pr(\mbox{change to } G) \\
& =  & \frac{1}{4}\left(\kappa\beta dt\right)
\end{eqnarray*}

\myNewSlide
There are four types of transitions:
	\[A\rightarrow G ,G\rightarrow A ,C\rightarrow T, T\rightarrow C \]
and eight types of transversions:
	\[A\rightarrow C ,A\rightarrow T ,G\rightarrow C ,G\rightarrow T ,C\rightarrow A, C\rightarrow G,T\rightarrow A, T\rightarrow G \]
	\[\mbox{Ti/Tv ratio} = \frac{\Pr(\mbox{any transition})}{\Pr(\mbox{any transversion})} = \frac{4\left(\frac{1}{4}\left(\kappa\beta dt\right)\right)}{8\left(\frac{1}{4}\left(\beta dt\right)\right)} = \frac{\kappa}{2}\]

For K2P instantaneous transition/transversion ratio is one-half the instantaneous transition/transversion {\bf rate ratio}

\myNewSlide
\section*{Felsenstein  1981 model or ``F81 model''}
\begin{table}[htdp]
\begin{center}
\begin{tabular}{cc|cccc}
& & \multicolumn{4}{c}{To State} \\
& & A & C & G & T \\
\hline
\multirow{2}{*}{From } & A &  $-$ & $\pi_C $ & $\pi_G$ & $\pi_T $    \\
\multirow{2}{*}{State } &C & $\pi_A$ & $- $ & $\pi_G$ & $\pi_T $    \\
 & G & $\pi_A$ & $\pi_C $ & $-$ & $\pi_T $    \\
 &  T & $\pi_A$ & $\pi_C $ & $\pi_G$ & $- $    \\
\end{tabular}
\end{center}
\end{table}


\myNewSlide
\section*{HKY  1985 model}
\begin{table}[htdp]
\begin{center}
\begin{tabular}{cc|cccc}
& & \multicolumn{4}{c}{To State} \\
& & A & C & G & T \\
\hline
\multirow{2}{*}{From } & A &  $-$ & $\pi_C $ & $\kappa\pi_G$ & $\pi_T $    \\
\multirow{2}{*}{State } &C & $\pi_A$ & $- $ & $\pi_G$ & $\kappa\pi_T $    \\
 & G & $\kappa\pi_A$ & $\pi_C $ & $-$ & $\pi_T $    \\
 &  T & $\pi_A$ & $\kappa\pi_C $ & $\pi_G$ & $- $    \\
\end{tabular}
\end{center}
\end{table}

\myNewSlide
\includepdf[pages={38}]{../nonfreeimages/pol/pol-09-model.pdf} 

\myNewSlide
\section*{General Time Reversible -- GTR model}
\begin{table}[htdp]
\begin{center}
\begin{tabular}{cc|cccc}
& & \multicolumn{4}{c}{To State} \\
& & A & C & G & T \\
\hline
\multirow{2}{*}{From } & A &  $-$ & $a\pi_C $ & $b\pi_G$ & $c\pi_T $    \\
\multirow{2}{*}{State } &C & $a\pi_A$ & $- $ & $d\pi_G$ & $e\pi_T $    \\
 & G & $b\pi_A$ & $d\pi_C $ & $-$ & $f\pi_T $    \\
 &  T & $c\pi_A$ & $e\pi_C $ & $f\pi_G$ & $- $    \\
\end{tabular}
\end{center}
\end{table}
In PAUP, $f=1$ indicating that $G\rightarrow T$ is the reference rate


\myNewSlide


\bibliography{phylo}

\end{document}     




\myNewSlide
\section*{Balanced Minimum Evolution}
\cite{Pauplin2000} showed that you can calculate a tree length from the pairwise distances without calculating branch lengths.
The key is weighting the distances:
\[l = \sum_i^N\sum_{j=i+1}^N w_{ij} d_{ij}\] 
where:
\[w_{ij} = \frac{1}{2^{n(i,j)}}\]
and $n(i,j)$ is the number of nodes on the path from $i$ to $j$.
\myNewSlide
\section*{Balanced Minimum Evolution}
 \citet{DesperG2002, DesperG2004} called minimizing this estimate of the tree length Balanced Minimum Evolution
 and showed that it is equivalent to a form of weighted least squares in which distances are down-weighted by an exponential function of the topological distances between the leaves.

Desper and Gascuel (2005) showed that NJ is star decomposition under BME. See \citet{GascuelS2006}.

\myNewSlide
\section*{NJ review}
Neighbor-joining is a quick $O(N^3)$ algorithm for estimating the balanced minimum evolution tree.

Performance (based on simulation studies) is often close to the performance of searches under minimum evolution.

Branch length estimates in neighbor joining do not take into account the higher variance associated with large dissimilarity measurements.
Thus the method tends to be sensitive to having long distances in the input data.

\myNewSlide
\section*{Distance-based phylogenetics review}
Least squares and minimum evolution try to find trees and branch lengths such that the path length (or ``patristic distance'') between all taxa is as close as possible to the dissimilarity that is based on your data.

Frequently the observed $p$-distances must be corrected for multiple hits using models of character evolution.

Even when using corrected distances, distance-based approaches suffer from not adequately enforcing all of the natural constraints on the processes that generate the data.


\myNewSlide
\section*{Weaknesses in distance-based approaches}
The fact that different errors in the pairwise distances is \citep[the generalized least-square approach outlined by][]{Bulmer1991} is computationally expensive.

Pairwise distance estimates, even when corrected, do not benefit from insights about which positions are evolving fast and which are slow (or other higher level patterns of sequence evolution).

Reconstructed tree is not forced to be compatible with reasonable constraints on sequence evolution.

\myNewSlide
\section*{Farris' example}
\begin{table}[htdp]
\begin{center}
\begin{tabular}{|c|c|}
\hline
Taxon & Character State\\
\hline
1 & A \\
2 & C \\
3 & G \\ 
4 & T \\
\hline
\end{tabular}
\end{center}
\label{default}
\end{table}%

\myNewSlide
\section*{Another example}
\begin{table}[htdp]
\begin{center}
\begin{tabular}{|c|ccc|}
\hline
Taxon & & & \\
\hline
A & {\tt 1000000} & {\tt\color{red} 11111} & {\tt\color{darkgreen} 00000} \\
B & {\tt 0100000} & {\tt\color{red} 11111} & {\tt\color{darkgreen} 11111} \\
C & {\tt 0010000} & {\tt\color{red} 01111} & {\tt\color{darkgreen} 00000} \\
D & {\tt 0001000} & {\tt\color{red} 00011} & {\tt\color{darkgreen} 00000} \\
E & {\tt 0000100} & {\tt\color{red} 00001} & {\tt\color{darkgreen} 10000} \\
F & {\tt 0000010} & {\tt\color{red} 00000} & {\tt\color{darkgreen} 01100} \\
G & {\tt 0000001} & {\tt\color{red} 00000} & {\tt\color{darkgreen} 00011} \\
\hline
\end{tabular}
\end{center}
\label{default}
\end{table}%
If we consider just the characters in black and red, then we'd have a homoplasy-free data matrix.
The characters in green display character-conflict (they are incompatible with some of the characters in green and all of the read characters).
Homework: What tree does ordinary (unweighted) least squares on the uncorrected total distance favor?
What tree does parsimony favor? Perform a MP reconstruction of characters on the OLS tree. Which branches look suspicious according to parsimony? (you can use PAUP.  I'll post a NEXUS file).
