\documentclass[landscape]{foils} 
\input{../common-preamble-start}
\input{../preamble.tex}
\usepackage{url}
\usepackage{hyperref}
\hypersetup{backref,  linkcolor=blue, citecolor=red, colorlinks=true, hyperindex=true}

\begin{document}
\pagecolor{white}


\section*{Homework 6 (due Mon, Oct 11)}
\normalsize
Print out and fill in the next page.
\begin{compactenum}
	\item likelihood calculations
\begin{compactenum}
	\item As data, use a character in which  {\tt A} is the state for taxa 1 and 2 , taxon 3 has state {\tt G}, and taxon 4 has a {\tt C}
	\item Fill in the circle with character state (either the observed data for tips or the inferred states)
	\item There are 16 trees on each page, use them to show all 16 possible ancestral character state combinations.
	\item Calculate the probability of each reconstruction under the Kimura model.
	\item Use $\kappa=2$
	\item For branch lengths, use {\tt(1:0.05,3:0.4,(2:0.05,4:0.4):0.05)}
\end{compactenum}
	\item Which branch length has the higher ratio of the transition probability to transversion probability?
\end{compactenum}
\myNewSlide
I recommend, using a spreadsheet to:
\begin{compactenum}
	\item  calculate all three change probabilities (no change, transition, transversion) for the 0.05 branch length.
	\item  calculate all three change probabilities (no change, transition, transversion) for the 0.4 branch length
	\item  calculate multiply the appropriate numbers.
\end{compactenum}
If you label the cells of the spreadsheet logically, I'll be able to figure out your system.
\includepdf[pages={2}]{../images/parshomework4taxontrees.pdf}
\end{document}
