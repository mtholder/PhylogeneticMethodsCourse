\documentclass[11pt]{article}
%\usepackage{graphicx}
%\DeclareGraphicsRule{.tif}{png}{.png}{`convert #1 `dirname #1`/`basename #1 .tif`.png}
\textwidth = 6.5 in
\textheight = 9 in
\oddsidemargin = 0.0 in
\evensidemargin = 0.0 in
\topmargin = 0.0 in
\headheight = 0.0 in
\headsep = 0.0 in
\parskip = 0.2in
\parindent = 0.0in
\pagestyle{empty}
\usepackage{paralist} %compactenum

%\newtheorem{theorem}{Theorem}
%\newtheorem{corollary}[theorem]{Corollary}
%\newtheorem{definition}{Definition}
\usepackage{tipa}
\usepackage{amsfonts}
\usepackage[mathscr]{eucal}

\newcommand{\exampleMacro}[1]{\mu_{#1}}

\usepackage{url}
\usepackage{hyperref}
\hypersetup{backref,  pdfpagemode=FullScreen,  linkcolor=blue, citecolor=red, colorlinks=true, hyperindex=true}

\begin{document}
\section*{Homework 2 -- Due Wed.~Sept.~1}
{
\Large
Find the tree that displays the greatest number of splits ({\em not} the set of most strongly supported splits) from the table shown below.
Write the tree in Newick notation with the `Freq' shown as the branch length.
\begin{center}
{\tt 
\begin{tabular}{|lp{0.1cm}r|}
\hline
{\tt 123456789} & & Freq \\
\hline
{\tt ...*..*..} & & 60 \\
{\tt *......*.} & & 55 \\
{\tt *..*..**.} & & 45 \\
{\tt *..***.**} & & 21 \\
{\tt .****.*..} & & 15 \\
\hline
\end{tabular}
}
\end{center}
The recommended route to a solution is to:
\begin{compactenum}
	\item Construct a split compatibility graph (the nodes represent splits and you put an edge to connect each pair of nodes if those nodes are compatible).
	\item Identify the largest clique in the graph (the largest subgraph in which each node is adjacent to every other node).
	\item Construct the tree that contains the splits that are represented by the nodes in the maximal clique.
	\item Label the edges of the tree with the frequency value that corresponds to the split that the edge maps to.
\end{compactenum}
}
\end{document}
