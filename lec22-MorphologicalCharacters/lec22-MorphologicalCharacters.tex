\documentclass[landscape]{foils} 
%\newif\ifpdf
%\ifx\pdfoutput\undefined
%\pdffalse % we are not running PDFLaTeX
%\else
%\pdfoutput=1 % we are running PDFLaTeX
%\pdftrue
%\fi

%\ifpdf
\usepackage[pdftex]{graphicx}
%\else
%\usepackage{graphicx}
%\fi

%\ifpdf
\DeclareGraphicsExtensions{.pdf, .jpg, .tif, .png}
%\else
%\DeclareGraphicsExtensions{.eps, .jpg}
%\fi

%\usepackage{pslatex}
\usepackage{tabularx,dcolumn, graphicx, amsfonts,amsmath}  
\usepackage[sectionbib]{natbib}
\bibliographystyle{apalike}
\usepackage{picinpar}
\usepackage{multirow}
\usepackage{rotating}
\usepackage{paralist} %compactenum
\setlength{\voffset}{-0.5in}
%\setlength{\hoffset}{-0.5in}
%\setlength{\textwidth}{10.5in}
\setlength{\textheight}{7in}
\setlength{\parindent}{0pt}
%\pagestyle{empty}
%\renewcommand{\baselinestretch}{2.0}
\DeclareMathSymbol{\expect}{\mathalpha}{AMSb}{'105}
\def\p{\rm p}
\def\pp{\rm P}
% this are commands that come with the color package
\usepackage{color}
\usepackage{fancyhdr}


\pagestyle{empty}
%define colors
\definecolor{mediumblue}{rgb}{0.0509,0.35,0.568}
\definecolor{blue}{rgb}{0.0109,0.15,0.468}
\definecolor{black}{rgb}{0.04,0.06,0.2}
\definecolor{darkblue}{rgb}{0.03,0.1,0.2}
\definecolor{darkgreen}{rgb}{0.03,0.5,0.2}
\definecolor{lightblue}{rgb}{0.85,0.9333,0.95}
\definecolor{lightblue2}{rgb}{0.270588, 0.45098, 0.701961}
\definecolor{white}{rgb}{1.0,1.0,1.0}
\definecolor{yellow}{rgb}{0.961,0.972,0.047}
\definecolor{red}{rgb}{0.9,0.1,0.1}
\definecolor{orange}{rgb}{1.0,0.4,0.0}
\definecolor{grey}{rgb}{0.5,0.5,0.5}
\definecolor{violet}{rgb}{0.619608, 0.286275, 0.631373}
\definecolor{mybackgroundcolor}{rgb}{1.0,1.0,1.0}

%\definecolor{light}{rgb}{.5,0.5,0.0}
\definecolor{light}{rgb}{.3,0.3,0.3}

% sets backgroundcolor for whole document 
\pagecolor{mybackgroundcolor}
% sets text color
%\color{black}
% see below for an example how to change just a few words
% using \textcolor{color}{text}

\font \courier=pcrb scaled 2000
\newcommand{\notetoself}[1]{{\textsf{\textsc{\color{red} #1}}}\\}

\newcommand{\answer}[1]{{\sf \color{red} #1}}

\usepackage{pdfpages}

\newcommand{\section}{\secdef \newsection\newsection}
%\renewcommand{\labelitemi}{\includegraphics[width=5mm]{images/bullet.pdf}}
\newcommand{\newsection}[1]{%
{
	\par\flushleft\large\sf\bfseries \vskip -2cm #1\\\rule[0.7\baselineskip]{\textwidth}{0.5mm}\par}}

\newcommand{\subsection}{\secdef \test\test}
\newcommand{\test}[1]{%
	{\par\flushleft\normalsize\sf\bfseries #1: }}
\newcommand{\M}{\mathcal{M}}
\newcommand{\prob}{{\rm Prob~}}
\def\showy#1{{\normalsize\sf\bfseries #1}}
\def\donotuse#1{}

\newcommand{\entrylabel}[1]{\mbox{#1}\hfil}
\newenvironment{entry}
	{\begin{list}{}%
		{\renewcommand{\makelabel}{\entrylabel}%
		\setlength{\labelwidth}{35pt}%
		\setlength{\leftmargin}{\labelwidth+\labelsep}%
	}%
	{\end{list}}}

\newcommand{\poltext}{{\copyright\ 2002--2010 by Paul O. Lewis -- Modified by  Mark Holder with permission from Paul Lewis}}

\newcommand{\pol}{{\footnotesize \poltext}}
\newcommand{\myBackground}{\begin{picture}(0,0)(0,0)  \put(-40,-70){\makebox(0,0)[l]{\includegraphics[width=33cm]{images/baby_blue.jpg}}} \end{picture}}
\newcommand{\myFooter}{}
%\begin{picture}(0,0)(0,0)
%	\put(0,-185){\pol}
%\end{picture}}
\newcommand{\myNewSlide}{\newpage\myFooter} % \myBackground}

\usepackage{url}
\usepackage{hyperref}
\hypersetup{backref,  linkcolor=blue, citecolor=red, colorlinks=true, hyperindex=true}

\usepackage{pdfpages}
\usepackage{bm}
\usepackage{ifsym}
\newcommand{\hilite}[1]{{\color{red} \bf #1}}
\newcommand{\disruption}{\theta}
\begin{document}
\pagecolor{white}
\unitlength=1mm
\myNewSlide
\Large
\begin{enumerate}
	\item Can we use the CFN model for morphological traits?
	\item Can we use something like the GTR model for morphological traits?
	\item Stochastic Dollo.
	\item Continuous characters.
\end{enumerate}

\myNewSlide
\section*{M$k$ models}
$k$-state variants of the Jukes-Cantor model -- all rates equal.

$$\Pr(i\rightarrow i|\nu) = \frac{1}{k} + \left(\frac{k-1}{k}\right)e^{-\left(\frac{k}{k-1}\right)\nu}$$\\
$$\Pr(i\rightarrow j|\nu) = \frac{1}{k} - \left(\frac{1}{k}\right)e^{-\left(\frac{k}{k-1}\right)\nu} $$

\myNewSlide
\section*{Sampling morphological characters}
Using our models assumes that our characters can be thought of as having been a random sample from a universe of $iid$ characters.

\begin{compactenum}
	\item We never have constant morphological characters.
	\begin{compactenum}
		\item There are plenty of attributes that do not vary.
		\item The ``rules'' of coding morphological characters are well-defined.
		\item How many constant characters ``belong'' in our matrix?
	\end{compactenum}
\end{compactenum}

\myNewSlide
\section*{Solutions to the lack of constant characters}

\begin{compactenum}
	\item Score our taxa for a random selection of characters -- {\em not} a selection of characters that are chosen because they are appropriate for our group.  (Is this possible or desirable?)
	\item Account for the fact that our data is filtered.
\end{compactenum}

\myNewSlide
\section*{M$k_v$ model}
Introduced by \citet{Lewis2001} using a trick Felsenstein used for restriction site data.

We condition our inference on the fact that we know that (by design) our characters are variable.

If ${\mathcal V}$ is the set of variable data patterns, then we do inference on:

$$ \Pr(x_i|T,\nu, x_i\in {\mathcal V}) $$
rather than:
$$ \Pr(x_i|T,\nu) $$

\myNewSlide
\section*{Conditional likelihood}

If $x_i\in {\mathcal V}$, then:
$$  \Pr(x_i|T,\nu, x_i\in {\mathcal V}) \Pr(x_i\in {\mathcal V} |T,\nu) = \Pr(x_i|T,\nu)  $$

So:
$$ \Pr(x_i|T,\nu, x_i\in {\mathcal V}) = \frac{\Pr(x_i|T,\nu)}{\Pr(x_i\in {\mathcal V} |T,\nu)}$$


\myNewSlide
Note that:
$$\Pr(x_i\in {\mathcal V} |T,\nu) = 1 - \Pr(x_i\notin {\mathcal V} |T,\nu)$$

If $ {\mathcal C}$ is the set of constant data patterns:
$$x_i\notin {\mathcal V} \equiv x_i\in {\mathcal C}$$
So:
$$\Pr(x_i\in {\mathcal V} |T,\nu) = 1 - \Pr(x_i\in {\mathcal C} |T,\nu)$$
There are not that many constant patterns, so we can just calculate the likelihood for each one of them.

\myNewSlide
\section*{Inference under M$2_v$}
\begin{compactenum}
	\item Calculate $\Pr(x_i|T,\nu)$ for each site $i$
	\item Calculate $$\Pr(x\in{\mathcal C}|T,\nu) = \Pr(000\ldots0 |T,\nu) + \Pr(111\ldots1 |T,\nu)$$
	\item For each site, calculate:
		$$ \Pr(x_i|T,\nu, x_i\in{\mathcal V}) = \frac{\Pr(x_i|T,\nu)}{1 - \Pr(x\in{\mathcal C}|T,\nu)} $$
	\item Take the product of $\Pr(x_i|T,\nu, x_i\in{\mathcal V})$ over all characters.
\end{compactenum}

\myNewSlide
\section*{M$k_v$ and M$k_{pars-inf}$}
The following were proved by \citet{AllmanHR2010}
\begin{compactenum}
	\item M$k_v$ is a consistent estimator of the tree and branch lengths,
	\item If you filter your data to only contain parsimony-informative charecters:
		\begin{compactenum}
			\item A four-leaf tree cannot be identified!
			\item Trees of eight or more leaves can be identified using inference under M$k_{pars-inf}$
		\end{compactenum}
\end{compactenum}

\myNewSlide
\section*{Can we estimate biases in state-transitions and state frequencies from morphological data?}

\myNewSlide
\section*{Can we estimate biases in state-transitions and state frequencies from morphological data?}
Of course! (remember Pagel's model, which we have already encountered).

But we have to bear in mind that $0$ in one character has nothing to do with $0$ in another.

This means that we have to use character-specific parameters or mixtures models (to reduce the number of parameters).  Typically this is done in a Bayesian setting.

\myNewSlide
\section*{Other tidbits about likelihood modeling of non-molecular data}
\large
\begin{compactenum}
	\item We can use the No-common-mechanism model \citep{TuffleyS1997} to generate a likelihood score from a parsimony score (for combined analyses).
	\item By setting some rates to 0 we can test transformation assumptions about irreversibility.
	\item Modification to the pruning algorithm lead to models of Dollo's law (no independent gain of a character state). For further details, see \citet{AlekseyenkoLS2008}.
	\item The use of ontologies to describe characters may revolutionize modeling of morphological data and the prospects for constructing ``morphological super-matrices''
\end{compactenum}


\myNewSlide
\bibliography{phylo}

\end{document}