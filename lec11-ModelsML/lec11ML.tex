\documentclass[landscape]{foils} 
\input{../common-preamble-start}
\input{../preamble.tex}
\usepackage{url}
\usepackage{hyperref}
\hypersetup{backref,  linkcolor=blue, citecolor=red, colorlinks=true, hyperindex=true}

\usepackage{pdfpages}
\usepackage{bm}

\newcommand{\disruption}{\theta}
\begin{document}
\pagecolor{white}
\unitlength=1mm
\begin{center}
{\Large Some of these slides have been borrowed from Dr.\ Paul Lewis, Dr.\ Joe Felsenstein. Thanks!}
\vskip 15mm
\large Paul has many great tools for teaching phylogenetics at his web site: \\
\url{http://hydrodictyon.eeb.uconn.edu/people/plewis}
\end{center}


\myNewSlide
\section*{JC instantaneous rate matrix yet again}
\Large
We estimate branch lengths in terms of expected number of changes {\em per site}.
To do this we standardize the total rate of divergence in the Q matrix and estimate $\nu = \mu t = 3\alpha t$
for each branch.


\begin{table}[htdp]
\begin{center}
\begin{tabular}{cc|cccc}
& & \multicolumn{4}{c}{To State} \\
& & A & C & G & T \\
\hline
\multirow{2}{*}{From } & A &  $-1$ & $\frac{1}{3}$ & $\frac{1}{3}$ & $\frac{1}{3}$    \\
\multirow{2}{*}{State } &C & $\frac{1}{3}$ & $-1$ & $\frac{1}{3}$ & $\frac{1}{3}$    \\
 &G & $\frac{1}{3}$ & $\frac{1}{3}$ & $-1$ & $\frac{1}{3}$    \\
 &T & $\frac{1}{3}$ & $\frac{1}{3}$ & $\frac{1}{3}$ & $-1$    \\
\end{tabular}
\end{center}
\end{table}

\myNewSlide
\section*{JC ``state-comparison'' probabilities}
$i$ and $j$ refer to states (A, C, G, T). $i\neq j$:
$$\Pr(i\rightarrow i|\nu) = \frac{1}{4} + \frac{3}{4}e^{\frac{-4\nu}{3}}$$\\
$$\Pr(i\rightarrow j|\nu) = \frac{1}{4} - \frac{1}{4}e^{\frac{-4\nu}{3}}$$

\myNewSlide
\section*{CFN transition probabilities}
$$\Pr(0\rightarrow 0|\nu) = \Pr(1\rightarrow 1|\nu) = \frac{1}{2} + \frac{1}{2}e^{-2\nu}$$\\
$$\Pr(0\rightarrow 1|\nu) = \Pr(1\rightarrow 0|\nu) = \frac{1}{2} - \frac{1}{2}e^{-2\nu}$$

\myNewSlide
\section*{``M$k$'' transition probabilities}
$k$-state version of the one-rate model.
\Large
$$\Pr(i\rightarrow i|\nu) = \frac{1}{k} + \frac{(k-1)e^{-\left(\frac{k}{k-1}\right)\nu}}{k}$$\\
$$\Pr(i\rightarrow j|\nu) = \frac{1}{k} - \frac{e^{-\left(\frac{k}{k-1}\right)\nu}}{k} $$




\myNewSlide
\section*{\citet{Kimura1980} model or ``the K80 model''}

Transitions and transversions occur at different rates:

\begin{table}[htdp]
\begin{center}
\begin{tabular}{cc|cccc}
& & \multicolumn{4}{c}{To State} \\
& & A & C & G & T \\
\hline
\multirow{2}{*}{From } & A &  $-2\beta - \alpha$ & $\beta $ & $\alpha$ & $\beta $    \\
\multirow{2}{*}{State } &C & $\beta $ & $-2\beta - \alpha$ & $\beta $ & $\alpha$    \\
 &G & $\alpha$ & $\beta $ & $-2\beta - \alpha$ & $\beta $    \\
 &T & $\beta $ & $\alpha$ & $\beta $ & $-2\beta - \alpha$    \\
\end{tabular}
\end{center}
\end{table}

\myNewSlide
\section*{\citet{Kimura1980} model or ``the K80 model''. Reparameterized.}
\large
We only care about the {\em relative} rates, so we can choose one rate to be frame of reference. This turns the 2 parameter model into a 1 parameter form:

\begin{table}[htdp]
\begin{center}
\begin{tabular}{cc|cccc}
& & \multicolumn{4}{c}{To State} \\
& & A & C & G & T \\
\hline
\multirow{2}{*}{From } & A &  $-(2+\kappa)\beta$ & $\beta $ & $\kappa\beta$ & $\beta $    \\
\multirow{2}{*}{State } &C & $\beta $ & $-(2+\kappa)\beta$ & $\beta $ & $\kappa\beta$    \\
 &G & $\kappa\beta$ & $\beta $ & $-(2+\kappa)\beta$ & $\beta $    \\
 &T & $\beta $ & $\kappa\beta$ & $\beta $ & $-(2+\kappa)\beta$    \\
\end{tabular}
\end{center}
\end{table}

\myNewSlide
\section*{\citet{Kimura1980} model or ``the K80 model''. Reparameterized again.}
\Large
\begin{table}[htdp]
\begin{center}
\begin{tabular}{cc|cccc}
& & \multicolumn{4}{c}{To State} \\
& & A & C & G & T \\
\hline
\multirow{2}{*}{From } & A &  $-2-\kappa$ & $1 $ & $\kappa$ & $1 $    \\
\multirow{2}{*}{State } &C & $1 $ & $-2-\kappa$ & $1 $ & $\kappa$    \\
 &G & $\kappa$ & $1 $ & $-2 - \kappa$ & $1 $    \\
 &T & $1 $ & $\kappa$ & $1 $ & $-2 - \kappa$    \\
\end{tabular}
\end{center}
\end{table}

\myNewSlide
Kappa is the transititon/transversion rate ratio:
\[\kappa = \frac{\alpha}{\beta}\]
(if $\kappa = 1$ then we are back to JC).

\myNewSlide
What is the instantaneous probability of an  particular transversion?
\begin{eqnarray*}
	\Pr(A\rightarrow C) & = & \Pr(A)\Pr(\mbox{change to } C) \\
& =  & \frac{1}{4}\left(\beta dt\right)
\end{eqnarray*}

\myNewSlide
What is the instantaneous probability of an  particular transition?
\begin{eqnarray*}
	\Pr(A\rightarrow G) & = & \Pr(A)\Pr(\mbox{change to } G) \\
& =  & \frac{1}{4}\left(\kappa\beta dt\right)
\end{eqnarray*}

\myNewSlide
\large
There are four types of transitions:
	\[A\rightarrow G ,G\rightarrow A ,C\rightarrow T, T\rightarrow C \]
and eight types of transversions:
	\[A\rightarrow C ,A\rightarrow T ,G\rightarrow C ,G\rightarrow T ,C\rightarrow A, C\rightarrow G,T\rightarrow A, T\rightarrow G \]
	\[\mbox{Ti/Tv ratio} = \frac{\Pr(\mbox{any transition})}{\Pr(\mbox{any transversion})} = \frac{4\left(\frac{1}{4}\left(\kappa\beta dt\right)\right)}{8\left(\frac{1}{4}\left(\beta dt\right)\right)} = \frac{\kappa}{2}\]

For K2P instantaneous transition/transversion ratio is one-half the instantaneous transition/transversion {\bf rate ratio}

\myNewSlide
\section*{Kimura model change probabilities}
\Large
$$\Pr(A\rightarrow A|\nu) = \frac{1}{4}\left(1 + e^{-\left(\frac{4}{2+\kappa}\right)\nu} + 2e^{-\left(\frac{2+2\kappa}{2+\kappa}\right)\nu}\right)$$\\
$$\Pr(A\rightarrow G|\nu) = \frac{1}{4}\left(1 + e^{-\left(\frac{4}{2+\kappa}\right)\nu} - 2e^{-\left(\frac{2+2\kappa}{2+\kappa}\right)\nu}\right)$$\\
$$\Pr(A\rightarrow C|\nu) = \frac{1}{4}\left(1 - e^{-\left(\frac{4}{2+\kappa}\right)\nu}\right)$$

\myNewSlide
\section*{Felsenstein  1981 model or ``F81 model''}
\begin{table}[htdp]
\begin{center}
\begin{tabular}{cc|cccc}
& & \multicolumn{4}{c}{To State} \\
& & A & C & G & T \\
\hline
\multirow{2}{*}{From } & A &  $-$ & $\pi_C $ & $\pi_G$ & $\pi_T $    \\
\multirow{2}{*}{State } &C & $\pi_A$ & $- $ & $\pi_G$ & $\pi_T $    \\
 & G & $\pi_A$ & $\pi_C $ & $-$ & $\pi_T $    \\
 &  T & $\pi_A$ & $\pi_C $ & $\pi_G$ & $- $    \\
\end{tabular}
\end{center}
\end{table}


\myNewSlide
\section*{HKY  1985 model}
\begin{table}[htdp]
\begin{center}
\begin{tabular}{cc|cccc}
& & \multicolumn{4}{c}{To State} \\
& & A & C & G & T \\
\hline
\multirow{2}{*}{From } & A &  $-$ & $\pi_C $ & $\kappa\pi_G$ & $\pi_T $    \\
\multirow{2}{*}{State } &C & $\pi_A$ & $- $ & $\pi_G$ & $\kappa\pi_T $    \\
 & G & $\kappa\pi_A$ & $\pi_C $ & $-$ & $\pi_T $    \\
 &  T & $\pi_A$ & $\kappa\pi_C $ & $\pi_G$ & $- $    \\
\end{tabular}
\end{center}
\end{table}

\myNewSlide
\includepdf[pages={38}]{../nonfreeimages/pol/pol-09-model.pdf} 

\myNewSlide
\section*{General Time Reversible -- GTR model}
\begin{table}[htdp]
\begin{center}
\begin{tabular}{cc|cccc}
& & \multicolumn{4}{c}{To State} \\
& & A & C & G & T \\
\hline
\multirow{2}{*}{From } & A &  $-$ & $a\pi_C $ & $b\pi_G$ & $c\pi_T $    \\
\multirow{2}{*}{State } &C & $a\pi_A$ & $- $ & $d\pi_G$ & $e\pi_T $    \\
 & G & $b\pi_A$ & $d\pi_C $ & $-$ & $f\pi_T $    \\
 &  T & $c\pi_A$ & $e\pi_C $ & $f\pi_G$ & $- $    \\
\end{tabular}
\end{center}
\end{table}
In PAUP, $f=1$ indicating that $G\rightarrow T$ is the reference rate



\myNewSlide
\includepdf[pages={4}]{../nonfreeimages/pol/pol-ML.pdf} 

\myNewSlide
\section*{How can we calculate the likelihood score}
\Large 
Under the JC (or K2P) model:
\begin{eqnarray*}
	\ln L & =  &  12 \ln\pi_A + 7 \ln\pi_C + 7 \ln\pi_G + 6 \ln\pi_T \\
		 & =  &  12 \ln0.25 + 7 \ln0.25 + 7 \ln0.25 + 6 \ln0.25 \\
		 & = & -44.361
\end{eqnarray*}

\myNewSlide
\section*{How can we calculate the likelihood score}
\Large 
Under the F81 (or HKY or GTR) model:
\begin{eqnarray*}
	\ln L & =  &  12 \ln\pi_A + 7 \ln\pi_C + 7 \ln\pi_G + 6 \ln\pi_T \\
\end{eqnarray*}
But what are the values for the parameters: $\pi_A, \pi_C, \pi_G, \pi_T$ ?

In many cases we refer to these parameters as ``nuisance parameters.''
They must be specified in order to calculate the likelihood, but we are not interested in them by themselves.


\myNewSlide
\section*{ML parameter estimates}
\large
We can find the maximum likelihood estimates of the parameters to give us the ML score: the maximum likelihood obtainable under this model:
\begin{eqnarray*}
	\ln L & =  &  12 \ln\pi_A + 7 \ln\pi_C + 7 \ln\pi_G + 6 \ln\pi_T \\
	 & =  &  12 \ln\widehat{\pi_A} + 7 \ln\widehat{\pi_C} + 7 \ln\widehat{\pi_G} + 6 \ln\widehat{\pi_T} \\
	 & =  &  12 \ln0.375 + 7 \ln0.21875 + 7 \ln0.21875+ 6 \ln0.1875 \\
	 & =  & -43.091 
\end{eqnarray*}
But how did I get the numbers to fill for the parameters?
How do we know that $\widehat{\pi_A} = 0.375$ and $\widehat{\pi_C} = 0.21875$

\myNewSlide
\section*{ML parameter estimates}
We might guess that:
\begin{eqnarray*}
	\widehat{\pi_A} & = &  \frac{12}{32} = 0.375\\
	\widehat{\pi_C} & = &  \frac{7}{32} = 0.21875\\
	\widehat{\pi_G} & = &  \frac{7}{32} = 0.21875\\
	\widehat{\pi_T} & = &  \frac{6}{32} = 0.1875
\end{eqnarray*}
but how do we prove it?

\myNewSlide
\section*{ML parameter estimates}
For simple problems we solve for the point in parameter space for which derivatives with respect to all parameters are 0 (we also have to consider boundary points).

We would have to do constrained optimization because \[\pi_A + \pi_C + \pi_G + \pi_T = 1\]
and that is a pain.

\myNewSlide
\section*{ML parameter estimates}
\normalsize
We can reparameterize:
\begin{eqnarray*}
	r & = & {\pi_A+\pi_G} \\
   a  & = & \frac{\pi_A}{\pi_A+\pi_G} \\
   c & = & \frac{\pi_C}{\pi_C+\pi_T}
\end{eqnarray*}
and always recover the original parameters:
\begin{eqnarray*}
	\pi_A & = & ra \\
	\pi_G & = & r(1-a) \\
	\pi_C & = & (1-r)c \\
	\pi_T & = & (1-r)(1-c) \\
\end{eqnarray*}

\myNewSlide
\section*{ML parameter estimates}
\begin{eqnarray*}
	\ln L & = & 12 \ln\pi_A + 7 \ln\pi_C + 7 \ln\pi_G + 6 \ln\pi_T \\
	& = & 12 \ln\left[ra\right] + 7 \ln\left[(1-r)c\right] + 7 \ln\left[r(1-a)\right] + 6 \ln\left[(1-r)(1-c)\right] 
\end{eqnarray*}
Recall that:
\begin{eqnarray*}
	\frac{\partial \ln f(x)}{\partial x} =\frac{\frac{\partial f(x)}{\partial x}}{f(x)}
\end{eqnarray*}


\myNewSlide
\begin{eqnarray*}
	\ln L & = & 12 \ln\left[ra\right] + 7 \ln\left[(1-r)c\right] + 7 \ln\left[r(1-a)\right] + 6 \ln\left[(1-r)(1-c)\right]  \\
	\frac{\partial \ln L}{\partial a} & = &\frac{12 r}{ra} + \frac{7(-r)}{r(1-a)} \\
	& = & \frac{12}{a} -\frac{7}{(1-a)}\\
	 0 & = & \frac{12}{\hat{a}} -\frac{7}{(1-\hat{a})}  \\
	 \hat{a} & = & \frac{12}{19}
\end{eqnarray*}

\myNewSlide
\begin{eqnarray*}
	\ln L & = & 12 \ln\left[ra\right] + 7 \ln\left[(1-r)c\right] + 7 \ln\left[r(1-a)\right] + 6 \ln\left[(1-r)(1-c)\right]  \\
	\frac{\partial \ln L}{\partial c} & = &\frac{7 (1-r)}{(1-r)c} + \frac{6-(1-r)}{(1-r)(1-a)} \\
	& = & \frac{7}{c} -\frac{6}{(1-c)}\\
	 0 & = & \frac{7}{\hat{c}} -\frac{6}{(1-\hat{c})}  \\
	 \hat{c} & = & \frac{7}{13}
\end{eqnarray*}

\myNewSlide
\begin{eqnarray*}
	\ln L & = & 12 \ln\left[ra\right] + 7 \ln\left[(1-r)c\right] + 7 \ln\left[r(1-a)\right] + 6 \ln\left[(1-r)(1-c)\right]  \\
	\frac{\partial \ln L}{\partial r} & = &\frac{12 a}{ra} + \frac{7(-c)}{(1-r)c} + \frac{7(1-a)}{r(1-a)} + \frac{6(-(1-c))}{(1-r)(1-c)} \\
	& = & \frac{12}{r} -\frac{7}{(1-r)} + \frac{7}{r} - \frac{6}{1-r}\\
	& = & \frac{19}{r} -\frac{13}{(1-r)}\\
	 0 & = & \frac{19}{\hat{r}} -\frac{13}{(1-\hat{r})}  \\
	 \hat{r} & = & \frac{19}{32}
\end{eqnarray*}


\myNewSlide
ML inference displays ``scale invariance'' so we can just transform the ML estimates into our original parameters:
\begin{eqnarray*}
	\widehat{\pi_A} = \hat{r}\hat{a}  & = & \left(\frac{19}{32}\right)\left(\frac{12}{19}\right) = \frac{12}{32} \\
	\widehat{\pi_G}  = \hat{r}(1-\hat{a})& = & \left(\frac{19}{32}\right)\left(\frac{7}{19}\right) = \frac{7}{32} \\
	\widehat{\pi_C}  =  (1-\hat{r})\hat{c} & = & \left(\frac{13}{32}\right)\left(\frac{6}{13}\right) = \frac{7}{32} \\
	\widehat{\pi_T}  =  (1-\hat{r})(1-\hat{c})& = & \left(\frac{13}{32}\right)\left(\frac{6}{13}\right) = \frac{6}{32} \\
\end{eqnarray*}

\myNewSlide
\section*{Likelihood ratio testing}
\large
$$\ln L_{JC} = -44.361 $$
$$\ln L_{F81} = -43.091 $$
But the F81 model has 3 more free parameters than JC.

The likelihood ratio test is a hypothesis testing approach to model selection.


\myNewSlide
\section*{Likelihood ratio testing}
$H_0$: the data were generated under the simpler model\\
$H_A$: the data were generated under the more complex model. \par

test statistic: $2\left(\ln L_{\mbox{complex}} - \ln L_{\mbox{simple}}\right)$

The LRT only works if the simple model is nested inside the more complex model (if the free parameters for the simple model are a subset of the free parameters for the more complex model).

\myNewSlide
\section*{Likelihood ratio testing}
Null distribution: $\chi^2$ distribution with \\
$d.f.$ = difference in the number of free parameters.

If the LR test statistic is $>$ than the critical value from the appropriate chi-square table, then we reject the simple model and prefer the more complex model.

\myNewSlide
\section*{Likelihood ratio testing - example}
\large
$$\ln L_{JC} = -44.361 $$
$$\ln L_{F81} = -43.091 $$
$$LRT = 2(-43.091 - (-44.361)) = 2.54$$
$$df = 3 - 0 = 3$$
$$\chi^2_3(\mbox{critical}, P = 0.05) = 7.815$$
Not significant. Do not reject the JC model.

(if we look up the $P$-value for this test statistic it is 0.4681).

\myNewSlide
\section*{Likelihoods on the simplest possible tree}
\begin{center}
{\Huge
\tt GA$\rightarrow$GG}
\end{center}
\large
\begin{eqnarray*}
	L & = & L_1 L_2 \\
	 & = & \Pr(G)\Pr(G\rightarrow G) \Pr(A)\Pr(A\rightarrow G)\\
	 & = & \Pr(G)\Pr(G\rightarrow G|\nu) \Pr(A)\Pr(A\rightarrow G|\nu) \\
	 & = & \left(\frac{1}{4}\right)\left(\frac{1}{4} + \frac{3}{4}\;e^{\frac{-4\nu}{3}}\right)\left(\frac{1}{4}\right)\left(\frac{1}{4}-\frac{1}{4}\;e^{\frac{-4\nu}{3}}\right)
\end{eqnarray*}

\myNewSlide
\normalsize
\begin{eqnarray*}
	d & = & \frac{1}{4}-\frac{1}{4}\;e^{\frac{-4\nu}{3}} \\
	\left(\frac{1}{4} + \frac{3}{4}\;e^{\frac{-4\nu}{3}}\right) & = & 1 -3d \\
	L & = & \left(\frac{1}{4}\right)\left(\frac{1}{4} + \frac{3}{4}\;e^{\frac{-4\nu}{3}}\right)\left(\frac{1}{4}\right)\left(\frac{1}{4}-\frac{1}{4}\;e^{\frac{-4\nu}{3}}\right) \\
	& = & \frac{(1-3d)d}{16} \\
	\frac{\partial \ln L}{\partial d} & = &\frac{1-6d}{16}\\
	0 & = &\frac{1-6\hat{d}}{16} \\
	\hat{d}& = &\frac{1}{6}\\
	\hat{\nu } & = & 0.82396 \\
	L & = & 0.005208
\end{eqnarray*}

\myNewSlide
\large
You may recall that the JC distance correction from lecture 8 looked like this:
\[\nu = \frac{-3}{4}\;\ln\left(1-\frac{4p}{3}\right) \]
If you put in $p= 0.5$, because half the sites differ in our example then you the same branch length:
\[\nu  = 0.82396 \]
Our JC distance correction formula is actually an ML estimator of the branch length between a pair of taxa.

\myNewSlide
The first 30 nucleotides of the $\psi\eta$-globin gene
{\tt
\begin{table}[htdp]
\begin{center}
\begin{tabular}{lc}
gorilla & G{\color{red}A}A{\color{red}G}TCCTTGAGAAATAAACTGCACACTGG \\
orangutan & G{\color{red}G}A{\color{red}C}TCCTTGAGAAATAAACTGCACACTGG\\
\end{tabular}
\end{center}
\end{table}%
}
\vskip -2cm
\[ L = \left[\left(\frac{1}{4}\right)\left(\frac{1}{4} + \frac{3}{4}\;e^{\frac{-4\nu}{3}}\right)\right]^{28}\left[\left(\frac{1}{4}\right)\left(\frac{1}{4}-\frac{1}{4}\;e^{\frac{-4\nu}{3}}\right)\right]^2\]
\begin{picture}(0,0)(0,40)
	\put(0,0){\makebox(0,0)[l]{\includegraphics[scale=0.5]{../images/GorOrangProfile.pdf}}}
	\put(70,-20){$\hat{\nu}=0.06982$}
	\put(70,-30){$\ln L= -51.13396$}
\end{picture}

\myNewSlide
\includepdf[pages={9-12}]{../nonfreeimages/pol/pol-ML.pdf} 

\myNewSlide
\begin{picture}(0,0)(-30,200)
	\put(0,0){\makebox(0,0)[l]{\includegraphics{../images/pruning_tree.pdf}}}
	\put(0,170){\begin{tabular}{|c|c|}
\hline
Taxon & Character \\
\hline
1 & A \\
2 & C \\
3 & C \\
4 & C \\
5 & G \\
\hline
\end{tabular}
}
\end{picture}

\myNewSlide
\[L = \sum_x \sum_y \sum_z \sum_w\Pr(x,y,z,w,A,C,C,C,G|{\bm\nu}) \]
\begin{picture}(0,0)(-30,200)
	\put(0,100){\makebox(0,0)[l]{\includegraphics{../images/pruning_treeC.pdf}}}
\end{picture}

\myNewSlide
\normalsize
\begin{eqnarray*}
L & = & \sum_x \sum_y \sum_z \sum_w\Pr(x)\Pr(y|x,\nu_6)\Pr(A|y,\nu_1)\Pr(C|y,\nu_2) \cdots \\
	&& \Pr(z|x,\nu_8)\Pr(C|z,\nu_3)\Pr(w|z,\nu_7) \Pr(C|w,\nu_4)\Pr(G|w,\nu_5) 
\end{eqnarray*}
\begin{picture}(0,0)(-30,200)
	\put(0,100){\makebox(0,0)[l]{\includegraphics{../images/pruning_treeC.pdf}}}
\end{picture}

\myNewSlide
\normalsize
\begin{eqnarray*}
L & = & \sum_x \sum_y \sum_z \Pr(x)\Pr(y|x,\nu_6)\Pr(A|y,\nu_1)\Pr(C|y,\nu_2) \cdots \\
	&& \Pr(z|x,\nu_8)\Pr(C|z,\nu_3){\color{red}\Big(\sum_w\Pr(w|z,\nu_7) \Pr(C|w,\nu_4)\Pr(G|w,\nu_5) \Big)}
\end{eqnarray*}
\begin{picture}(0,0)(-30,200)
	\put(0,100){\makebox(0,0)[l]{\includegraphics{../images/pruning_treeC.pdf}}}
\end{picture}

\myNewSlide
\normalsize
\begin{eqnarray*}
L & = & \sum_x \sum_y  \Pr(x)\Pr(y|x,\nu_6)\Pr(A|y,\nu_1)\Pr(C|y,\nu_2) \cdots \\
	&& {\color{cyan}\Big(\sum_z\Pr(z|x,\nu_8)\Pr(C|z,\nu_3)}{\color{red}\Big(\sum_w\Pr(w|z,\nu_7) \Pr(C|w,\nu_4)\Pr(G|w,\nu_5) \Big)}{\color{cyan}\Big)}
\end{eqnarray*}
\begin{picture}(0,0)(-30,200)
	\put(0,100){\makebox(0,0)[l]{\includegraphics{../images/pruning_treeC.pdf}}}
\end{picture}

\myNewSlide
\normalsize
\begin{eqnarray*}
L & = & \sum_x   \Pr(x){\color{darkgreen}\Big(\sum_y\Pr(y|x,\nu_6)\Pr(A|y,\nu_1)\Pr(C|y,\nu_2)\Big)} \cdots \\
	&& {\color{cyan}\Big(\sum_z\Pr(z|x,\nu_8)\Pr(C|z,\nu_3)}{\color{red}\Big(\sum_w\Pr(w|z,\nu_7) \Pr(C|w,\nu_4)\Pr(G|w,\nu_5) \Big)}{\color{cyan}\Big)}
\end{eqnarray*}
\begin{picture}(0,0)(-30,200)
	\put(0,100){\makebox(0,0)[l]{\includegraphics{../images/pruning_treeC.pdf}}}
\end{picture}

\myNewSlide
\includepdf[pages={13-14}]{../nonfreeimages/pol/pol-ML.pdf} 

\myNewSlide
\bibliography{phylo}

\end{document}     
